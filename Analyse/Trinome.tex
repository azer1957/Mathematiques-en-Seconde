\chapter{Fonction carr\'ee \\ Fonctions trin\^omes} \label{trinome}
\minitoc

\fancyhead{} % efface les entêtes précédentes
\fancyhead[LE,RO]{\footnotesize \em \rightmark} % section en entête
\fancyhead[RE,LO]{\scriptsize \em Seconde} % classe et année en entête

    \fancyfoot{}
		\fancyfoot[RE]{\scriptsize \em \href{http://perpendiculaires.free.fr/}{http://perpendiculaires.free.fr/}}
		\fancyfoot[LO]{\scriptsize \em David ROBERT}
    \fancyfoot[LE,RO]{\textbf{\thepage}}

%\sautpage


\section{Activités}

\begin{act}[Fonction trin\^ome]
\emph{Cette activit\'e n\'ecessite l'utilisation du logiciel \href{http://www.geogebra.org/cms/}{Geogebra}.}\\
Sur Geogebra, cr\'eer trois curseurs nomm\'es $a$, $b$ et $c$ pouvant varier de $-5$ \`a $5$ selon des incr\'ements de 0,1 pour $a$ et de $1$ pour $b$ et $c$ puis, dans la zone de saisie, cr\'eer la fonction $f(x)=a*x^2+b*x+c$.
\begin{enumerate}
 \item Donner \`a $a$ la valeur 0. Qu'observe-t-on ?\\
       Pour toute la suite on prendra $a\neq0$.
 \item Donner \`a $a$ la valeur 1 et \`a $b$ et $c$ la valeur 0.
       \begin{enumerate}
        \item De quelle nature est la courbe obtenue ?
        \item Indiquer l'abscisse de son sommet et ses \'el\'ements de sym\'etrie.
        \item Donner l'expression de $f(x)$.
        \item Par lecture graphique, dresser le tableau des variations de $f$.
       \end{enumerate}
 \item Donner \`a $b$ et $c$ la valeur 0 et faire varier $a$.
       \begin{enumerate}
        \item Quel semble \^etre le \og r\^ole \fg{} de $a$ ?
	\item Dans quel cas le tableau de variations de $f$ est-il identique au pr\'ec\'edent et dans quel cas est-il diff\'erent ?
       \end{enumerate}
 \item Donner \`a $a$ la valeur 1, \`a $b$ la valeur 0 et faire varier $c$.
       \begin{enumerate}
        \item Quel semble \^etre le \og r\^ole \fg{} de $c$ ?
        \item Que peut-on dire de l'intersection de la courbe avec l'axe des ordonn\'ees ?
        \item D\'emontrer par le calcul que toute fonction de la forme $f(x)=ax^2+bx+c$ coupe l'axe des ordonn\'ees en un point dont les coordonn\'ees ne d\'ependent que de $c$.
       \end{enumerate}
 \item	On notera $x_0$ l'abscisse du sommet de la courbe.
	\begin{enumerate}
        \item Donner \`a $a$ la valeur 1, \`a $c$ la valeur 0 et faire varier $b$.\\
	Compl\'eter le tableau suivant :
	\begin{center}
	  \begin{tabular}{m{3cm}*{11}{|m{0.5cm}}}
	  \centering $b$ & $-5$ & $-4$ & $-3$ & $-2$ & $-1$ & 0 & 1 & 2 & 3 & 4 & 5 \\ \hline
	  \centering $x_0$ &&&&&&&&&&& \\
	  \end{tabular}
	\end{center}
	\item Donner \`a $a$ la valeur 2, \`a $c$ la valeur 0 et faire varier $b$.\\
	Compl\'eter le tableau suivant :
	\begin{center}
	  \begin{tabular}{m{3cm}*{11}{|m{0.5cm}}}
	  \centering $b$ & $-5$ & $-4$ & $-3$ & $-2$ & $-1$ & 0 & 1 & 2 & 3 & 4 & 5 \\ \hline
	  \centering $x_0$ &&&&&&&&&&& \\
	  \end{tabular}
	\end{center}
	\item Donner \`a $a$ la valeur $-0,5$, \`a $c$ la valeur 0 et faire varier $b$.\\
	Compl\'eter le tableau suivant :
	\begin{center}
	  \begin{tabular}{m{3cm}*{11}{|m{0.5cm}}}
	  \centering $b$ & $-5$ & $-4$ & $-3$ & $-2$ & $-1$ & 0 & 1 & 2 & 3 & 4 & 5 \\ \hline
	  \centering $x_0$ &&&&&&&&&&& \\
	  \end{tabular}
	\end{center}
	\item Faire varier $c$. Cela influence-t-il $x_0$ ?
	\item Conjecturer l'expression de $x_0$ en fonction de $a$ et $b$.\\
	      Que peut-on dire des \'el\'ements de sym\'etrie de la courbe dans tous les cas ?
       \end{enumerate}
       \item R\'egler le curseur $b$ pour que son incr\'ement soit maintenant de 0,1.\\
	     On admettra qu'un projectile lanc\'e en l'air suit une trajectoire parfaitement parabolique.\\
	     Un projectile est lanc\'e depuis une colline depuis une altitude de 400\,m symbolis\'ee par le point $A\,(0\,;\,4)$. Il doit atteindre une cible situ\'ee \`a 1\,000\,m \`a l'altitude 0, symbolis\'ee par le point $B\,(10\,;\,0)$. Pour des raisons de s\'ecurit\'e, son altitude maximum ne doit pas d\'epasser 800\,m.\\
	     D\'eterminer des valeurs de $a$, $b$ et $c$ permettant d'obtenir une courbe symbolisant la trajectoire de ce projectile et satisfaisant toutes ces conditions.
\end{enumerate}
\end{act}

\begin{act}[Forme canonique]
\emph{Cette activit\'e n\'ecessite l'utilisation du logiciel \href{http://www.geogebra.org/cms/}{Geogebra}.}\\
Sur Geogebra, cr\'eer trois curseurs nomm\'es $\alpha$, $\beta$ et $\gamma$ pouvant varier de $-5$ \`a $5$ selon des incr\'ements de 0,5  puis, dans la zone de saisie, cr\'eer la fonction $f(x)=\alpha*(x-\beta)^2+\gamma$.
\begin{enumerate}
 \item \begin{enumerate}
        \item Dans la zone de saisie, cr\'eer la fonction $g(x)=2x^2-2x+4$.
	\item D\'eterminer les valeurs de $\alpha$, $\beta$ et $\gamma$ telles que la courbe de $f$ et celle de $g$ soient confondues.
	\item V\'erifier par le calcul que les deux fonctions sont bien \'egales.
	\item Noter l'abscisse du sommet de la courbe.
	\item Par le calcul, d\'eterminer l'ensemble des solutions de l'\'equation $g(x)=0$.\\ Comment cela se traduit-il graphiquement ?
       \end{enumerate}
 \item M\^emes questions avec $g(x)=-1,5x^2-6x-4,5$.
 \item M\^emes questions avec $g(x)=-0,5x^2-2x-1,5$.
 \item \begin{enumerate}
        \item Conjecturer quelles doivent \^etre les valeurs de $\alpha$ et de $\beta$.
	\item \textbf{Par le calcul}, en utilisant la conjecture pr\'ec\'edente, d\'eterminer les valeurs de $\alpha$, $\beta$ et $\gamma$ pour que la fonction $f$ soit \'egale \`a la fonction $g(x)=2x^2-4x-1$.
	\item D\'eduire les valeurs exactes des coordonn\'ees des points d'intersection de la courbe de $g$ avec l'axe des abscisses. \\ V\'erifier si vos r\'esultats co\"incident avec la courbe de la fonction sur Geogebra.
       \end{enumerate}
\end{enumerate}
\end{act}

\section{Fonction carr\'ee}

\begin{definition}
 On appelle \emph{fonction carr\'ee} la fonction d\'efinie pour tout r\'eel $x$ par $f(x)=x^2$.\\
\end{definition}

Sa courbe repr\'esentative est une \emph{parabole} qui poss\`ede l'origine du rep\`ere comme \emph{sommet} et l'axe des ordonn\'ees comme \emph{axe de sym\'etrie}.

\begin{prop}
 La fonction carr\'ee est strictement d\'ecroissante pour $x\in]-\infty\,;\,0]$ et strictement croissante pour $x\in[0\,;\,+\infty[$.
\[\tabvar{%
\tx{x}&\tx{-\infty}&&\tx{0}&&\tx{+\infty}\cr
\tx{f(x)=x^2}&&\fd&\txb{0}&\fm&\cr
}\]
\end{prop}

\begin{proof}
 La preuve sera faite en classe.
\end{proof}

De la propri\'et\'e pr\'ec\'edente, on en d\'eduit imm\'ediatement :
\begin{prop}
 Si $a$ et $b$ positifs tels que $a<b$, alors $a^2<b^2$.\\
 Si $a$ et $b$ n\'egatifs tels que $a<b$, alors $a^2>b^2$.
\end{prop}

\section{Fonctions trin\^omes}

\begin{definition}
 Toute fonction pouvant s'\'ecrire sous la forme $f(x)=ax^2+bx+c$ o\`u $a\neq0$ est appel\'ee \emph{fonction trin\^ome}.
\end{definition}

Sa courbe est une parabole admettant le point d'abscisse $-\frac{b}{2a}$ comme sommet et la droite parall\`ele \`a l'axe des ordonn\'ees passant par ce sommet comme axe de sym\'etrie.

\begin{prop}
 Toute fonction trin\^ome peut s'\'ecrire sous la forme $f(x)=a(x-\beta)^2+\gamma$ o\`u $\beta=-\frac{b}{2a}$. Cette forme s'appelle la \emph{forme canonique} du trin\^ome.
\end{prop}

On l'admettra.

\begin{prop}
Soit $f(x)=ax^2+bx+c$ une fonction trinôme. Alors $f$ a les variations résumées dans les tableaux ci-dessous :

\begin{itemize}\vspace{-1em}\begin{multicols}{2}
	\item Si $a>0$ :
	\[\tabvar{%
\tx{x}&\tx{-\infty}&&\tx{-\frac{b}{2a}}&&\tx{+\infty}\cr
\tx{f}&\txh{+\infty}&\fd&&\fm&\txh{+\infty}\cr
}\]
	\item Si $a<0$ :
	\[\tabvar{%
\tx{x}&\tx{-\infty}&&\tx{-\frac{b}{2a}}&&\tx{+\infty}\cr
\tx{f}&\txb{-\infty}&\fm&&\fd&\txb{-\infty}\cr
}\]
\end{multicols}\end{itemize}
\end{prop}

\begin{proof}La preuve sera faite en classe.
 \end{proof}
 
 \sautpage

\section{Exercices}

\subsection{Fonction carr\'ee}

\begin{exo}
 En s'aidant \'eventuellement de la courbe de la fonction carr\'ee ou de son tableau de variation, compl\'eter par ce qu'il est possible de d\'eduire pour $x^2$ :
  \vspace{-1em}\begin{multicols}{3}\begin{enumerate}
  \item Si $x>3$ alors 
  \item Si $x<-\sqrt{2}$ alors  \dotfill
  \item Si $x>2$ alors  \dotfill
  \item Si $x<-3$ alors  \dotfill
  \item Si $x<4$ alors  \dotfill
  \item Si $x>-10$ alors  \dotfill
  \item Si $x<1$ alors  \dotfill
  \item Si $x>-5$ alors  \dotfill
 \end{enumerate}\end{multicols}\vspace{-1em}
\end{exo}

\begin{exo}
 \vspace{-1em}\begin{multicols}{2}
  \begin{enumerate}
   \item On pose : $-7 \leqslant x \leqslant 5\sqrt{2}$. \\ Compl\'eter :
	\begin{enumerate}
	  \item Si $-7 \leqslant x \leqslant 0 $ alors \dotfill $x^2$ \dotfill
	  \item Si $0 \leqslant x \leqslant 5\sqrt{2}$ alors \dotfill $x^2$ \dotfill
	  \item Donc si $-7\leqslant x \leqslant 5\sqrt{2}$ alors \dotfill $\leqslant x^2 \leqslant$ \dotfill
	\end{enumerate}
 \item Compl\'eter de la m\^eme mani\`ere :
	\begin{enumerate}
	  \item Si $-3\leqslant x \leqslant 1$ alors \dotfill $\leqslant x^2 \leqslant$ \dotfill
	  \item Si $-2\leqslant x \leqslant 3$ alors \dotfill $\leqslant x^2 \leqslant$ \dotfill
	  \item Si $-3\leqslant x \leqslant 3$ alors \dotfill $\leqslant x^2 \leqslant$ \dotfill
	\end{enumerate}
\end{enumerate}\end{multicols}\vspace{-1em}
\end{exo}

\begin{exo}
 R\'esoudre les \'equations ou in\'equations suivantes :
 \vspace{-1em}\begin{multicols}{3}\begin{enumerate}
  \item $x^2=4$ ;
  \item $x^2=5$ ;
  \item $x^2=0$ ;
  \item $x^2=-2$ ;
  \item $x^2<4$ ;
  \item $x^2\geqslant 9$ ;
  \item $x^2>-2$ ;
  \item $x^2\leqslant -3$ ;
  \item $4\leqslant x^2 \leqslant 9$ ;
\item $-1\leqslant x^2\leqslant 9$ ;
\item $0\leqslant x^2 \leqslant 8$ ;
\item $4> x^2 > 1$.
 \end{enumerate}\end{multicols}\vspace{-1em}
\end{exo}

\begin{exo}
 L'\'enonc\'e \og si $x \geqslant 2$, alors $x^2\geqslant 4$ \fg{} est appel\'e \textbf{une implication}. On dit aussi \og $x \geqslant 2$ implique $x^2\geqslant 4$ \fg{} ou bien \og $x \geqslant 2$ donc $x^2\geqslant 4$ \fg{}.
 On note \og $x \geqslant 2 \Rightarrow x^2\geqslant 4$ \fg.
\begin{enumerate}
 \item L'implication propos\'ee est-elle vraie ? Justifier.
 \item Parmi les implications suivantes, indiquer celles qui sont vraies et celles qui sont fausses.
       \vspace{-1em}\begin{multicols}{2}\begin{enumerate}
        \item $x < -1 \Rightarrow x^2>1$
	\item $x^2=4 \Rightarrow x=2$
	\item $x <0 \Rightarrow x^2<0$
	\item $x <\sqrt{3} \Rightarrow x^2<3$
	\item $x^2 = 2 \Rightarrow x=-\sqrt{2}$ ou $x=\sqrt{2}$
       \end{enumerate}\end{multicols}\vspace{-1em}
%\sautpage
 \item Traduisez par une implication les propositions suivantes :
	\begin{enumerate}
	 \item Un nombre compris entre 0 et 1 est sup\'erieur \`a son carr\'e.
	 \item Si le nombre $x$ est tel que $-1\leqslant x \leqslant 1$, alors $1-x^2$ est positif.
	 \item Un nombre sup\'erieur \`a 1 a un carr\'e sup\'erieur \`a 1.
	\end{enumerate}

\end{enumerate}
\end{exo}

\begin{exo}
 Les nombres $a$ et $b$ sont positifs.\\
 L'\'enonc\'e \og $a<b$ \'equivaut \`a $a^2<b^2$ \fg{} signifie que $a<b \Rightarrow a^2<b^2$ et que $a^2<b^2 \Rightarrow a<b$. On dit aussi \og $a<b$ si et seulement si $a^2<b^2$.\\
 On note $a<b \ssi a^2<b^2$.


Parmi les \'equivalences suivantes, indiquer celles qui sont vraies et celles qui sont fausses.
      \begin{enumerate}
       \item Pour tous r\'eels $a$ et $b$, $a<b\ssi a^2<b^2$
       \item Pour tous r\'eels n\'egatifs $a$ et $b$, $a<b \ssi a^2>b^2$
       \item Pour tous r\'eels $a$ et $b$, $a^2=b^2 \ssi a=b$ ou $a=-b$
       \item $x^2<1 \ssi x<1$
      \end{enumerate}

\end{exo}

\sautpage

\subsection{Fonctions trin\^omes}

\begin{multicols}{2}
\begin{exo}
On donne :
%\vspace{-1em}\begin{multicols}{3}
\begin{itemize}
 \item $f(x)=5-(x+1)^2$ ;
  \item $g(x)=(x-1)(2+3x)$ ;
 \item $h(x)=(x-1)(2x+1)-(x+1)$.
\end{itemize}%\end{multicols}\vspace{-1em}
\begin{enumerate}
 \item Montrer que les 3 fonctions sont des fonctions trin\^omes.
 \item Dresser leurs tableaux de variation.
 \item Indiquer les \'el\'ements de sym\'etrie de leurs courbes repr\'esentatives.
\end{enumerate}
\end{exo}

\begin{exo}
 On donne $f(x)=x^2+2x-1$.
 \begin{enumerate}
  \item Montrer que $f(x)=(x+1)^2-2$.
  \item En d\'eduire les solutions de l'\'equation $f(x)=0$.
  \item Dresser son tableau de variation en y faisant appara\^itre les solutions pr\'ec\'edentes.
  \item En d\'eduire les solutions de l'in\'equation $f(x)\leqslant 0$.
 \end{enumerate}

\end{exo}

%\sautcol

\begin{exo}
 On donne $f(x)=x^2+2x-15$ pour tout $x$.
 \begin{enumerate}
  \item Montrer que $f(x)=(x-3)(x+5)$.
  \item Montrer que $f(x)=(x+1)^2-16$.
  \item En utilsant la forme la plus adapt\'ee :
	\begin{enumerate}
	 \item R\'esoudre $f(x)=0$.
	 \item R\'esoudre $f(x)\geqslant 9$.
	\end{enumerate}
 \end{enumerate}
\end{exo}

\begin{exo}
 On donne $f(x)=x^2+2\sqrt{2} x-6$ pour tout $x$.
 \begin{enumerate}
  \item Montrer que $f(x)=(x-3\sqrt{2})(x+\sqrt{2})$.
  \item Montrer que $f(x)=(x-\sqrt{2})^2-8$.
  \item En utilsant la forme la plus adapt\'ee :
	\begin{enumerate}
	 \item R\'esoudre $f(x)=4$.
	 \item R\'esoudre $f(x)\leqslant 0$.
	\end{enumerate}
 \end{enumerate}
\end{exo}

\begin{exo}
 On donne $f(x)=2x^2+3x-2$ pour tout $x$.
 \begin{enumerate}
  \item Montrer que $f(x)=(2x-1)(x+2)$.
  \item Montrer que $f(x)=2\left(x+\frac{3}{4}\right)^2-\frac{25}{8}$.
  \item En utilsant la forme la plus adapt\'ee :
	\begin{enumerate}
	 \item R\'esoudre $f(x)=0$.
	 \item R\'esoudre $f(x)\leqslant \frac{11}{8}$.
	\end{enumerate}
 \end{enumerate}
\end{exo}

%\sautcol

\begin{exo}
 Sur le graphique ci-dessous sont trac\'ees une droite $\mathcal{D}$ et une parabole $\mathcal{P}$. Cette derni\`ere repr\'esente la fonction $f$ d\'efinie sur $\R$ par $f(x)=3-x^2$.
\begin{enumerate}
 \item \begin{enumerate}
        \item R\'esoudre l'\'equation $f(x)=0$.
	\item En d\'eduire, graphiquement, le signe de $f(x)$ en fonction de $x$.
       \end{enumerate}
 \item \begin{enumerate}
        \item D\'eterminer la fonction affine $g$ repr\'esent\'ee par $\mathcal{D}$.
	\item R\'esoudre, graphiquement, l'in\'equation $f(x)>g(x)$.
       \end{enumerate}
 \item On d\'esire retrouver par le calcul le r\'esultat pr\'ec\'edent.
       \begin{enumerate}
        \item Prouver que $f(x)>g(x)$ \'equivaut \`a $-x^2+x+2>0$.
	\item V\'erifier que $(x+1)(2-x)=-x^2+x+2$.
	\item R\'esoudre alors l'in\'equation $f(x)>g(x)$.
       \end{enumerate}

\end{enumerate}

\begin{center}\small
\def\xmin{-3.1} \def\xmax{3.1} \def\ymin{-3.1} \def\ymax{4.1}
\psset{xunit=1cm,yunit=1cm}
\begin{pspicture*}(\xmin,\ymin)(\xmax,\ymax)
\psgrid[griddots=5,gridlabels=0pt,gridwidth=.3pt, gridcolor=black, subgridwidth=.3pt, subgridcolor=black, subgriddiv=1](0,0)(\xmin,\ymin)(\xmax,\ymax)
\psaxes[labels=all,labelsep=1pt, Dx=1,Dy=1]{->}(0,0)(\xmin,\ymin)(\xmax,\ymax)
\uput[dl](0,0){$O$}
%\pcline[linewidth=1pt]{->}(0,0)(1,0) \uput[d](0.5,0){\small $\vec i$}
%\pcline[linewidth=1pt]{->}(0,0)(0,1) \uput[l](0,0.5){\small $\vec j$}
\uput[dl](\xmax,0){$x$}
\uput[ur](0,\ymax){$y$}

\psplot[plotpoints=200,algebraic=true]{\xmin}{\xmax}{3-x^2}
\psplot[plotpoints=200,algebraic=true]{\xmin}{\xmax}{-x+1}
\end{pspicture*}
\end{center}\normalsize
\end{exo}


\end{multicols}

\sautpage

\subsection{Probl\`emes}

\begin{multicols}{2}\begin{prob}
 $ABCD$ est un carr\'e de c\^ot\'e 4\,cm. $M$ est un point de $[AB]$ et $N$ un point de $[AD]$ tel que $AM=DN$. $P$ est le point tel que $AMPN$ est un rectangle.\\
 On cherche \`a trouver la position de $M$ telle que l'aire du rectangle $AMPN$ soit maximale.

On note $AM=x$ et on appelle $f(x)$ la fonction qui donne l'aire du rectangle $AMPN$ en fonction de $x$.
	\begin{enumerate}
	 \item Sur quel intervalle $f$ est-elle d\'efinie ?
	 \item Donner l'expression de $f(x)$.
	\item En d\'eduire la r\'eponse au probl\`eme.
	\end{enumerate}

%\vspace{-1em}
\begin{center}
\def\xmin{-0.75} \def\xmax{4.5} \def\ymin{-0.75} \def\ymax{4.5}
\begin{pspicture*}(\xmin,\ymin)(\xmax,\ymax)
\psline(0,0)(4,0)(4,4)(0,4)(0,0)
\psline(1,0)(1,3)(0,3)
\uput[dl](0,0){$A$}
\uput[dr](4,0){$B$}
\uput[ur](4,4){$C$}
\uput[ul](0,4){$D$}
\uput[l](0,3){$N$}
\uput[ur](1,3){$P$}
\uput[d](1,0){$M$}
\end{pspicture*}
\end{center}
\end{prob}

%\sautpage




\begin{prob}
 \emph{Ce probl\`eme n\'ecessite l'utilisation du logiciel \href{http://www.geogebra.org/cms/}{Geogebra}.}\\
 $ABC$ est un triangle rectangle isoc\`ele en $A$ tel que $AB=4$\,cm.\\
 $I$ est le milieu du segment $[BC]$ et $M$ un point variable du segment $[AB]$.\\
 On pose $AM=x$ avec $0\leqslant x \leqslant 4$.\\
 On construit les points $N$ de $[BC]$ et $P$ de $[AC]$ tels que $AMNP$ soit un rectangle.\\
 Le but du probl\`eme est de comparer les aires du rectangle $AMNP$ et du triangle $ABI$.

 \begin{enumerate}
  \item \begin{enumerate}
         \item R\'ealiser la figure sur Geogebra.
	 \item Faire afficher la longueur $AM$, puis les aires du rectangle et du triangle.
        \end{enumerate}
 \item D\'eplacer $M$.\\
       Quelle semble \^etre l'aire la plus grande ?
       Pour quelle position de $M$ les deux aires semblent-elles \'egales ?
 \item Prouver la conjecture pr\'ec\'edente.
 \end{enumerate}

\end{prob}

\begin{prob}
 \emph{Ce probl\`eme n\'ecessite l'utilisation du logiciel \href{http://www.geogebra.org/cms/}{Geogebra}.}\\
Dans un rep\`ere orthonormal, on donne le point $A\,(0\,;\,1)$ et un point $M\,(m\,;\,0)$, libre sur l'axe des abscisses. La perpendiculaire \`a $(AM)$ passant par $M$ coupe l'axe $(Oy)$ en $N$.\\
$P$ est le point tel que $OMPN$ est un rectangle.\\
Le but de l'exercice est de chercher sur quelle ligne se trouve $P$ lorsque $M$ d\'ecrit l'axe des abscisses.
\begin{enumerate}
 \item \begin{enumerate}
	\item R\'ealiser la figure sur Geogebra.
	\item Activer le mode trace pour le point $P$ (propri\'et\'es) et d\'eplacer le point $M$.
	\item Conjecturer la nature de la coubre d\'ecrite par $P$.
	\end{enumerate}
 \item \begin{enumerate}
        \item Prouver que $\widehat{OMA}=\widehat{ONM}$.
       \item Calculer $\tan \widehat{OMA}$ et $\tan \widehat{ONM}$ et en d\'eduire que $ON=m^2$.
	\item Donner les expressions, en fonction de $m$, des coordonn\'ees de $P$ et en d\'eduire que $P$ est un point de la parabole qu'\'equation $y=x^2$.
       \end{enumerate}

\end{enumerate}
\end{prob}

\begin{prob}
 Un jardinier dispose d'un terrain rectangulaire de 12\,m sur 8\,m. Il d\'esire le partager en quatre parcelles bord\'ees par deux all\'ees perpendiculaires de m\^eme largeur $x$. Il estime que l'aire des deux all\'ees doit repr\'esenter $\frac{1}{6}$ de la superficie de son terrain.\\
 Le but de ce probl\`eme est de d\'eterminer la largeur $x$ des all\'ees.
\begin{enumerate}
 \item Exprimer en fonction de $x$ l'aire des deux all\'ees.
 \item \begin{enumerate}
        \item Prouver que le probl\`eme revient \`a r\'esoudre l'\'equation $x^2-20x+16=0$.
        \item V\'erifier que $x^2-20x+16=(x-10)^2-84$.
        \item En d\'eduire $x$.
       \end{enumerate}

\end{enumerate}


\begin{center}
\psset{xunit=1cm , yunit=0.70cm}
\begin{pspicture*}(-1.2,-0.7)(5.2,3.2)
\psline(0,0)(4,0)(4,3)(0,3)(0,0)
\psline(1,0)(1,1)(0,1)
\psline(2,0)(2,1)(4,1)
\psline(4,2)(2,2)(2,3)
\psline(0,2)(1,2)(1,3)
\psline{<->}(0,-0.5)(4,-0.5)
\rput(2,-0.3){12 m}
\psline{<->}(-0.5,0)(-0.5,3)
\rput{90}(-0.7,1.5){8 m}
\psline{<->}(4.5,1)(4.5,2)
\rput(4.7,1.5){$x$}
\end{pspicture*}
\end{center}
\end{prob}

\sautcol

\begin{prob}
Une entreprise produit de la farine de blé.\\
On note $q$ le nombre de tonnes de farine fabriquée avec $0<q<80$.\\
On appelle $C(q)$ le coût total de fabrication, $R(q)$ la recette obtenue par la vente et $B(q)$ le bénéfice obtenu par la vente de $q$ tonnes de farine.
\begin{enumerate}
	\item Sachant que chaque tonne est vendue 120 \euro, exprimer $R(q)$ en fonction de $q$.
	\item Sachant que $C(q)=2q^2+10q+900$ :
		\begin{enumerate}
			\item D\'eterminer l'expression de $B(q)$.
			\item Montrer que $B(q)= -2(q-10)(q-45)$.
		\end{enumerate}
	\item Déterminer la quantité de farine à produire pour que la production soit rentable.
	\item D\'eterminer la production correspondant au bénéfice maximal et le montant de ce bénéfice.
\end{enumerate}
\end{prob}

\begin{prob}
 Le propri\'etaire d'un cin\'ema de 1\,000 places estime, pour ses calculs, qu'il vend 300 billets \`a 7\,\euro{} par s\'eance. Il a constat\'e qu'\`a chaque fois qu'il diminue le prix du billet de 0,1\,\euro{}, il vend 10 billets de plus.\\
Il engage une campagne de promotion.
\begin{enumerate}
 \item Il d\'ecide de vendre le billet 5\,\euro{}.
	\begin{enumerate}
	 \item Combien y aura-t-il de spectateurs pour une s\'eance ?
	  \item Quelle est alors la recette pour une s\'eance ?
	\end{enumerate}
\item \`A quel prix devrait-il vendre le billet pour remplir la salle ? Commenter.
\item Le propri\'etaire envisage de proposer $x$ r\'eductions de 0,1\,\euro{}.
      \begin{enumerate}
       \item Quel est alors le prix d'un billet en fonction de $x$ ?
       \item Exprimer en fonction de $x$ la recette, not\'ee $r(x)$, pour une s\'eance et v\'erifier que $r(x)=-x^2+40x+2100$.
        \item En d\'eduire la recette maximale, le prix du billet et le nombre de spectateurs \`a cette s\'eance.
      \end{enumerate}

\end{enumerate}

\end{prob}

\sautcol

\begin{prob}
Une société de livres par correspondance a actuellement $10\,000$ abonnés qui paient, chacun, 50 \euro{} par an.
Une étude a montré que chaque fois qu'on augmente d'1\,\euro{} le prix de l'abonnement annuel, cela entraîne une diminution de 100 abonnés et chaque fois qu'on baisse d'1\,\euro{} le prix de l'abonnement annuel, cela entra\^ine une augmentation de 100 abonn\'es.\\
On se propose de trouver comment modifier le prix de l'abonnement annuel pour obtenir le maximum de recette.\\
$n$ désigne la variation du prix de l'abonnement annuel en euros ($n$ est un entier relatif).
\begin{enumerate}
	\item Exprimer en fonction de $n$ le prix de l'abonnement annuel, et le nombre d'abonnés correspondant.
	\item Exprimer en fonction de $n$ la recette annuelle de cette socité, notée $R(n)$.
	\item Déterminer la valeur de $n$ pour laquelle $R(n)$ est maximum.\\
	Quel est alors le montant de l'abonnement annuel, le nombre d'abonnés et la recette totale correspondante ?
\end{enumerate}
\end{prob}

\begin{prob}
Une zone de baignade rectangulaire est délimitée par une corde (agrémentée de bouées) de longueur 50 m. Quelles doivent être les dimensions de la zone pour que la surface soit maximale ?
\begin{center}
\psset{xunit=1cm , yunit=0.8cm}
\begin{pspicture*}(-0.7,-0.7)(5.7,2.1)
\psline(-0.5,0)(5.5,0)
\rput(2.5,-0.5){\small \em plage}
\psline(1,0)(1,1.5)(4,1.5)(4,0)
\rput(2.5,0.75){\small \em zone de baignade}
\end{pspicture*}
\end{center}
\end{prob}





\end{multicols}
