\chapter{\'Equations de droites -- Fonctions affines} \label{fctref}
\minitoc

\fancyhead{} % efface les entêtes précédentes
\fancyhead[LE,RO]{\footnotesize \em \rightmark} % section en entête
\fancyhead[RE,LO]{\scriptsize \em Seconde} % classe et année en entête

    \fancyfoot{}
		\fancyfoot[RE]{\scriptsize \em \href{http://perpendiculaires.free.fr/}{http://perpendiculaires.free.fr/}}
		\fancyfoot[LO]{\scriptsize \em David ROBERT}
    \fancyfoot[LE,RO]{\textbf{\thepage}}

%\sautpage


\section{Activités}

\begin{act}
 Le plan est muni d'un rep\`ere.
 On cherche \`a repr\'esenter tous les points $M\,(x\,;\,y)$ du plan dont les coordonn\'ees v\'erifient $y=2x-3$.
\begin{enumerate}
 \item Compl\'eter le tableau suivant :
    \begin{center}
\begin{tabular}{c*{12}{|>{\centering}m{0.75cm}}}
     $x$ & 0 & 1 & 2 &   &   &   & $-1$ & $-2$ & $-3$ &&& \tabularnewline \hline
     $y$ &   &   &   & 0 & 1 & 2 &      &      &      & $-1$ & $-2$ & $-3$
    \end{tabular}                 \end{center}
 \item Placer les points correspondants dans un rep\`ere.
 \item Que constate-t-on, g\'eom\'etriquement ?
 \item Prendre un autre point ayant la m\^eme caract\'eristique g\'eom\'etrique. Ses coordonn\'ees v\'erifient-elles l'\'e\-qua\-tion $y=2x-3$ ?
 \item Prendre un point n'ayant pas la m\^eme caract\'eristique g\'eom\'etrique. Ses coordonn\'ees v\'erifient-elles l'\'e\-qua\-tion $y=2x-3$ ?
\end{enumerate}

\end{act}


%%%%%%%%%%%%%%%%%%%%%%%%%%%%%%%%%%%%%%%%%%%%%%%%%%%%%%%

\begin{act}
Soient $\mathcal{D}_1$, $\mathcal{D}_2$, $\mathcal{D}_3$, $\mathcal{D}_4$, $\mathcal{D}_5$ et $\mathcal{D}_6$ les droites d'équations :
\vspace{-1em}\begin{multicols}{3}\begin{itemize}
	\item $\mathcal{D}_1 : y=3x+1$ ;
	\item $\mathcal{D}_2 : y=1x+1$ ;
	\item $\mathcal{D}_3 : y=0,25x+1$ ;
		\item $\mathcal{D}_4 : y=0x+1$ ;
			\item $\mathcal{D}_5 : y=-x+1$ ;
				\item $\mathcal{D}_6 : y=-2x+1$ ;
\end{itemize}\end{multicols}\vspace{-1em}
\begin{enumerate}
	\item Montrer que le point $(0\,;\,1)$ appartient à toutes ces droites.
	\item Déterminer, pour chacune de ces droites, un autre point lui appartenant.
	\item Placer ces points dans un repère orthogonal puis tracer les droites.
	\item Quelle semble être l'influence du coefficient du $x$ sur \og l'allure \fg{} de ces droites ?
\end{enumerate}
\end{act}

%%%%%%%%%%%%%%%%%%%%%%%%%%%%%%%%%%%%%%%%%%%%%%%%%%%%%%%%

\sautpage

\begin{act}
Trois taxis $T_1$, $T_2$ et $T_3$ proposent les tarifs suivants :
\begin{itemize}
	\item $T_1$ : 5 \euro{} de prise en charge, puis 0,40 \euro{} du kilomètre ;
	\item $T_2$ : 4 \euro{} de prise en charge, puis 0,50 \euro{} du kilomètre ;
	\item $T_3$ : 7 \euro{} de prise en charge, puis 0,30 \euro{} du kilomètre ;
\end{itemize}
\begin{enumerate}
	\item Quel est le taxi le plus économique pour un trajet de
\vspace{-1em}\begin{multicols}{3}\begin{itemize}
	\item 5 km ?
	\item 10 km ?
	\item 15 km ?
\end{itemize}\end{multicols}\vspace{-1em}
	\item On note $x$ la distance que veut parcourir un client en taxi. Exprimer les tarifs $f_1(x)$, $f_2(x)$ et $f_3(x)$ des taxis $T_1$, $T_2$ et $T_3$ en fonction de $x$.
	\item Représenter dans le repère ci-dessous les courbes $\mathcal{C}_1$, $\mathcal{C}_2$ et $\mathcal{C}_3$ des fonctions $f_1$, $f_2$ et $f_3$.
	\begin{center}\small
\def\xmin{-1.1} \def\xmax{25.6} \def\ymin{-1.1} \def\ymax{15.6}
\psset{xunit=0.5cm,yunit=0.5cm}
\begin{pspicture*}(\xmin,\ymin)(\xmax,\ymax)
\psset{xunit=0.25cm,yunit=0.25cm}
\psgrid[griddots=5,gridlabels=0pt,gridwidth=.3pt, gridcolor=black, subgridwidth=.3pt, subgridcolor=black, subgriddiv=1](0,0)(-2,-2)(51,31)
\psset{xunit=0.5cm,yunit=0.5cm}
\psaxes[labels=all,labelsep=1pt, Dx=1,Dy=1]{->}(0,0)(-0.5,-0.5)(\xmax,\ymax)
\uput[dl](0,0){$O$}
%\pcline[linewidth=1pt]{->}(0,0)(1,0) \uput[d](0.5,0){\small $\vec i$}
%\pcline[linewidth=1pt]{->}(0,0)(0,1) \uput[l](0,0.5){\small $\vec j$}
\uput[dl](\xmax,0){$x$}
\uput[ur](0,\ymax){$y$}

%\psplot[plotpoints=200,algebraic=true]{\xmin}{\xmax}{}
\end{pspicture*}
\end{center}\normalsize
\item En vous basant sur le graphique, indiquez pour quelles distances il est plus économique de prendre le taxi $T_1$, la taxi $T_2$ ou le taxi $T_3$. \emph{On donnera les réponses sous forme d'intervalle}.
\item Un client désire faire plus de 20 km, et choisira le taxi $T_3$. Il vous charge étudier le coût de son trajet en fonction de la distance $x$.
\begin{enumerate}
	\item Compléter le tableau ci-dessous :

\begin{center}
\begin{tabular}{c*{9}{|>{\centering}m{1cm}}}
Distance $x$ & 20 &21 & 22 & 23 & 24 & 25 & 30 & 40 & 50  \tabularnewline \hline
Coût $f_3(x)$&	&  &    &     &    &   & & & \tabularnewline
\end{tabular}
\end{center}
\item La distance et le coût sont-ils des grandeurs proportionnelles ?
\item À l'aide du tableau précédent conjecturer de combien augmente le coût lorsque la dis\-tan\-ce augmente de :
\vspace{-1em}\begin{multicols}{3}\begin{itemize}
	\item 1 km
	\item 2 km
	\item 5 km
\end{itemize}\end{multicols}\vspace{-1em}
\item Que peut-on dire alors des grandeurs \og augmentation de la distance \fg{} et \og augmentation du coût \fg{} ?
\end{enumerate}
\end{enumerate}
\end{act}
\sautpage
\section{Bilan et compléments}

\subsection{\'Equations de droites}
Le plan est muni d'un repère.

\begin{theo}
Toute droite $\mathcal{D}$ non parallèle à l'axe des ordonnées est caractérisée par une relation de la forme $y=mx+p$, où $m$ et $p$ sont deux nombres réels constants.

Cela signifie que :
\begin{itemize}
	\item tout point $M$ appartenant à la droite $\mathcal{D}$ a ses coordonnées $(x\,;\,y)$ qui vérifient l'équation $y=mx+p$ ;
	\item tout point $M$ dont les coordonnées $(x\,;\,y)$ vérifient l'équation $y=mx+p$ appartient à la droite $\mathcal{D}$.
\end{itemize}
On dit que $y=mx+p$ est \emph{l'équation réduite} de $\mathcal{D}$.
\vskip 1em
Toute droite $\mathcal{D}$ parallèle à l'axe des ordonnées est caractérisée par une relation de la forme $x=k$, où $k$ est un nombre réel constant.
\end{theo}

On l'admettra.


\begin{prop}\label{eqdteprop1}
Soit $\mathcal{D}$ une droite d'équation réduite $y=mx+p$, donc non parallèle à l'axe des ordonnées, et $A\,(x_A\,;\,y_A)$ et $B\,(x_B\,;\,y_B)$ deux points distincts quelconques de $\mathcal{D}$.

Alors on a :
\begin{itemize}
	\item $m=\frac{y_B-y_A}{x_B-x_A}=\frac{\Delta y}{\Delta x}$ et ce nombre est appel\'e \emph{coefficient directeur} de la droite $\mathcal{D}$ ;
	\item le point de coordonnées $(0\,;\,p)$ appartient à $\mathcal{D}$ et le nombre $p$ est appel\'e \emph{ordonn\'ee \`a l'origine} de la droite $\mathcal{D}$.
\end{itemize}
\end{prop}

\begin{proof} On admettra le premier point. \\ Par ailleurs, le point d'abscisse 0 appartenant à $\mathcal{D}$ a pour ordonnée $y=m\times 0 + p=p$.
\end{proof}

\vspace{-1em}\begin{multicols}{2}
\begin{prop}\label{eqdteprop2}
Soit $\mathcal{D}$ une droite d'équation réduite $y=mx+p$, donc non parallèle à l'axe des ordonnées, et $A\,(a\,;\,b)$ un point quelconque de $\mathcal{D}$.\\
Alors le point $B\,(a+1\,;\,b+m)$ appartient à $\mathcal{D}$.
\end{prop}

\begin{center}
\psset{xunit=1cm,yunit=1cm}
\def\xmin{-2.1} \def\xmax{4.6} \def\ymin{0.6} \def\ymax{3.6}
\begin{pspicture*}(\xmin,\ymin)(\xmax,\ymax)
%\psset{xunit=1cm,yunit=1cm}
\psgrid[griddots=7,gridlabels=0pt,gridwidth=.3pt, gridcolor=black, subgridwidth=.3pt, subgridcolor=black, subgriddiv=1](0,0)(\xmin,\ymin)(\xmax,\ymax)
%\psset{xunit=1cm,yunit=1cm}
%\psaxes[labels=all,labelsep=1pt, Dx=1,Dy=1]{-}(0,0)(\xmin,\ymin)(\xmax,\ymax)
%\uput[dl](0,0){$O$}
%\pcline[linewidth=1pt]{->}(0,0)(1,0) \uput[d](0.5,0){\small $\vec \imath$}
%\pcline[linewidth=1pt]{->}(0,0)(0,1) \uput[l](0,0.5){\small $\vec \jmath$}

\psplot[algebraic=true]{\xmin}{\xmax}{2*x+1}
\psdots(0,1)(1,3)
\uput[ul](0,1){$A\,(a\,;\,b)$}
\uput[ul](1,3){$B\,(a+1\,;\,b+m)$}
\psline[linestyle=dashed]{->}(0,1)(1,1)
\psline[linestyle=dashed]{->}(1,1)(1,3)
\uput[d](0.5,1){$1$}
\uput[r](1,2){$m$}
\end{pspicture*}
\end{center}
\end{multicols}\vspace{-1em}

\begin{proof}
Soit $\mathcal{D}$ une droite d'équation réduite $y=mx+p$ et $A\,(a\,;\,b)$ un point de $\mathcal{D}$ et $B\,(a+1\,;\,b+m)$.\\
Montrons que $B$ appartient à $\mathcal{D}$.\\
On sait que $A\,(a\,;\,b)\in\mathcal{D}$ donc $b=m\times a + p$.\\
Cherchons l'ordonn\'ee du point de la droite donc l'abscisse est $a+1$.\\
On sait que $y=m(a+1)+p=m\times a + m + p = b+m$.\\
Donc $B$ appartient à $\mathcal{D}$.
\end{proof}
\begin{prop}
Soit $\mathcal{D}$ et $\mathcal{D}'$ deux droites d'équations réduites respectives $y=mx+p$ et $y=m'x+p'$, donc non parallèles à l'axe des ordonnées.

$\mathcal{D}$ et $\mathcal{D}'$ sont parallèles (éventuellement confondues) si et seulement si $m=m'$.
\end{prop}

On l'admettra.

\subsection{Fonctions affines}
\begin{definition}
Les fonctions $f$, définies sur $\R$, dont l'expression peut se mettre sous la forme \[f(x)=mx+p \text{ où $m$ et $p$ sont des réels}\] sont appelées \emph{fonctions affines}.

Cas particuliers :
\begin{itemize}
	\item si $m=0$ alors $f(x)=p$ est dite \emph{constante} ;
	\item si $p=0$ alors $f(x)=mx$ est dite \emph{linéaire}.
\end{itemize}
\end{definition}

\begin{prop}
La représentation graphique d'une fonction affine dans un repère est une droite.

Celle d'une fonction linéaire est une droite passant par l'origine du repère.
\end{prop}

\begin{prop}
Soit $f$ une fonction déinie sur $\R$.
\begin{itemize}
	\item Si les variations des $x$ et des $f(x)$ sont proportionnelles, alors $f$ est une fonction affine.
	\item Réciproquement, si $f$ est une fonction affine, alors les variations des $x$ et des $f(x)$ sont proportionnelles.
\end{itemize}
Dit autrement, on a :

$\frac{\Delta f(x)}{\Delta x}=$constante $\ssi f$ est une fonction affine.

Ou encore :

Pour tout $x$ et $x'$, $\frac{f(x)-f(x')}{x-x'}=$constante $\ssi f$ est une fonction affine
\end{prop}

On l'admettra.

\begin{prop}
Soit $f : x \mapsto mx+p$ une fonction affine.
\begin{itemize}
	\item Si $m>0$ alors $f$ est strictement croissante sur $\R$.
	\item Si $m<0$ alors $f$ est strictement décroissante sur $\R$.
	\item Si $m=0$ alors $f$ est constante sur $\R$.
\end{itemize}
\end{prop}

%\sautpage

La preuve sera faite en classe.

\begin{prop}
Soit $f : x \mapsto mx+p$ une fonction affine avec $m\neq0$. Alors :
\begin{enumerate}
	\item $f(x)=0$ pour $x_0=-\frac{p}{m}$ et
	\item Le signe de $f(x)$ selon les valeurs de $x$ est donné par le tableau suivant :

	\begin{itemize}
	\item Si $m>0$
\begin{center}
\begin{tabular}{c|*{5}{>{\centering}m{0.35cm}}}
$x$										& $-\infty$ &			& $x_0$	&			& $+\infty$ \tabularnewline \hline
Signe de $f(x)$	&						& $-$ & $0$		& $+$ & \tabularnewline
\end{tabular}
\end{center}
\item Si $m<0$
\begin{center}
\begin{tabular}{c|*{5}{>{\centering}m{0.35cm}}}
$x$										& $-\infty$ &			& $x_0$	&			& $+\infty$ \tabularnewline \hline
Signe de $f(x)$	&						& $+$ & $0$		& $-$ & \tabularnewline
\end{tabular}
\end{center}
\end{itemize}
\end{enumerate}
\end{prop}
%\sautpage
La preuve sera faite en classe.

\sautpage

\section{Exercices}

\subsection{\'Equations de droites}
\begin{exo}
La droite $\mathcal{D}$ est d'équation réduite $y=-3x+0,5$.\\ Déterminer si $A\,(150,5\,;\,-451)$ ou $B\,(-73,25\,;\,219,5)$ appartiennent à $\mathcal{D}$.
\end{exo}

\begin{exo}
Dans chacun des cas suivants, dire si le point $A$ appartient à la droite $\mathcal{D}$ :
\vspace{-1em}\begin{multicols}{2}\begin{enumerate}
	\item $A\,\left(\frac{1}{3}\,;\,\frac{13}{6}\right)$ et $\mathcal{D}\,:\, y=6x+\frac{1}{6}$ ;
	\item $A\,(1\,;\,-7)$ et $\mathcal{D}\,:\, y=-\frac{3}{4}(x+2)-5$ ;%\sautcol
	\item $A\,(2\,;\,5)$ et $\mathcal{D}\,:\, x=5$ ;%\sautcol
	\item $A\,\left(\frac{1}{3}\,;\,\frac{1}{6}\right)$ et $\mathcal{D}\,:\, y=\frac{1}{6}$.
\end{enumerate}\end{multicols}\vspace{-1em}
\end{exo}

\begin{exo}
La droite $\mathcal{D}$ est d'équation réduite : $y=\frac{5}{2}x-1$.
\begin{enumerate}
	\item $A$ est le point de $\mathcal{D}$ d'abscisse 12. Quelle est son ordonnée ?
	\item $B$ est le point de $\mathcal{D}$ d'ordonnée $-\frac{1}{2}$. Quelle est son abscisse ?
\end{enumerate}
\end{exo}

\begin{exo}
Dans un même repère, tracer les droites dont les équations sont les suivantes :
\vspace{-1em}\begin{multicols}{3}\begin{itemize}
	\item $\mathcal{D}_1:y=-\frac{1}{2}x+5$ ;
	\item $\mathcal{D}_2:y=4x-2$ ;%\sautcol
	\item $\mathcal{D}_3:y=-3$ ;%\sautcol
	\item $\mathcal{D}_4:y=\frac{3}{4}x-4$ ;%\sautcol
	\item $\mathcal{D}_5:x=6$.
\end{itemize}\end{multicols}\vspace{-1em}
\end{exo}

\begin{exo}
Dans un même repère, tracer les droites dont les équations sont les suivantes :
\vspace{-1em}\begin{multicols}{3}\begin{itemize}
	\item $\mathcal{D}_1:y=-5x+10$ ;
	\item $\mathcal{D}_2:y=6x-14$ ;%\sautcol
	\item $\mathcal{D}_3:y=\frac{3x-1}{6}$ ;%\sautcol
	\item $\mathcal{D}_4:y=\frac{-2x+1}{4}$ ;%\sautcol
	\item $\mathcal{D}_5:2x-5y=3$.
\end{itemize}\end{multicols}\vspace{-1em}
\end{exo}

%\sautpage

\begin{exo}
Dans un même repère, tracer les droites suivantes :
%\vspace{-1em}\begin{multicols}{2}
\begin{itemize}
	\item $\mathcal{D}_1$ passant par $A\,(3\,;\,1)$ et de coefficient directeur $-1$ ;
	\item $\mathcal{D}_2$ passant par $B\,(-3\,;\,2)$ et de coefficient directeur $-\frac{1}{4}$ ;
	\item $\mathcal{D}_3$ passant par $C\,(1\,;\,0)$ et de coefficient directeur $3$ ;
	\item $\mathcal{D}_4$ passant par $D\,(0\,;\,2)$ et de coefficient directeur $\frac{4}{3}$ ;
	\item $\mathcal{D}_5$ passant par $E\,(-2\,;\,2)$ et de coefficient directeur $0$.
\end{itemize}%\end{multicols}\vspace{-1em}
\end{exo}

%\sautpage

\begin{exo}
Dans chacun des cas suivants, déterminer l'équation de la droite $(AB)$ :
\vspace{-1em}\begin{multicols}{3}\begin{enumerate}
	\item $A\,(1\,;\,2)$ et $B\,(3\,;\,-1)$ ;
	\item $A\,(4\,;\,4)$ et $B\,(-1\,;\,2)$ ;
	\item $A\,(0\,;\,-1)$ et $B\,(2\,;\,3)$ ;
	\item $A\,(-2\,;\,2)$ et $B\,(3\,;\,2)$ ;
	\item $A\,(1\,;\,3)$ et $B\,(1\,;\,4)$.
\end{enumerate}\end{multicols}\vspace{-1em}
\end{exo}

\begin{exo}
 \begin{enumerate}
  \item D\'eterminer l'\'equation r\'eduite de la droite $\mathcal{D}$ sachant que $A\,(2\,;\,1)\in\mathcal{D}$ et que $\mathcal{D}$ est parall\`ele \`a la droite $\mathcal{D}' : y=3x-1$.
 \item \begin{enumerate}
        \item On donne $A\,(2\,;\,3)$, $B\,(-1\,;\,2)$, $C\,(4\,;\,1)$ et $D\,(-2\,;\,5)$. Les droites $(AB)$ et $(CD)$ sont-elles parall\`eles ?
	\item M\^eme question avec $A\,(1\,;\,2)$, $B\,(2\,;\,-4)$, $C\,(0\,;\,3)$ et $D\,(1\,;\,-3)$.
       \end{enumerate}
 \item D\'eterminer l'ordonn\'ee du point $A$ sachant que :
  \begin{itemize}
   \item $A\in\mathcal{D}$ ;
   \item $\mathcal{D}$ parall\`ele \`a la droite $(BC)$ o\`u $B\,(2\,;\,1)$ et $C\,(-1\,;\,3)$ ;
   \item l'abscisse de $A$ vaut 8
  \end{itemize}

 \end{enumerate}

\end{exo}


\sautpage

\begin{exo}
Déterminer graphiquement les équations réduites des droites représentées sur le schéma suivant :
\begin{center}
\psset{xunit=1cm,yunit=1cm}
\def\xmin{-4.6} \def\xmax{4.6} \def\ymin{-4.6} \def\ymax{4.6}
\begin{pspicture*}(\xmin,\ymin)(\xmax,\ymax)
%\psset{xunit=1cm,yunit=1cm}
\psgrid[griddots=7,gridlabels=0pt,gridwidth=.3pt, gridcolor=black, subgridwidth=.3pt, subgridcolor=black, subgriddiv=1](0,0)(\xmin,\ymin)(\xmax,\ymax)
%\psset{xunit=1cm,yunit=1cm}
\psaxes[labels=all,labelsep=1pt, Dx=1,Dy=1]{-}(0,0)(\xmin,\ymin)(\xmax,\ymax)
\uput[dl](0,0){$O$}
\pcline[linewidth=1pt]{->}(0,0)(1,0) \uput[d](0.5,0){\small $\vec \imath$}
\pcline[linewidth=1pt]{->}(0,0)(0,1) \uput[l](0,0.5){\small $\vec \jmath$}
\psplot[algebraic=true]{\xmin}{\xmax}{-2*x+3}
\uput[ur](3,-3){$\mathcal{D}_1$}
\psplot[algebraic=true]{\xmin}{\xmax}{x+1}
\uput[ul](3,4){$\mathcal{D}_2$}
\psplot[algebraic=true]{\xmin}{\xmax}{x/3-1}
\uput[u](3,0){$\mathcal{D}_3$}
\psplot[algebraic=true]{\xmin}{\xmax}{-x/4+3}
\uput[dl](-4,4){$\mathcal{D}_4$}
\psline(\xmin,-2)(\xmax,-2)
\uput[ur](4,-2){$\mathcal{D}_5$}
\psline(-3,\ymin)(-3,\ymax)
\uput[ur](-3,4){$\mathcal{D}_6$}
\psdots[dotstyle=x](-4,4)(0,3)(4,2)(2,-1)(1,1)(3,-3)(-4,-3)(-3,-2)(-2,-1)(-1,0)(0,1)(1,2)(2,3)(3,4)(0,-1)(3,0)
\end{pspicture*}
\end{center}
\end{exo}

%\sautpage

\begin{exo}
Déterminer graphiquement les équations réduites des droites représentées sur le schéma suivant :
\begin{center}
\psset{xunit=0.5cm,yunit=0.5cm}
\def\xmin{-9.1} \def\xmax{9.1} \def\ymin{-9.1} \def\ymax{9.1}
\begin{pspicture*}(\xmin,\ymin)(\xmax,\ymax)
%\psset{xunit=1cm,yunit=1cm}
\psgrid[griddots=7,gridlabels=0pt,gridwidth=.3pt, gridcolor=black, subgridwidth=.3pt, subgridcolor=black, subgriddiv=1](0,0)(\xmin,\ymin)(\xmax,\ymax)
%\psset{xunit=1cm,yunit=1cm}
\psaxes[labels=all,labelsep=1pt, Dx=5,Dy=5]{-}(0,0)(\xmin,\ymin)(\xmax,\ymax)
\uput[dl](0,0){$O$}
\pcline[linewidth=1pt]{->}(0,0)(1,0) \uput[d](0.5,0){\small $\vec \imath$}
\pcline[linewidth=1pt]{->}(0,0)(0,1) \uput[l](0,0.5){\small $\vec \jmath$}
\psplot[algebraic=true]{\xmin}{\xmax}{-5*x+10}
\uput[ur](1,5){$\mathcal{D}_1$}
\psplot[algebraic=true]{\xmin}{\xmax}{2*x/3+10}
\uput[ul](-6,6){$\mathcal{D}_2$}
\psplot[algebraic=true]{\xmin}{\xmax}{3*x/4-10}
\uput[ul](8,-4){$\mathcal{D}_3$}
\psplot[algebraic=true]{\xmin}{\xmax}{2*x-11}
\uput[ul](6,1){$\mathcal{D}_4$}
\psdots[dotstyle=x](1,5)(2,0)(3,-5)(-9,4)(-6,6)(-3,8)(4,-7)(8,-4)(1,-9)(2,-7)(4,-3)(5,-1)(6,1)(7,3)(8,5)(9,7)
\end{pspicture*}
\end{center}
\end{exo}
\sautpage


\subsection{Fonctions affines}

\begin{exo}
Voici les tarifs pratiqués par deux agences de location de voitures pour des véhicules identiques (tarifs journaliers, assurance comprise) :
%\vspace{-1em}\begin{multicols}{2}
\begin{itemize}
	\item agence A : Forfait de 50 \euro{} plus 0,42 \euro{} par km ;
	\item agence B : Forfait de 40 \euro{} plus 0,50 \euro{} par km.
\end{itemize}%\end{multicols}\vspace{-1em}
\begin{enumerate}
	\item Quelle est l'agence la plus économique selon que l'on désire faire un parcours de
\vspace{-1em}\begin{multicols}{3}
\begin{itemize}
	\item 50 km ?
	\item 150 km ?
	\item 300 km ?
\end{itemize}
\end{multicols}\vspace{-1em}
\item On appelle $x$ la distance que l'on désire parcourir. Déterminer selon les valeurs de $x$ l'agence la plus économique.
\item \'Ecrire un algorithme prenant comme argument la distance \`a parcourir et indiquant quelle agence est la plus int\'eressante pour cette distance ainsi que le tarif \`a payer.
\end{enumerate}
\end{exo}

\begin{exo}
Les tarifs mensuels d'un abonnement pour un téléphone mobile sont les suivants : Forfait d'une heure 15 \euro{} plus 0,30 \euro{} par minute supplémentaire.
\begin{enumerate}
	\item Compléter le tableau suivant, où la durée est la durée totale des communications du mois en minute et le coût est le montant final de la facture en euros :
\begin{center}
\begin{tabular}{c*{3}{|>{\centering}m{1cm}}}
Durée & 45& 80& 120 \tabularnewline \hline
Coût &&&\tabularnewline
\end{tabular}
\end{center}
\item Existe-t-il une fonction affine $f$ qui à une durée de communication $x$ associe le coût $f(x)$ ?
\end{enumerate}
\end{exo}



\begin{exo}
Soit $f$ une fonction affine. Déterminer l'expression de $f$ dans chacun des cas suivants :
\vspace{-1em}\begin{multicols}{2}
\begin{enumerate}
	\item $f(1)=2$ et $f(4)=8$
	\item $f(-1)=4$ et $f(2)=3$
	\item $f(5)=-1$ et $f(3)=3$
	\item $f(-4)=5$ et $f(1)=7$
\end{enumerate}
\end{multicols}\vspace{-1em}
\end{exo}

\begin{exo}
 Repr\'esenter dans un m\^eme rep\`ere les courbes des fonctions affines suivantes :
 \vspace{-1em}\begin{multicols}{3}\begin{itemize}
  \item $f(x)=2x-3$ ;
  \item $g(x)=2x+1$ ; \sautcol
  \item $h(x)=\frac{1}{3}x+3$ ;
  \item $i(x)=-x+4$ ; \sautcol
  \item $j(x)=-\frac{3}{4}x+5$ ;
  \item $k(x)=-3x+5$.
 \end{itemize}\end{multicols}\vspace{-1em}
\end{exo}

%\sautpage

\begin{exo}
 Dans le rep\`ere ci-dessous on a repr\'esent\'e les courbes de quatre fonctions affines $f$, $g$, $h$ et $i$.\\
 D\'eterminer leurs expressions respectives.
 \vspace{-1em}\begin{center}
\psset{xunit=1cm , yunit=0.5cm}
\def\xmin{-5} \def\xmax{8} \def\ymin{-5} \def\ymax{8}
\begin{pspicture*}(\xmin,\ymin)(\xmax,\ymax)
%\psset{xunit=1cm,yunit=1cm}
\psgrid[griddots=10,gridlabels=0pt,gridwidth=.3pt, gridcolor=black, subgridwidth=.3pt, subgridcolor=black, subgriddiv=1](0,0)(\xmin,\ymin)(\xmax,\ymax)
\psaxes[labels=all,labelsep=1pt, Dx=1,Dy=1]{->}(0,0)(\xmin,\ymin)(\xmax,\ymax)
\psplot[algebraic=true,plotpoints=200]{\xmin}{\xmax}{3*x-4}
\uput[ul](3,5){$\mathcal{C}_f$}
\psplot[algebraic=true,plotpoints=200]{\xmin}{\xmax}{-2*x/3+2}
\uput[ur](-3,4){$\mathcal{C}_g$}
\psplot[algebraic=true,plotpoints=200]{\xmin}{\xmax}{x+1}
\uput[dr](5,6){$\mathcal{C}_h$}
\psplot[algebraic=true,plotpoints=200]{\xmin}{\xmax}{x/2-2}
\uput[ul](6,1){$\mathcal{C}_i$}
\end{pspicture*}          \end{center}
\end{exo}

\sautpage

\section{Probl\`emes}

\begin{multicols}{2}
 


\begin{prob}
On donne $P(x)=(2x+1)(-x+2)$.
\begin{enumerate}
 \item \label{signeq1}
	\begin{enumerate}
	\item \'Etudier le signe de $2x+1$ selon les valeurs de $x$.
	\item \'Etudier le signe de $-x+2$ selon les valeurs de $x$.
	\item En d\'eduire le signe de $P(x)$ selon les valeurs de $x$.
	\emph{On pourra étudier le signe de chacun des facteurs et faire un tableau de signes.}
	\item En d\'eduire l'ensemble des solutions de l'in\'equation $P(x)<0$.
       \end{enumerate} 
  \item \'Etudier le signe, selon les valeurs de $x$, de chacune des fonctions suivantes :
	      \begin{itemize}
		\item $Q(x)=(-2x+1)(-3x+4)$ ;
		\item $R(x)=(-x+4)(5-2x)$ ;
		\item $T(x)=(x-1)(-2x+4)(2x-1)$.
	       \end{itemize}
	  \item R\'esoudre les in\'equations suivantes :
	      \begin{itemize}
		\item $S(x)\geqslant 0$ sachant que $S(x)=(2x+3)(x-1)$ ;
		\item $U(x)\leqslant 0$ sachant que $U(x)=(4-x)(x+1)(2x+2)$.
	      \end{itemize}

\end{enumerate}
\end{prob}


\begin{prob}
On donne $f(x)=(3x+4)(x-4)-(2x-3)(3x+4)$.\\
Résoudre $f(x)>0$. \emph{On pourra commencer par factoriser.}
\end{prob}

\begin{prob}
 On donne $g(x)=(2x+1) - (3x-1)$. R\'esoudre $g(x)\leqslant0$.
\end{prob}


\begin{prob}
Résoudre les inéquations suivantes :
\vspace{-1em}\begin{multicols}{3}
\begin{enumerate}
	\item $x\leqslant x^2$
	\item $\frac{1}{x}\leqslant x$
	\item $x^3\leqslant x^2$
\end{enumerate}
\end{multicols}
\end{prob}

\begin{prob}
 On donne $P(x)=2x^3+3x^2-2x-3$.
\begin{enumerate}
 \item Montrer que $P(x)=(x^2-1)(2x+3)$.
 \item En d\'eduire l'ensemble des solutions de l'in\'equation $P(x)\geqslant0$.
\end{enumerate}

\end{prob}



%\sautpage


%\section{Probl\`emes}

\begin{prob}
%\begin{multicols}{2}
 $ABCD$ est un carr\'e de centre $O$. \\$E$ est le sym\'etrique de $D$ par rapport \`a $C$. \\$Q$ est l'intersection des droites $(OE)$ et $(BC)$. \\$P$ est le milieu du segment $[BE]$.\\
 On se place dans le rep\`ere $(A,B,D)$ et on admettra que les points $A$, $B$, $C$ et $D$ sont de coordonn\'ees respectives $(0\,;\,0)$, $(1\,;\,0)$, $(1\,;\,1)$ et $(0\,;\,1)$.
%\sautcol
\begin{center}
\psset{xunit=2cm,yunit=2cm}
\def\xmin{-0.25} \def\xmax{2.25} \def\ymin{-0.25} \def\ymax{1.25}
\begin{pspicture*}(\xmin,\ymin)(\xmax,\ymax)
\psline(2,1)(1,0)(0,0)(0,1)(2,1)(0.5,0.5)(1,0)(1,1)
\psline(0,1)(0.5,0.5)
\psdots(1,0)(0,0)(0,1)(2,1)(0.5,0.5)(1,1)(1,0.66667)(1.5,0.5)
\uput[dl](0,0){$A$}
\uput[dr](1,0){$B$}
\uput[u](1,1){$C$}
\uput[ul](0,1){$D$}
\uput[ur](2,1){$E$}
\uput[dl](0.5,0.5){$O$}
\uput[dr](1.5,0.5){$P$}
\uput[ul](1,0.66667){$Q$}
\end{pspicture*}
\end{center}
%\end{multicols}
\begin{enumerate}
 \item Montrer que $O$ a pour coordonn\'ees $\left(\frac{1}{2}\,;\,\frac{1}{2}\right)$.
 \item Montrer que $E$ a pour coordonn\'ees $\left(2\,;\,1\right)$.
 \item Montrer que $P$ a pour coordonn\'ees $\left(\frac{3}{2}\,;\,\frac{1}{2}\right)$.
 \item D\'eterminer une \'equation de la droite $(OE)$ puis en d\'eduire les coordonn\'ees de $Q$.
 \item Montrer que les points $D$, $Q$ et $P$ sont align\'es.
\end{enumerate}
\end{prob}

%\sautpage

\begin{prob}
%\begin{multicols}{2}
On donne $AB=6$ cm. $M$ est un point du segment $[AB]$ et on pose $AM=x$.\\
Dans le m\^eme demi-plan, on construit les carr\'es $AMNP$ et $MBQR$.\\
$f$ est la fonction d\'efinie sur $[0\,;\,6]$ qui \`a $x$ associe la longueur $f(x)$ de la ligne polygonale $APNRQB$
(trac\'ee en gras sur la figure ci-dessous).\\
Notez que la figure a \'et\'e faite dans le cas o\`u $x$ est compris entre 0 et 3.
\begin{center}
\psset{xunit=0.75cm , yunit=0.75cm}
\begin{pspicture*}(-1,-1)(7,5)
\psline(0,0)(6,0)
\psline[linewidth=2pt](0,0)(0,2)(2,2)(2,4)(6,4)(6,0)
\psline[linestyle=dashed](2,0)(2,2)
\uput[dl](0,0){$A$}
\uput[dr](6,0){$B$}
\uput[ul](0,2){$P$}
\uput[r](2,2){$N$}
\uput[ul](2,4){$R$}
\uput[ur](6,4){$Q$}
\uput[d](1,0){$x$}
\uput[d](2,0){$M$}
\end{pspicture*}                \end{center}%\end{multicols}\vspace{-1em}
\begin{enumerate}
 \item Faites une deuxi\`eme figure dans le cas o\`u $x$ est dans l'intervalle $[3\,;\,6]$.
 \item V\'erifiez que $f(x)=18-2x$, si $x\in[0\,;\,3]$ et que $f(x)=6+2x$, si $x\in[3\,;\,6]$.
 \item Dans un rep\`ere orthogonal (unit\'es : 1 cm sur les abscisses, 0,5 cm sur les ordonn\'ees), construisez la courbe repr\'esentatice de $f$.
 \item Trouvez graphiquement l'ensemble des valeurs de $x$ pour lesquelles la longueur de la ligne polygonale est comprise entre 14 et 16 cm.
\end{enumerate}


\end{prob}

\end{multicols}
%\sautpage





