\chapter{Probabilit\'es} \label{proba}
\minitoc

\fancyhead{} % efface les entêtes pr\'ec\'edentes
\fancyhead[LE,RO]{\footnotesize \em \rightmark} % section en entête
\fancyhead[RE,LO]{\scriptsize \em Seconde} % classe et ann\'ee en entête

    \fancyfoot{}
		\fancyfoot[RE]{\scriptsize \em \href{http://perpendiculaires.free.fr/}{http://perpendiculaires.free.fr/}}
		\fancyfoot[LO]{\scriptsize \em David ROBERT}
    \fancyfoot[LE,RO]{\textbf{\thepage}}

   %\sautpage

\section{Vocabulaire des ensembles}

\begin{tabular}{cc}

\begin{minipage}[l]{0.65\linewidth}
\begin{definition}[Intersection]
L'\emph{intersection} de deux ensembles $A$ et $B$ est l'ensemble des \'el\'ements qui sont communs \`a $A$ et $B$.\\
On la note $A\cap B$.
\end{definition}
\end{minipage}
&
\begin{minipage}[r]{0.35\linewidth}
\begin{center}
\def\xmin{-1.6} \def\xmax{4.6} \def\ymin{-1.6} \def\ymax{1.6}
\psset{xunit=1cm,yunit=0.75cm}
\begin{pspicture*}(\xmin,\ymin)(\xmax,\ymax)

\psclip{%
\pscustom[linestyle=none]{%
\psellipse(.5,0)(1.5,1)}
\pscustom[linestyle=none]{%
\psellipse(2.5,0)(1.5,1)}}
\psframe*[linecolor=grisDR](\xmin,\ymin)(\xmax,\ymax)
\endpsclip

\psellipse(.5,0)(1.5,1)
\psellipse(2.5,0)(1.5,1)
\rput(-1,1){$A$}
\rput(4,1){$B$}
\psdot[dotstyle=x](1.2,0.2)
\uput[dr](1.2,0.2){$e$}

\end{pspicture*}
\end{center}
\end{minipage}
\end{tabular}
Ainsi $e \in A \cap B$ signifie $e \in A$ \textbf{et} $e \in B$.

\begin{rmq} Lorsque $A \cap B = \vide$, on dit que les ensembles $A$ et $B$ sont disjoints.\end{rmq}

\begin{tabular}{cc}

\begin{minipage}[l]{0.65\linewidth}
\begin{definition}[R\'eunion]
La \emph{r\'eunion} de deux ensembles $A$ et $B$ est l'ensemble des \'el\'ements qui sont dans $A$ ou dans $B$. \\
On la note $A \cup B$.
\end{definition}
\end{minipage}
&
\begin{minipage}[r]{0.35\linewidth}
\begin{center}
\def\xmin{-1.6} \def\xmax{4.6} \def\ymin{-1.6} \def\ymax{1.6}
\psset{xunit=1cm,yunit=0.75cm}
\begin{pspicture*}(\xmin,\ymin)(\xmax,\ymax)

\psclip{%
\pscustom[linestyle=none]{%
\psellipse(.5,0)(1.5,1)}}
\psframe*[linecolor=grisDR](\xmin,\ymin)(\xmax,\ymax)
\endpsclip

\psclip{%
\pscustom[linestyle=none]{%
\psellipse(2.5,0)(1.5,1)}}
\psframe*[linecolor=grisDR](\xmin,\ymin)(\xmax,\ymax)
\endpsclip

\psellipse(.5,0)(1.5,1)
\psellipse(2.5,0)(1.5,1)
\rput(-1,1){$A$}
\rput(4,1){$B$}
\psdot[dotstyle=x](0,0.2)
\uput[dr](0,0.2){$e$}

\end{pspicture*}
\end{center}
\end{minipage}
\end{tabular}

Ainsi $e \in A \cup B$ signifie $e \in A$ \textbf{ou} $e \in B$.

\begin{tabular}{cc}

\begin{minipage}[l]{0.70\linewidth}
\begin{definition}[Inclusion]
On dit qu'un ensemble $A$ est \emph{inclus} dans un ensemble $B$ si tous les \'el\'ements de $A$ sont des \'el\'ements
de $B$. \\
On note alors $A \subset B$ (\og $A$ inclus dans $B$ \fg) ou $B \supset A$ (\og $B$ contient $A$ \fg).
\end{definition}
\end{minipage}
&
\begin{minipage}[r]{0.30\linewidth}
\begin{center}
\def\xmin{-1.6} \def\xmax{4.6} \def\ymin{-1.6} \def\ymax{1.6}
\psset{xunit=1cm,yunit=0.75cm}
\begin{pspicture*}(\xmin,\ymin)(\xmax,\ymax)

\psellipse(1,0)(0.75,0.5)
\psellipse(1.5,0)(1.5,1)
\rput(2,0){$A$}
\rput(3,1){$B$}

\end{pspicture*}
\end{center}
\end{minipage}
\end{tabular}

On dit alors que $A$ est une \emph{partie} de $B$ ou que $A$ est un \emph{sous-ensemble} de $B$.

\begin{rmq} $\vide$ et $E$ sont toujours des parties de $E$ (partie vide et partie pleine).\end{rmq}

On notera $\mathcal{P}(E)$ l'ensemble de toutes les parties de $E$.

\begin{tabular}{cc}

\begin{minipage}[l]{0.65\linewidth}
\begin{definition}[Compl\'ementaire]
Soit $E$ un ensemble et $A$ une partie de $E$. Le \emph{compl\'ementaire} de $A$ dans $E$ est l'ensemble des \'el\'ements
de $E$ qui n'appartiennent pas \`a $A$. On le note $\overline{A}$.
\end{definition}
\end{minipage}
&
\begin{minipage}[r]{0.35\linewidth}
\begin{center}
\def\xmin{-1.6} \def\xmax{4.6} \def\ymin{-1.6} \def\ymax{1.6}
\psset{xunit=1cm,yunit=0.75cm}
\begin{pspicture*}(\xmin,\ymin)(\xmax,\ymax)

\psclip{%
\pscustom[linestyle=none]{%
\psellipse(1.5,0)(1.5,1)}}
\psframe*[linecolor=grisDR](\xmin,\ymin)(\xmax,\ymax)
\endpsclip

\psellipse(1.5,0)(1.5,1)
\psellipse*[linecolor=white](1,0)(0.75,0.5)
\psellipse(1,0)(0.75,0.5)
\rput(1,0){$A$}
\rput(3,1){$E$}
\psline{->}(3.5,0)(2.75,0)
\uput[r](3.5,0){ $\overline{A}$}
\end{pspicture*}
\end{center}
\end{minipage}
\end{tabular}

\begin{rmq} $A \cup \overline{A} = E$ et $A \cap \overline{A} = \vide$. \end{rmq}

\begin{tabular}{cc}
\begin{minipage}[l]{0.65\linewidth}
\begin{definition}\label{partition}
Des parties $A_1$, $A_2$, $\ldots$, $A_p$ d'un ensemble $E$ constituent une \emph{partition} de $E$ si elles sont deux \`a deux
disjointes et si leur r\'eunion est $E$.\\
Ainsi :
\begin{itemize}
	\item Pour tous $i$ et $j$ de $\{1\, ;\,\ldots \,;\, p\} : i \neq j \Rightarrow A_i \cap A_j = \vide$ ;
	\item $\displaystyle\coprod_{i=1}^p A_i= A_1 \cup A_2 \cup \ldots \cup A_p = E$.
\end{itemize}
\end{definition}
\end{minipage}
&
\begin{minipage}[r]{0.35\linewidth}
\begin{center}
\def\xmin{-3.6} \def\xmax{3.6} \def\ymin{-2.1} \def\ymax{2.1}
\psset{unit=1cm}
\begin{pspicture*}(\xmin,\ymin)(\xmax,\ymax)

\psplot[plotpoints=200,algebraic=true]{-3}{3}{(1-(x/3)^2)^0.5*1.5}
\psplot[plotpoints=200,algebraic=true]{-3}{3}{(1-(x/3)^2)^0.5*(-1.5)}
\psline(-2,1.11803)(-2,-1.11803)
\psline(-1,1.41421)(-1,-1.41421)
\psline(0,1.5)(0,-1.5)
\psline(1,1.41421)(1,-1.41421)
\psline(2,1.11803)(2,-1.11803)
\rput(-2.5,0){$A_1$}
\rput(-1.5,0){$A_2$}
\rput(-0.5,0){$A_3$}
\rput(0.5,0){$A_4$}
\rput(1.5,0){$\ldots$}
\rput(2.5,0){$A_p$}
\rput(2.5,1.5){$E$}

\end{pspicture*}
\end{center}
\end{minipage}
\end{tabular}

\begin{exemple*}
 Au Lyc\'ee Dupuy de L\^ome, les \'el\`eves de Premi\`ere ou Terminale g\'en\'erale sont en S, ES, L. On prend au hasard un \'el\`eve de Premi\`ere ou de Terminale g\'en\'erale.\\ On appelle $S$ (respectivement $ES$ et $L$) l'\'ev\`enement \og l'\'el\`eve est en S (respectivement ES et L) \fg. On appelle aussi $F$ et $G$ les \'ev\`enements \og l'\'el\`eve est une fille \fg{} et \og l'\'el\`eve est un gar\c{c}on \fg.\\
 L'univers $\Omega$ est l'ensemble des \'el\`eves possibles de Premi\`ere et Terminale. \\
 Les \'ev\`enements $F$ et $G$ forment une partition de l'univers $\Omega$. Ils sont aussi compl\'ementaires.\\
 Les \'ev\`enements $S$, $ES$ et $L$ forment une partition de l'univers $\Omega$.
\end{exemple*}


\begin{definition}[Cardinal]
Le nombre d'\'el\'ements d'un ensemble fini $E$ est appel\'e \emph{cardinal de $E$}. Ce nombre est not\'e Card($E$).
On convient que Card($\vide$)$ = 0$.
\end{definition}

\begin{rmq} La notion de cardinal ne s'applique pas aux ensembles infinis (comme $\R$).\end{rmq}

\section{Exp\'eriences al\'eatoires}

\subsection{Issues, univers}

\begin{definition}
Une exp\'erience est dite \emph{al\'eatoire} lorsqu'elle a plusieurs issues (ou r\'esultats) possibles et que l'on ne peut ni pr\'evoir avec certitude, ni calculer laquelle de ces issues sera r\'ealis\'ee.\\L'ensemble de toutes les \emph{issues} d'une \emph{exp\'erience al\'eatoire} est appel\'e \emph{univers} (ou univers des possibles). On note
g\'en\'eralement cet ensemble $\Omega$.
\end{definition}

À notre niveau, $\Omega$ sera toujours un ensemble fini.

\begin{exemples*}~
\begin{itemize}
	\item On lance un d\'e et on regarde la face obtenue : $\Omega = \{1\,;\,2\,;\,3\,;\,4\,;\,5\,;\,6\}$
	\item On lance un d\'e et on regarde si le num\'ero de la face obtenue est pair ou impair : $\Omega = \{P\,;\,I\}$
	\item On lance une pi\`ece de monnaie : $\Omega = \{P\,;\,F\}$
	\item On lance deux pi\`eces de monnaie, l'une apr\`es l'autre : $\Omega = \{P_1\cap P_2\,;\,P_1\cap F_2\,;\,F_1\cap P_2\,;\,F_1\cap F_2\}$
	\item On lance un javelot et on mesure la distance entre la ligne de lancer et le point de contact du javelot avec le sol
\end{itemize}
\end{exemples*}
Remarquons qu'une même exp\'erience peut d\'eboucher sur des univers diff\'erents suivant ce que l'on observe, comme le montrent les deux premiers exemples ci-dessus.

\subsection{\'Ev\`enements}

Exemple : on lance deux d\'es et on consid\`ere la somme $S$ obtenue.\\ L'univers des possible est $\Omega = \{2\,;\,3\,;\,\cdots\,;\,11\,;\,12\}$.

Le tableau \ref{probavoctab} \vpageref{probavoctab} 
d\'efinit le vocabulaire relatif aux \emph{\'ev\`enements} (en probabilit\'e) :

\begin{table}[!h]
\centering
\caption{Vocabulaire relatif aux \'ev\`enements en probabilit\'e}\label{probavoctab}
\begin{tabular}{|>{\centering}m{0.25\textwidth}|>{\centering}m{0.25\textwidth}|m{0.4\textwidth}|}
\hline
Vocabulaire & Signification & Illustration \tabularnewline \hline
\begin{tabular}{c}
\'Ev\`enement\\
 (notation quelconque)\\
\end{tabular}
& Ensemble de plusieurs issues  &
\begin{tabular}{l}
Obtenir un nombre pair : \\
$A = \{2\,;\,4\,;\,6\,;\,8\,;\,10\,;\,12\}$ \\
 Obtenir un multiple de trois : \\
 $B = \{3\,;\,6\,;\,9\,;\,12\}$ \\
Obtenir une somme sup\'erieure \`a 10 : \\
$C = \{10\,;\,11\,;\,12\}$ \\
Obtenir une somme inf\'erieure \`a 6 : \\
$D = \{2\,;\,3\,;\,4\,;\,5\,;\,6\}$ \\
\end{tabular}
\tabularnewline \hline
\begin{tabular}{c}
\'Ev\`enement \'el\'ementaire \\
(not\'e $\omega$)\\
\end{tabular} & L'une des issues de la situation
\'etudi\'ee (un \'el\'ement de $\Omega$) & Obtenir 7 : $\omega = \{7\}$ \tabularnewline \hline


\begin{tabular}{c}
\'Ev\`enement impossible \\
 (not\'e $\vide$)\\
\end{tabular}
 & C'est un \'ev\`enement qui ne peut pas se produire & \og Obtenir 13 \fg{} est un \'ev\`enement impossible.
 \tabularnewline \hline

 \begin{tabular}{c}
\'Ev\`enement certain \\
 (not\'e $\Omega$)\\
\end{tabular}
 & C'est un \'ev\`enement qui se produira obligatoirement & \og Obtenir entre 2 et 12 \fg{} est un \'ev\`enement certain.
 \tabularnewline \hline

\begin{tabular}{c}
\'Ev\`enement \og A et B \fg \\
 (not\'e $A \cap B$)\\
\end{tabular}
 & \'Ev\`enement constitu\'e des issues
communes aux 2 \'ev\`enements & $A \cap B = \{6\,;\,12\}$ \tabularnewline \hline

\begin{tabular}{c}
\'Ev\`enement \og A ou B \fg \\
 (not\'e $A \cup B$)\\
\end{tabular}
 & \'Ev\`enement constitu\'e de toutes
les issues des deux \'ev\`enements & $A \cup B = \{2\,;\,3\,;\,4\,;\,6\,;\,8\,;\,9\,;\,10\,;\,12\}$ \tabularnewline \hline

\begin{tabular}{c}
\'Ev\`enements \\
incompatibles \\
 (on note alors \\
 $A \cap B=\vide$)\\
\end{tabular}
 & Ce sont des \'ev\`enements qui
n'ont pas d'\'el\'ements en
commun & $C \cap D = \vide$ donc $C$ et $D$ sont incompatibles. Par contre, $A$ et $B$ ne le sont pas. \tabularnewline \hline

\begin{tabular}{c}
\'Ev\`enements\\
 contraires \\
 (l'\'ev\`enement contraire \\
 de $A$ se note $\overline{A}$)\\
\end{tabular}
 & Ce sont deux \'ev\`enements
incompatibles dont la r\'eunion
forme la totalit\'e des issues ($\Omega$) & Ici, $\overline{A}$ repr\'esente l'\'ev\`enement \og obtenir une
somme impaire \fg. On a alors :
\begin{itemize}
	\item $A \cap \overline{A} = \vide$ (ensembles disjoints)
	\item $A \cup \overline{A} = \Omega$
\end{itemize}
 \tabularnewline \hline

\end{tabular}
\end{table}

%\FloatBarrier


\section{Loi de probabilit\'e sur un univers $\Omega$}

\subsection{Cas g\'en\'eral}

\begin{definition}
Soit $\Omega=\{\omega_1\,;\, \omega_2\,;\,\cdots\,;\,\omega_n\}$ l'univers d'une exp\'erience al\'eatoire.\\
D\'efinir une loi de probabilit\'e $P$ sur $\Omega$, c'est associer, \`a chaque
\'ev\`enement \'el\'ementaire $w_i$, des nombres $p_i \in [0\,;\,1]$, appel\'es \emph{probabilit\'es}, tels que :
\begin{itemize}
	\item $\displaystyle\sum_{i} p_i=p_1+p_2+\cdots+p_n=1$ ;
	\item la probabilit\'e d'un \'ev\`enement $A$, not\'ee $p(A)$, est la somme des probabilit\'es $p_i$ des \'ev\`enements \'el\'ementaires $\omega_i$ qui constituent $A$.
\end{itemize}
\end{definition}

\begin{rmq} On note aussi : $p_i = p(\{\omega_i\})=p(\omega_i)$.\end{rmq}

D\'ecrire la loi de probabilit\'e revient \`a indiquer, pour chaque \'ev\`enement \'el\'ementaire, sa probabilit\'e. On la pr\'esente g\'en\'eralement sous forme de tableau.

\FloatBarrier

\begin{exemple*}\label{detruque}
Soit un d\'e truqu\'e dont les probabilit\'es d'apparitions des faces sont donn\'ees par le tableau suivant :
\begin{center}
\begin{tabular}{|*{7}{c|}}\hline
Issue $\omega$ & 1& 2 &3 &4 &5 &6 \\ \hline
Probabilit\'e $p(\omega)$& 0,05& 0,05& 0,1& 0,1& 0,2& inconnue\\ \hline
\end{tabular}
\end{center}

\begin{enumerate}
	\item Calculer la probabilit\'e de l'\'ev\`enement $A =$ \og obtenir un r\'esultat inf\'erieur ou \'egal \`a 4 \fg.\\
D'apr\`es la d\'efinition, $p(A) = p(1) + p(2) + p(3) + p(4) = 0,3$\footnote{La notation rigoureuse est $p(\{1\})$ mais on peut noter $p(1)$ quand il n'y a pas de risque de confusion.}.

\item Calculer la probabilit\'e d'obtenir 6 :\\
D'apr\`es la d\'efinition, $p(1) + p(2) + p(3) + p(4) + p(5) + p(6) = 1$, donc $p(6) = 0,5$.
\end{enumerate}
\end{exemple*}

\begin{prop}
Soit $A$ et $B$ deux \'ev\`enements de $\Omega$, alors :
\vspace{-1em}\begin{multicols}{2}\begin{itemize}
	\item $p ( A \cup B ) = p ( A ) + p ( B ) - p ( A \cap B )$ ;
	\item $p (\overline{A}) = 1 - p ( A )$.
\end{itemize}\end{multicols}\vspace{-1em}
\end{prop}

\begin{rmq} Comme, par d\'efinition, la probabilit\'e de l'\'ev\`enement certain est 1 alors la probabilit\'e de l'\'ev\`enement impossible, qui est son contraire, est 0. \end{rmq}



\subsection{Cas particulier : l'\'equiprobabilit\'e}

\begin{definition}
Lorsque toutes les issues d'une exp\'erience al\'eatoire ont même probabilit\'e, on dit qu'il y a \emph{\'equiprobabilit\'e} ou que
la loi de probabilit\'e est \emph{\'equir\'epartie}.
\end{definition}

Dans ce cas, la r\`egle de calcul de la probabilit\'e d'un \'ev\`enement $A$ est la suivante :

\begin{prop}
Dans une situation d'\'equiprobabilit\'e sur un univers $\Omega$, pour tout \'ev\`enement \'el\'ementaire $\omega$ et tout \'ev\`enement $A$ on a :\\
\begin{tabularx}{\linewidth}{lXr}
 $p(\omega) = \delair{\dfrac{1}{\text{Card}(\Omega)}}$ & &
 $p(A) = \delair{\dfrac{\text{nombre d'issues favorables}}{\text{nombre d'issues possibles}}=\dfrac{\text{Card}(A)}{\text{Card}(\Omega)}}$
\end{tabularx}

\end{prop}

Certaines exp\'eriences ne sont pas des situations d'\'equiprobabilit\'e, mais on peut parfois s'y ramener tout de même.

\begin{exemple*}
On lance deux d\'es \'equilibr\'es et on s'int\'eresse \`a la somme des deux d\'es.\\
L'univers est $\Omega=\{2\,;\,3\,;\,4\,;\,5\,;\,6\,;\,7\,;\,8\,;\,9\,;\,10\,;\,11\,;\,12\}$ mais il n'y a pas \'equiprobabilit\'e car chaque \'ev\`enement n'a pas la même probabilit\'e. Ainsi il est plus difficile d'obtenir 2 que 7.

On se ram\`ene \`a une situtation d'\'equiprobabilit\'e : chaque d\'e \'etant \'equilibr\'e, on a \'equiprobabilit\'e sur chaque d\'e (chaque face \`a une probabilit\'e de $\frac{1}{6})$.

Il reste \`a d\'eterminer la façon d'obtenir chaque somme. Le tableau ci-dessous r\'esume les possibilit\'es pour chaque d\'e et la somme obtenue :
\begin{center}
\begin{tabular}{|*{7}{c|}}\hline
\diaghead{\theadfont D\'e 1 et d\'e 2}%
{d\'e 1}{d\'e 2}& \thead{1} & \thead{2} & \thead{3} & \thead{4} & \thead{5} & \thead{6} \\ \hline
1 & 2 & 3 & 4 & 5 & 6 & 7 \\ \hline
2 & 3 & 4 & 5 & 6 & 7 & 8 \\ \hline
3 & 4 & 5 & 6 & 7 & 8 & 9 \\ \hline
4 & 5 & 6 & 7 & 8 & 9 & 10 \\ \hline
5 & 6 & 7 & 8 & 9 & 10 & 11 \\ \hline
6 & 7 & 8 & 9 & 10 & 11 & 12 \\ \hline
\end{tabular}
\end{center}
Chaque \og case \fg{} \'etant \'equiprobable ($\frac{1}{36}$) on obtient :
\begin{center}
\begin{tabular}{|*{13}{c|}}\hline
$\omega_i$ & 2 & 3 & 4 & 5& 6 & 7 & 8& 9& 10 & 11 & 12 & Total \\ \hline
$p_i$ & $\delair{\frac{1}{36}}$ & $\delair{\frac{2}{36}}$ & $\delair{\frac{3}{36}}$ & $\delair{\frac{4}{36}}$ & $\delair{\frac{5}{36}}$ & $\delair{\frac{6}{36}}$ &
$\delair{\frac{5}{36}}$ &
$\delair{\frac{4}{36}}$ &
$\delair{\frac{3}{36}}$ &
$\delair{\frac{2}{36}}$ &
$\delair{\frac{1}{36}}$ &
1 \\ \hline
\end{tabular}
\end{center}

\end{exemple*}



\section{Exercices}



\begin{exo}
On jette un d\'e dont les faces sont num\'erot\'ees de 1 \`a 6 et on s'int\'eresse au num\'ero apparaissant sur la face sup\'erieure.
\begin{enumerate}
	\item D\'ecrire l'ensemble $\Omega$, univers associ\'e \`a cette exp\'erience al\'eatoire.
	\item \'Ecrire sous forme de partie (d'ensemble) de $\Omega$ les \'ev\`enements :
\begin{itemize}
	\item $A$ : \og obtenir un num\'ero inf\'erieur ou \'egal \`a 2 \fg ;
	\item $B$ : \og obtenir un num\'ero impair \fg ;
	\item $C$ : \og obtenir un num\'ero strictement sup\'erieur \`a 4 \fg.
\end{itemize}
\item Pour chacun des \'ev\`enements suivants, les \'ecrire sous forme de partie de $\Omega$ et les d\'ecrire par une phrase la plus simple possible.
\vspace{-1em}\begin{multicols}{5}
\begin{itemize}
	\item $A\cup B$ ;
	\item $A\cap B$ ;
	\item $A\cup C$ ;
	\item $A\cap C$ ;
	\item $C\cup B$ ;
	\item $C\cap B$ ;
	\item $\overline{A}$ ;
	\item $\overline{A}\cup C$ ;
	\item $\overline{A}\cap C$ ;
\end{itemize}
\end{multicols}\vspace{-1em}
\end{enumerate}
\end{exo}

\begin{exo}
%\begin{multicols}{2}
On choisit au hasard une carte dans un jeu de 52 cartes.
\begin{enumerate}
	\item Combien y a-t-il d'issues possibles ?
	\item On consid\`ere les \'ev\`enements :
%\vspace{-1em}\begin{multicols}{2}
\begin{itemize}
	\item $A$ : \og obtenir un as \fg ;
	\item $P$ : \og obtenir un pique \fg.
\end{itemize}
%\end{multicols}\vspace{-1em}
\begin{enumerate}
	\item Combien y a-t-il d'\'eventualit\'es dans $A$ ?
	\item Combien y a-t-il d'\'eventualit\'es dans $P$ ?
	\item Traduire par une phrase les \'ev\`enements $A\cap P$ et $A\cup P$.
	\item D\'eterminer Card$(A\cap P)$ et Card$(A\cup P)$.
\end{enumerate}
\end{enumerate}
%\end{multicols}
\end{exo}



\sautpage

\begin{exo}
$E$ est l'ensemble des nombres de 1 \`a 20 inclus. On choisit au hasard l'un de ces nombres.

\begin{enumerate}
	\item Quelle est la probabilit\'e des \'ev\`enements suivants :
\begin{itemize}
	\item $A$ : \og il est un multiple de 2 \fg
	\item $B$ : \og il est un multiple de 4 \fg
	\item $C$ : \og il est un multiple de 5 \fg
	\item $D$ : \og il est un multiple de 2 mais pas de 4 \fg
\end{itemize}
%\sautcol
\item Calculer la probabilit\'e de :
\vspace{-1em}\begin{multicols}{4}
\begin{itemize}
	\item $A \cap B$ ;
	\item $A \cup B$ ;
	\item $A \cap C$ ;
	\item $A \cup C$.
\end{itemize}\end{multicols}\vspace{-1em}
\end{enumerate}

\end{exo}

\begin{exo}
Pour se rendre \`a son travail, un automobiliste rencontre trois feux tricolores. On suppose que les feux
fonctionnent de mani\`ere ind\'ependante, que l'automobiliste s'arrête s'il voit le feu orange ou rouge et qu'il passe
si le feu est vert. On suppose de plus que chaque feu est vert durant un temps \'egal \`a rouge et orange (autrement
dit, l'automobiliste \`a autant de chance de passer que de s'arrêter).
\begin{enumerate}
	\item Faire un arbre repr\'esentant toutes les situations possibles.
	\item Quelle est la probabilit\'e que l'automobiliste ait :
\vspace{-1em}\begin{multicols}{2}
\begin{enumerate}
	\item les trois feux verts ?
	\item deux des trois feux verts ?
\end{enumerate}
\end{multicols}\vspace{-1em}
\end{enumerate}
\end{exo}

\begin{exo}
Deux lignes t\'el\'ephoniques $A$ et $B$ arrivent \`a un standard.\\ On note :
\vspace{-1em}\begin{multicols}{2}
\begin{itemize}
	\item $E_1$ : \og la ligne $A$ est occup\'e \fg ;
	\item $E_2$ : \og la ligne $B$ est occup\'ee \fg.
\end{itemize}
\end{multicols}\vspace{-1em}
Apr\`es \'etude statistique, on admet les probabilit\'es :
\vspace{-1em}\begin{multicols}{3}
\begin{itemize}
	\item $p(E_1) = 0,5$ ;
	\item $p(E_2) = 0,6$ ;
	\item $p(E_1 \cap E_2) = 0,3$.
\end{itemize}
\end{multicols}\vspace{-1em}
Calculer la probabilit\'e des \'ev\`enements suivants :
\vspace{-1em}\begin{multicols}{2}
\begin{itemize}
	\item $F$ : \og la ligne $A$ est libre \fg ;
	\item $G$ : \og une ligne au moins est occup\'ee \fg ;
	\item $H$ : \og une ligne au moins est libre \fg.
\end{itemize}
\end{multicols}\vspace{-1em}
\end{exo}





\begin{exo}
Un couple de futurs parents d\'ecide d'avoir trois enfants.
On fait l'hypoth\`ese qu'ils auront, \`a chaque fois, autant de chances d'avoir un garçon qu'une fille et qu'il n'y aura pas de jumeaux.
Calculer la probabilit\'e des \'ev\`enements :
%\vspace{-1em}\begin{multicols}{2}
\begin{itemize}
	\item $A$ : \og ils auront trois filles \fg ;
	\item $B$ : \og ils auront trois enfants de même sexe \fg ;
	\item $C$ : \og ils auront au plus une fille \fg ;
	\item $D$ : \og les trois enfants seront de sexes diff\'erents \fg.
\end{itemize}
%\end{multicols}\vspace{-1em}
\end{exo}

%\sautpage

\begin{exo}
Un d\'e (\`a 6 faces) est truqu\'e de la façon suivante : chaque chiffre pair a deux fois plus de chance de sortir qu'un
num\'ero impair.
\begin{enumerate}
	\item Calculer la probabilit\'e d'obtenir un 6.
	\item On lance deux fois le d\'e.
\begin{enumerate}
	\item Calculer la probabilit\'e d'obtenir deux fois un chiffre pair.
	\item Calculer la probabilit\'e d'obtenir deux fois un 6.
\end{enumerate}
\end{enumerate}
\end{exo}

%\begin{exo}\label{lievreettortueenonce}
%On lance $n$ d\'es ($n \geq 1$). On note $A$ l'\'ev\`enement \og obtenir au moins un 6 \fg.
%\begin{enumerate}
%	\item D\'ecrire $\overline{A}$.
%	\item Exprimer en fonction de $n$ la probabilit\'e $p (\overline A)$.
%	\item En d\'eduire que $p(A) = 1 - \left(\frac{5}{6}\right)^n$.
%	\item Compl\'eter le tableau suivant :
%\begin{center}
%\begin{tabular}{|c|*{8}{m{1cm}|}}\hline
%$n$ & 1 & 2 & 3 & 4&5&6&7&8\\ \hline
%$p(A)$&&&&&&&& \\ \hline
%\end{tabular}
%\end{center}
%\item Combien de d\'es faut-il lancer pour que la probabilit\'e d'obtenir au moins un six soit sup\'erieure \`a $\frac{3}{4}$ ?
%\item Le li\`evre et la tortue font la course.

%Le li\`evre se divertit longuement mais quand il part, il file \`a l'arriv\'ee. La tortue, quant \`a elle, avance inexorablement mais lentement vers l'arriv\'ee.

%On consid\`ere qu'on peut assimiler cette course au lancement d'un d\'e :
%\begin{itemize}
%	\item si le 6 sort, le li\`evre avance ;
%	\item sinon la tortue avance d'une case et au bout de 4 cases la tortue a gagn\'e.\\
%	Voir figure \ref{lievreettortuefigure}.
%\end{itemize}

%\begin{figure}[htpb]
%\centering
%\caption{Figure de l'exercice \ref{lievreettortueenonce}}\label{lievreettortuefigure}
%\def\xmin{-0.5} \def\xmax{8.5} \def\ymin{-0.5} \def\ymax{3.5}
%\psset{xunit=1cm,yunit=1cm}
%\begin{pspicture*}(\xmin,\ymin)(\xmax,\ymax)

%\pspolygon(0,0)(3,0)(3,1)(0,1)
%\rput(1.5,0.5){D\'epart du li\`evre}
%\psline{->}(3,0.5)(5,0.5)
%\pspolygon(5,0)(8,0)(8,1)(5,1)
%\rput(6.5,0.5){Arriv\'ee}

%\pspolygon(0,2)(3,2)(3,3)(0,3)
%\rput(1.5,2.5){D\'epart de la tortue}

%\psline{->}(3,2.5)(3.5,2.5)
%\pspolygon(3.5,2)(4.5,2)(4.5,3)(3.5,3)
%\rput(4,2.5){1}

%\psline{->}(4.5,2.5)(5,2.5)
%\pspolygon(5,2)(6,2)(6,3)(5,3)
%\rput(5.5,2.5){2}

%\psline{->}(6,2.5)(6.5,2.5)
%\pspolygon(6.5,2)(7.5,2)(7.5,3)(6.5,3)
%\rput(7,2.5){3}

%\psline{->}(7,2)(7,1)

%\end{pspicture*}
%\end{figure}
%D\'eterminer la probabilit\'e que le li\`evre l'emporte et celle que la tortue l'emporte.
%\end{enumerate}
%\end{exo}
%\sautpage





