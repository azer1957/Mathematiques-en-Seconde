



\fancyhead{} % Delete current head settings

\fancyhead{} % efface les entêtes pr\'ec\'edentes
%\fancyhead[LE,RO]{\footnotesize \em \rightmark} % section en entête
\fancyhead[RE,LO]{\scriptsize \em Seconde} % classe et ann\'ee en entête

%\setcounter{ds}{1} %c'est le num\'ero du DS
%\setcounter{chaptertemp}{\thechapter} %stocke le num\'ero du chapitre courant dans un compteur temporaire
%\stepcounter{chapter} % avance le compteur de 1 et surtout remet tous les compteurs d\'ependant du chapitre à 0, dont les num\'eros d'exercice
%\setcounter{chapter}{\theds} % met le compteur de chapitre au num\'ero du ds



%\small

\chapter{Probl\`emes -- Optimisation}


\pagenumbering{roman} \setcounter{page}{1}

\begin{prob}
 On roule une feuille A4 (de dimensions $21\times 29,7$) dans les deux sens jusqu'\`a constituer un cylindre (dit \emph{de r\'evolution}) parfait.\\
 Dans quel sens a-t-on le plus grand volume ?
\end{prob}

\begin{prob}
 On constitue le patron d'une boite (sans couvercle) \`a partir d'une feuille A4 en d\'ecoupant un carr\'e \`a chacun de ses coins.\\
 Quelle d\'ecoupe permettra d'obtenir la boite ayant le plus grand volume ?\\
 \emph{On pourra faire des essais pour d\'emarrer.}
\end{prob}

\begin{prob}
 \emph{Une brique de lait est fournie.}\\
 Une brique de lait de 1L est un pav\'e dont le patron est fabriqu\'e \`a partir d'une feuille rectangulaire de dimensions $24\times 32$.
 

\begin{figure}[h]
\centering
\caption{Le patron}\label{Lepatrondelaboitedelait}
\psset{unit=0.5cm}
\def\xmin{-1.5} \def\xmax{34} \def\ymin{-2} \def\ymax{26}
\begin{pspicture*}(\xmin,\ymin)(\xmax,\ymax)
%\psgrid[griddots=10,gridlabels=0pt,gridwidth=.3pt, gridcolor=black, subgridwidth=.3pt, subgridcolor=black, subgriddiv=1](0,0)(\xmin,\ymin)(\xmax,\ymax)
%\psaxes[labels=all,labelsep=1pt, Dx=1,Dy=1]{->}(0,0)(\xmin,\ymin)(\xmax,\ymax)
%\uput[dl](0,0){$O$}

\pspolygon(0,0)(32,0)(32,24)(0,24)
\psline{<->}(0,-1)(32,-1)
\rput*(16,-1){32}
\psline{<->}(-1,0)(-1,24)
\rput*(-1,12){24}


\def\h{2}
\def\k{22} %24-h
\def\largeur{4} %2h
\def\largeurbis{20} %16+2h

\psline{<->}(33,0)(33,\h)
\rput*(33,1){$x$}


\psline(0,\h)(32,\h)
\psline(0,\k)(32,\k)
\psline(\largeur,0)(\largeur,24)
\psline(16,0)(16,24)
\psline(\largeurbis,0)(\largeurbis,24)

\pspolygon[fillstyle=vlines](0,0)(\largeur,0)(\largeur,\h)(0,\h)
\rput*(2,1){pliage}

\pspolygon[fillstyle=vlines](0,\k)(\largeur,\k)(\largeur,24)(0,24)
\rput*(2,23){pliage}

\pspolygon[fillstyle=vlines](16,0)(\largeurbis,0)(\largeurbis,\h)(16,\h)
\rput*(18,1){pliage}

\pspolygon[fillstyle=vlines](16,\k)(\largeurbis,\k)(\largeurbis,24)(16,24)
\rput*(18,23){pliage}

\end{pspicture*}                \end{figure}

Une partie de cette feuille (la partie pli\'ee) sert donc \`a la coh\'esion de la boite, le reste constitue le patron de la boite. Voir la figure \ref{Lepatrondelaboitedelait} \vpageref{Lepatrondelaboitedelait} \`a ce sujet.

Sans tenir compte du pliage, pouvait-on obtenir une brique de plus grand volume \`a partir de la m\^eme feuille ?

\begin{enumerate}
 \item \`A l'aide du patron fourni, d\'eterminer, en fonction de $x$, la largeur, longueur, hauteur de la boite de lait.
 \item Quelles sont les valeurs minimale et maximale que l'on peut donner \`a $x$ ?
 \item En d\'eduire en fonction de $x$ le volume $\mathcal{V}(x)$ de la boite de lait.
 \item Compl\'eter le tableau ci-dessous :
 \begin{center}
  \begin{tabular}{c|c|c|c|c}
   $x$ & Largeur & Longueur & Hauteur & Volume \\
    & $\ell(x) = \ldots\ldots\ldots$ & $L(x)=\ldots\ldots\ldots$ & $h(x)=\ldots\ldots\ldots$ & $V(x)=\ldots\ldots\ldots\ldots\ldots\ldots$ \\ \hline
   0 & & & & \\ \hline
   1 & & & & \\ \hline
   2 & & & & \\ \hline
   3 & & & & \\ \hline
   4 & & & & \\ \hline
   5 & & & & \\ \hline
   6 & & & & \\ \hline
   7 & & & & \\ \hline
   8 & & & & \\ \hline
   9 & & & & \\ 
   
  \end{tabular}

 \end{center}
 \item Par t\^atonnement, \`a la calculatrice, estimer au millim\`etre la longueur de $x$ telle que le volume de la boite est maximal. La comparer \`a la longueur de la boite r\'eelle.
\end{enumerate}

\end{prob}

\begin{prob}
Un verre \`a pied a une forme conique dont la base est un disque de 8 cm de diam\`etre et de hauteur 6 cm. Il repose sur une table parfaitement horizontale.
On d\'esire le remplir de fa\c con \`a ce qu'il contienne la moiti\'e de son volume maximal.
Jusqu'\`a quelle hauteur il faut verser du liquide pour qu'il soit rempli \`a moiti\'e ?


\begin{center}
\psset{xunit=0.25cm , yunit=0.25cm}
\begin{pspicture*}(-8.1,-1.4)(5.1,14.1)
\def\xmin{-5} \def\xmax{5} \def\ymin{-1.3} \def\ymax{14}

\def\F{1 2 div 4 x 2 exp sub 0.5 exp mul}
\psplot[linecolor=black,linestyle=solid,plotpoints=200]{-2}{2}{\F}
\def\G{0 1 2 div 4 x 2 exp sub 0.5 exp mul sub}
\psplot[linecolor=black,linestyle=solid,plotpoints=200]{-2}{2}{\G}
\def\H{12 1.5 4 div 16 x 2 exp sub 0.5 exp mul add}
\psplot[linecolor=black,linestyle=solid,plotpoints=200]{-4}{4}{\H}
\def\I{12 1.5 4 div 16 x 2 exp sub 0.5 exp mul sub}
\psplot[linecolor=black,linestyle=solid,plotpoints=200]{-4}{4}{\I}
\psline(0,0)(0,6)
\psline(-3.85,11.6)(0,6)(3.85,11.6)
\psline{<->}(-3.9,12)(3.9,12)
\rput(0,12.5){\small 8 cm}
\psline{<->}(-4.5,12)(-4.5,6)
\uput[l](-3.8,9){\small 6 cm}
\psdot(4,12)\rput[l](4.1,12){$A$}
\psdot(-4,12)\rput[ur](-4.1,12.3){$B$}
\psdot(0,6)\rput[l](0.2,5.9){$S$}

\end{pspicture*}\end{center}


\end{prob}



%\end{multicols}\vspace{-1em}








