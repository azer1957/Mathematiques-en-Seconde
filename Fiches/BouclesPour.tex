



\fancyhead{} % Delete current head settings

\fancyhead{} % efface les entêtes pr\'ec\'edentes
%\fancyhead[LE,RO]{\footnotesize \em \rightmark} % section en entête
\fancyhead[RE,LO]{\scriptsize \em Seconde} % classe et ann\'ee en entête

%\setcounter{ds}{1} %c'est le num\'ero du DS
%\setcounter{chaptertemp}{\thechapter} %stocke le num\'ero du chapitre courant dans un compteur temporaire
%\stepcounter{chapter} % avance le compteur de 1 et surtout remet tous les compteurs d\'ependant du chapitre à 0, dont les num\'eros d'exercice
%\setcounter{chapter}{\theds} % met le compteur de chapitre au num\'ero du ds

\chapter{Boucle \og pour \fg}


%\pagenumbering{roman} %\setcounter{page}{1}




\begin{exo}\label{algobouclespourenonce}




\'Ecrire des algorithmes qui permettent de faire les dessins ci-dessous . %de la figure \ref{algobouclespour} \vpageref{algobouclespour}.

%\begin{figure*}[!h]

%\centering

%\caption{Figures de l'exercice \ref{algobouclespourenonce}}\label{algobouclespour}

%%%%%%%%%%%%%%%%

\begin{multicols}{3}
\begin{center}Dessin \no1

\psset{xunit=0.5cm,yunit=0.5cm}
\begin{pspicture}(-2,-0.5)(11,11)
\psgrid[gridlabels=0,subgriddiv=0,gridcolor=lightgray](0,0)(10,10)  
\psaxes{->}(0,0)(10.5,10.5)
\multido{\n=0+1}{11}{\psline[linewidth=2pt,linecolor=blue](\n,0)(\n,10)}
 \end{pspicture} \end{center}
 
\sautcol

\begin{center}Dessin \no2

\psset{xunit=0.5cm,yunit=0.5cm}
\begin{pspicture}(-2,-0.5)(11,11)
\psgrid[gridlabels=0,subgriddiv=0,gridcolor=lightgray](0,0)(10,10)  
\psaxes{->}(0,0)(10.5,10.5)
\multido{\n=0+1}{11}{\psline[linewidth=2pt,linecolor=blue](\n,\n)(\n,10)}
 \end{pspicture} \end{center}
 
 \sautcol
 
 \begin{center}Dessin \no3
 
 \psset{xunit=0.5cm,yunit=0.5cm}
\begin{pspicture}(-2,-0.5)(11,11)
\psgrid[gridlabels=0,subgriddiv=0,gridcolor=lightgray](0,0)(10,10)  
\psaxes{->}(0,0)(10.5,10.5)
\multido{\n=0+1}{11}{\psline[linewidth=2pt,linecolor=blue](10,\n)(0,0)}
 \end{pspicture} \end{center}



\end{multicols}

\begin{multicols}{3}
\begin{center}Dessin \no4

\psset{xunit=0.5cm,yunit=0.5cm}
\begin{pspicture}(-2,-0.5)(11,11)
\psgrid[gridlabels=0,subgriddiv=0,gridcolor=lightgray](0,0)(10,10)  
\psaxes{->}(0,0)(10.5,10.5)
\multido{\n=0+1}{11}{\psline[linewidth=2pt,linecolor=blue](0,\n)(\n,\n)(\n,0)}
 \end{pspicture} \end{center}
 
\sautcol

\begin{center}Dessin \no5

\psset{xunit=0.5cm,yunit=0.5cm}
\begin{pspicture}(-2,-0.5)(11,11)
\psgrid[gridlabels=0,subgriddiv=0,gridcolor=lightgray](0,0)(10,10)  
\psaxes{->}(0,0)(10.5,10.5)
\multido{\n=0+1}{11}{\psline[linewidth=2pt,linecolor=blue](10,\n)(0,0)(\n,10)}
 \end{pspicture} \end{center}
 
 \sautcol
 
 \begin{center}Dessin \no6

\psset{xunit=0.5cm,yunit=0.5cm}
\begin{pspicture}(-2,-0.5)(11,11)
\psgrid[gridlabels=0,subgriddiv=0,gridcolor=lightgray](0,0)(10,10)  
\psaxes{->}(0,0)(10.5,10.5)
\multido{\n=0+1}{11}{%
    \FPeval{coord}{10-\n}
    \psline[linewidth=2pt,linecolor=blue](\n,\n)(\coord,\n)
    \psline[linewidth=2pt,linecolor=blue](\n,\n)(\n,\coord)}
 \end{pspicture} \end{center}

\end{multicols}

\sautpage


\begin{multicols}{3}
\begin{center}Dessin \no7

\psset{xunit=0.5cm,yunit=0.5cm}
\begin{pspicture}(-2,-0.5)(11,11)
\psgrid[gridlabels=0,subgriddiv=0,gridcolor=lightgray](0,0)(10,10)  
\psaxes{->}(0,0)(10.5,10.5)
\multido{\n=0+1}{11}{\psline[linewidth=2pt,linecolor=blue](\n,0)(10,\n)}
 \end{pspicture} \end{center}
 
\sautcol

\begin{center}Dessin \no8

\psset{xunit=0.5cm,yunit=0.5cm}
\begin{pspicture}(-2,-0.5)(11,11)
\psgrid[gridlabels=0,subgriddiv=0,gridcolor=lightgray](0,0)(10,10)  
\psaxes{->}(0,0)(10.5,10.5)
\multido{\n=0+1}{11}{
	\FPeval{coord}{10-\n} 
	\psline[linewidth=2pt,linecolor=blue](\n,0)(10,\n)(\coord,10)(0,\coord)(\n,0)}
 \end{pspicture} \end{center}
 
 \sautcol
 
 \begin{center}Dessin \no9

\psset{xunit=0.5cm,yunit=0.5cm}
\begin{pspicture}(-2,-0.5)(11,11)
\psgrid[gridlabels=0,subgriddiv=0,gridcolor=lightgray](0,0)(10,10)  
\psaxes{->}(0,0)(10.5,10.5)
\multido{\n=0+1}{11}{%
	\FPeval{coord}{10-\n} 
	\psline[linewidth=2pt,linecolor=blue](\n,0)(\coord,10)
	\psline[linewidth=2pt,linecolor=blue](0,\n)(10,\coord)}
 \end{pspicture} \end{center}

\end{multicols}








%\end{figure*}


\end{exo}

\begin{exo}
 \'Ecrire un algorithme prenant comme argument un nombre entier $n$ et affichant tous les nombres entiers de $0$ \`a $n$.
\end{exo}

\begin{exo}
 \'Ecrire un algorithme prenant comme argument un nombre entier $n$ et affichant la somme de tous les nombres entiers de $0$ \`a $n$.
\end{exo}

\begin{exo}
 \'Ecrire un algorithme prenant comme argument un nombre entier $n$ et affichant le produit de tous les nombres entiers de $1$ \`a $n$.
\end{exo}

\begin{exo}
 \'Ecrire un algorithme prenant comme argument un nombre entier $n$ et affichant tous les diviseurs de $n$.
\begin{rmq}
 En langage Algobox, le reste de la division de $x$ par $y$ s'\'ecrit $x\%y$.
\end{rmq}

\end{exo}

\begin{exo}
 \'Ecrire un algorithme prenant comme argument un nombre entier $n$ et affichant le nombre de diviseurs de $n$.
\end{exo}




