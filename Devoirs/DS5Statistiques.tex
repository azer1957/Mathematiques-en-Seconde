\cleardoublepage

\fancyhead{} % Delete current head settings

		\fancyhead[LE,LO]{\footnotesize \em Nom : }
		\fancyhead[RE,RO]{\scriptsize \em Lundi 16 janvier 2017 -- 1h00}
		\fancyfoot{}

\setcounter{ds}{5} %c'est le numéro du DS	
\setcounter{chaptertemp}{\thechapter} %stocke le numéro du chapitre courant dans un compteur temporaire
\stepcounter{chapter} % avance le compteur de 1 et surtout remet tous les compteurs dépendant du chapitre à 0, dont les numéros d'exercice
\setcounter{chapter}{\theds} % met le compteur de chapitre au numéro du ds
		
    
\section*{Devoir surveill\'e n°\theds}\label{DS5}
{\centering \large Statistiques}
\addstarredchapter{Devoir surveill\'e n°\theds : Statistiques} 

\medskip

\hrule

\begin{exo}[6 points]\label{ds5exo1}
\emph{Cet exercice est \`a faire sur l'\'enonc\'e.}
\vspace{-1em}\begin{multicols}{2}
Compl\'eter le tableau ci-dessous par des s\'eries statistiques v\'erifiant toutes les contraintes suivantes : 
\begin{itemize}
 \item le nombre de donn\'ees de la s\'erie est 9 ;
 \item le minimum de la s\'erie est 10 ;
 \item le maximum de la s\'erie est 20 ;
 \item la donn\'ee m\'ediane est 15
\end{itemize}
et telles que :
\begin{itemize}
 \item la moyenne de la s\'erie 1 est 15 ;
 \item la moyenne de la s\'erie 2 est la plus grande possible ;
 \item la moyenne de la s\'erie 3 est la plus petite possible ;
 \item la moyenne de la s\'erie 4 est 16.
\end{itemize}
\end{multicols}
\begin{tabular}{c*{9}{|p{1.25cm}}}
 Donn\'ee & \no 1 & \no 2 & \no 3 & \no 4 & \no 5 & \no 6 & \no 7 & \no 8 & \no 9 \\ \hline
% &&&&&&&&& \\
 S\'erie 1 &&&&&&&&& \\ 
 &&&&&&&&& \\ \hline
% &&&&&&&&& \\
 S\'erie 2 &&&&&&&&& \\
 &&&&&&&&& \\ \hline
% &&&&&&&&& \\
 S\'erie 3 &&&&&&&&& \\ 
 &&&&&&&&& \\ \hline
% &&&&&&&&& \\
 S\'erie 4 &&&&&&&&& \\
 &&&&&&&&& \\
\end{tabular}

\end{exo}

\medskip

\hrule


\begin{exo}[6 points]\label{ds5exo2}
Le tableau suivant donne les r\'esultats (arrondis au point sup\'erieur)  obtenus par une classe de
Seconde lors d'un devoir en math\'ematiques :
\vspace{-1.5em}\begin{center}
\begin{tabular}{|*{22}{c|}}
\hline
Notes $x_i$ 	 & 0 & 1 & 2 & 3 & 4 & 5 & 6 & 7&  8 & 9 &10 &11 &12& 13 &14& 15
&16& 17& 18 &19 &20 \\ \hline
Effectifs $n_i$ & 0 & 0 & 1 & 2 & 3 & 0 & 2 & 7&  2 & 2 & 0 & 1 & 0& 2  &0 &1 &3& 1& 3& 4& 1\\ \hline
\end{tabular}
\end{center}
\begin{enumerate}
        \item \begin{enumerate}
	      \item On note $\overline{x}$ la note moyenne de cette classe. Calculer $\overline{x}$
(\emph{on arrondira au centi\`eme au besoin}).
	      \item D\'eterminer la valeur de $m$, la note m\'ediane de cette classe, en justifiant.
	      \item Le professeur consid\`ere que si l'\'ecart entre la moyenne et la m\'ediane est sup\'erieur \`a 0,75, alors il est important. 
	      Est-ce le cas ? Comment l'expliquer ?
	      \end{enumerate}
       \item \begin{enumerate}
	      \item D\'eterminer $Q_1$ et $Q_3$ les premier et
troisi\`eme quartiles de cette s\'erie, sans justifier.
	      \item Repr\'esenter, sur la figure ci-dessous, le diagramme en boite de cette s\'erie statistique.
	      \item Sur cette figure, on a d\'ej\`a repr\'esent\'e le diagramme en boite d'une s\'erie constitu\'ee des r\'esultats
d'un autre devoir de math\'ematiques de cette Seconde.\\
En vous basant sur ces diagrammes, comparer les r\'esultats de ces deux classes.\\
Question bonus : Les r\'esultats des deux devoirs sont tr\`es diff\'erents, pourtant il s'agit de la m\^eme classe ; comment pourrait-on expliquer cette diff\'erence ?
	      \end{enumerate}
      \end{enumerate}

\begin{center}
\def\xmin{-1.6} \def\xmax{20.6} \def\ymin{-1.1} \def\ymax{3.6}
\psset{xunit=0.75cm,yunit=0.75cm}
\begin{pspicture*}(\xmin,\ymin)(\xmax,\ymax)
\psset{xunit=0.375cm,yunit=0.375cm}
\psgrid[griddots=7,gridlabels=0pt,gridwidth=.3pt, gridcolor=black,
subgridwidth=.3pt, subgridcolor=black, subgriddiv=1](0,0)(-3,-2)(41,10)
\psset{xunit=0.75cm,yunit=0.75cm}
\psaxes[labels=all,labelsep=1pt, Dx=1,Dy=10]{->}(0,0)(0,0)(\xmax,0)

\rput(0,1){\small Devoir \no 1}
%\psline{*-}(5,1)(9.75,1)(9.75,1.5)(13,1.5)(13,0.5)(9.75,0.5)(9.75,1)
%\psline{*-}(19,1)(14,1)(14,1.5)(13,1.5)(13,0.5)(14,0.5)(14,1)
%\uput[d](5,1){min}
%\uput[d](19,1){max}
%\uput[d](9.75,0.5){$Q_1$}
%\uput[d](13,0.5){$m$}
%\uput[d](14,0.5){$Q_3$}

\rput(0,2.5){\small Devoir \no 2}
\psline{*-}(2,2.5)(14,2.5)(14,3)(15,3)(15,2)(14,2)(14,2.5)
\psline{*-}(20,2.5)(19.5,2.5)(19.5,3)(15,3)(15,2)(19.5,2)(19.5,2.5)
%\uput[d](4,4){min}
%\uput[d](18,4){max}
%\uput[d](7,3.5){$Q_1$}
%\uput[d](11,3.5){$m$}
%\uput[d](13,3.5){$Q_3$}

\end{pspicture*}                \end{center}
\end{exo}

%\medskip

%\hrule

\begin{exo}[8 points]
\emph{Cet exercice est un questionnaire \`a choix multiples (Q.C.M.).\\
Pour chaque question \underline{plusieurs r\'eponses sont possibles}.
Cocher la (ou les) bonne(s) r\'eponse(s). \\
%Chaque question est not\'ee sur 1. En cas d'erreur ou d'oubli, tous les points sont perdus.\\
Aucune justification n'est attendue.}

\begin{enumerate}
 \item Une s\'erie statistique a sa moyenne tr\`es sup\'erieure \`a sa m\'ediane. Cela peut provenir de :
 \renewcommand{\labelitemi}{$\square$}
	%\vspace{-1em}\begin{multicols}{1}
		\begin{itemize}
			\item les notes sup\'erieures \`a la m\'ediane sont tr\`es \'eloign\'ees de celle-ci %\sautcol
			\item les notes sup\'erieures \`a la m\'ediane sont tr\`es proches de celle-ci %\sautcol
			\item les notes inf\'erieures \`a la m\'ediane sont tr\`es \'eloign\'ees de celle-ci %\sautcol
			\item les notes inf\'erieures \`a la m\'ediane sont tr\`es proches de celle-ci 
		\end{itemize}%\end{multicols}
 \item On a relev\'e les diff\'erents prix de vente (en euros) des CD de plusieurs points de vente ainsi que le nombre de CD qui ont \'et\'e vendus pour chaque tarif et on a obtenu la s\'erie statistique suivante :
 \begin{center}
  \begin{tabular}{c|c|c|c|c|c}
   Prix de vente (en \euro{}) & 15 & 16 & 17 & 18 & 19 \\ \hline
   Nombre de CD vendus \`a ce tarif & 85 & 45 & 30 & 25 & 15
  \end{tabular}
 \end{center}
 \begin{enumerate}
  \item Le prix de vente moyen (en \euro{}) est de :
  \renewcommand{\labelitemi}{$\square$}
	\vspace{-1em}\begin{multicols}{3}
		\begin{itemize}
			\item 16,2 \sautcol
			\item 17 \sautcol
			\item 40
		\end{itemize}\end{multicols}
  \item Le prix de vente m\'edian (en \euro{}) est de :
    \renewcommand{\labelitemi}{$\square$}
	\vspace{-1em}\begin{multicols}{3}
		\begin{itemize}
			\item 16 \sautcol
			\item 17 \sautcol
			\item 30
		\end{itemize}\end{multicols}
 \end{enumerate}
 \item Une s\'erie statistique est telle que sa moyenne et sa m\'ediane sont sensiblement \'egales. Alors :
    \renewcommand{\labelitemi}{$\square$}
	%\vspace{-1em}\begin{multicols}{3}
		\begin{itemize}
			\item On peut \^etre s\^ur que les donn\'ees sont tr\`es dispers\'ees par rapport \`a la moyenne ou la m\'ediane %\sautcol
			\item On peut \^etre s\^ur que les donn\'ees sont tr\`es concentr\'ees autour de la moyenne et de la m\'ediane %\sautcol
			\item On ne peut rien dire.
		\end{itemize}%\end{multicols}
 \item Les notes d'un groupe d'\'el\`eves sont les suivantes : $\{7\,;\, 8\,;\, 9\,;\, 9\,;\, 9\,;\, 10\,;\, 10\,;\,10\,;\, 11\,;\, 17\,;\, 19\}$. Sans calculer ...
    \renewcommand{\labelitemi}{$\square$}
	%\vspace{-1em}\begin{multicols}{3}
		\begin{itemize}
			\item On peut conjecturer que la moyenne sera inf\'erieure \`a la m\'ediane %\sautcol
			\item On peut conjecturer que la moyenne sera sup\'erieure \`a la m\'ediane %\sautcol
			\item On peut conjecturer que la moyenne et la m\'ediane seront proches
			\item On ne peut rien dire.
		\end{itemize}%\end{multicols}
  \item La m\'ediane d'une s\'erie statistique est de 30. Alors
    \renewcommand{\labelitemi}{$\square$}
	%\vspace{-1em}\begin{multicols}{3}
		\begin{itemize}
			\item Syst\'ematiquement exactement la moiti\'e des donn\'ees sont des donn\'ees inf\'erieures \`a 30 et l'autre moiti\'e exactement des donn\'ees sup\'erieures \`a 30 %\sautcol
			\item En g\'en\'eral, environ la moiti\'e des donn\'ees sont des donn\'ees inf\'erieures \`a 30 et l'autre moiti\'e environ des donn\'ees sup\'erieures \`a 30 %\sautcol
			\item Syst\'ematiquement au moins la moiti\'e des donn\'ees sont des donn\'ees inf\'erieures \`a 30 et l'autre moiti\'e au moins des donn\'ees sup\'erieures \`a 30
			\item Il y a forc\'ement au moins une donn\'ee \'egale \`a 30 dans la s\'erie
		\end{itemize}%\end{multicols}
\end{enumerate}

\end{exo}








\setcounter{chapter}{\thechaptertemp} % remet le numéro de chapitre à ce qu'il était
