\cleardoublepage

\fancyhead{} % Delete current head settings

		\fancyhead[LO]{\footnotesize \em Nom : }
		\fancyhead[RE,RO]{\scriptsize \em Jeudi 9 f\'evrier 2017 -- 1h00}
		\fancyfoot{}

\setcounter{ds}{6} %c'est le numéro du DS	
\setcounter{chaptertemp}{\thechapter} %stocke le numéro du chapitre courant dans un compteur temporaire
\stepcounter{chapter} % avance le compteur de 1 et surtout remet tous les compteurs dépendant du chapitre à 0, dont les numéros d'exercice
\setcounter{chapter}{\theds} % met le compteur de chapitre au numéro du ds
		
    
\section*{Devoir surveill\'e n°\theds}\label{DS6}
{\centering \large \'Equations de droites -- Fonctions affines}
\addstarredchapter{Devoir surveill\'e n°\theds : \'Equations de droites -- Fonctions affines} 

\medskip

\hrule

\begin{exo}[12 points]\label{ds6exo1}
\emph{Les diff\'erentes questions sont ind\'ependantes.}
\begin{enumerate}
 \item Sans justifier, donner les \'equations r\'eduites des droites de la figure \ref{ds6exo1fig} \vpageref{ds6exo1fig}.
 \[\mathcal{D}_1 : \ldots\ldots\ldots\ldots\ldots\ldots\ldots\ldots \qquad \mathcal{D}_2 : \ldots\ldots\ldots\ldots\ldots\ldots\ldots\ldots \qquad
 \mathcal{D}_3 : \ldots\ldots\ldots\ldots\ldots\ldots\ldots\ldots\]
 
 \item\label{ds6exo1q2} Sur la figure \ref{ds6exo1fig} \vpageref{ds6exo1fig}, tracer les droites :
 \begin{itemize}
  \item $\mathcal{D}_4$ d'\'equation $y=-x+4$
  \item $\mathcal{D}_5$ sachant qu'elle passe par le point $A\,(-1\,;\,1)$ et que son coefficient directeur est $m=2$
  \item $\mathcal{D}_6$ sachant qu'elle passe par le point $B\,(2\,;\,0)$ et que son coefficient directeur est $m=-\frac{1}{3}$
 \end{itemize}

 
 \begin{figure}[!h]
  
 \centering
 
 \caption{Figure de l'exercice \ref{ds6exo1}}\label{ds6exo1fig}
 
 \medskip
 

\psset{xunit=1cm,yunit=1cm}
\def\xmin{-5.6} \def\xmax{6.6} \def\ymin{-5.6} \def\ymax{6.6}
\begin{pspicture*}(\xmin,\ymin)(\xmax,\ymax)
%\psset{xunit=1cm,yunit=1cm}
\psgrid[griddots=7,gridlabels=0pt,gridwidth=.3pt, gridcolor=black, subgridwidth=.3pt, subgridcolor=black, subgriddiv=1](0,0)(\xmin,\ymin)(\xmax,\ymax)
%\psset{xunit=1cm,yunit=1cm}
\psaxes[labels=all,labelsep=1pt, Dx=1,Dy=1]{-}(0,0)(\xmin,\ymin)(\xmax,\ymax)
\uput[dl](0,0){$O$}
\uput[ul](\xmax,0){$x$}
\uput[ur](0,\ymax){$y$}

\psplot[algebraic=true]{\xmin}{\xmax}{-1.5*x+9}
\uput[ur](4,3){$\mathcal{D}_3$}
\psplot[algebraic=true]{\xmin}{\xmax}{0.5*x+3}
\uput[ul](-4,1){$\mathcal{D}_1$}
\psplot[algebraic=true]{\xmin}{\xmax}{-3*x+5}
\uput[dl](2,-1){$\mathcal{D}_2$}

\end{pspicture*}
\end{figure}

\item D\'eterminer les \'equations des droites suivantes, en d\'etaillant les calculs sur sa copie :
  \begin{itemize}
   \item $\mathcal{D}_5$ d\'efinie \`a la question \ref{ds6exo1q2}
   \item $\mathcal{D}_6$ d\'efinie \`a la question \ref{ds6exo1q2}
   \item $\mathcal{D}_7$ sachant qu'elle passe par les points $C\,(1\,;\,-2)$ et $D\,(3\,;\,2)$.
  \end{itemize}

  \item \begin{enumerate}
         \item Les droites $\mathcal{D}_7$ et $\mathcal{D}_5$ sont-elles parall\`eles ? justifier sa r\'eponse.
         \item Les droites $\mathcal{D}_2$ et $\mathcal{D}_6$ sont-elles parall\`eles ? justifier sa r\'eponse.
        \end{enumerate}


\end{enumerate}
\end{exo}

\hrule

\sautpage

\begin{exo}[5 points]
 Le\"ila prend r\'eguli\`erement le taxi lors de ses d\'eplacements professionnels, toujours avec la m\^eme compagnie de taxi qui a un accord avec son entreprise. Elle paye elle-m\^eme le taxi et son entreprise lui rembourse ses notes. \\
 La comptable de son entreprise d\'esire v\'erifier que la compagnie de taxi applique bien le tarif convenu et demande \`a Le\"ila ses derni\`eres notes.\\
 Avant de les fournir \`a sa comptable, Le\"ila d\'esire \'etudier, avec votre aide, la tarification de cette compagnie.
 Voici les notes qu'elle a relev\'ees :
\begin{center}
 \begin{tabular}{cccc}
  Note & \no 1 & \no 2 & \no 3 \\
  Distance & 10\,km & 35\,km& 25\,km \\
  Tarif & 20\,\euro{}&57,50\,\euro{}&42,50\,\euro{}
 \end{tabular}\end{center}
\begin{enumerate}
 \item Montrer que le tarif pay\'e est une fonction affine de la distance parcourue. 
 \item Si on appelle $x$ la distance parcourue, en km, et $f(x)$ le tarif pay\'e, en \euro{}, d\'eterminer $m$ et $p$ tels que $f(x)=mx+p$.
 \item Interpr\'eter $m$ et $p$ dans le contexte.
\end{enumerate}

\end{exo}

\hrule


\begin{exo}[3 points]
 On donne, pour tout r\'eel $x$, $P(x)=(3x+6)(-4x+2)$.\\
 \'Etudier le signe de $P(x)$ selon les valeurs de $x$.\\
 \emph{On pourra s'aider d'un tableau de signe.}
\end{exo}

\hrule








\setcounter{chapter}{\thechaptertemp} % remet le numéro de chapitre à ce qu'il était
