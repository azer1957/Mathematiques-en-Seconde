\cleardoublepage
 



\fancyhead{} % Delete current head settings
\fancyfoot{}

%		\fancyhead[LE,LO]{\footnotesize \em Nom : }
%		\fancyhead[RE,RO]{\scriptsize \em Pour le vendredi 28 septembre}

\setcounter{ds}{2} %c'est le numéro du DS
\setcounter{chaptertemp}{\thechapter} %stocke le numéro du chapitre courant dans un compteur temporaire
\stepcounter{chapter} % avance le compteur de 1 et surtout remet tous les compteurs dépendant du chapitre à 0, dont les numéros d'exercice
\setcounter{chapter}{\theds} % met le compteur de chapitre au numéro du ds

\renewcommand{\headrulewidth}{0pt}

\begin{landscape}
 


\setlength{\columnsep}{80pt} %defaut = 10 pt
\begin{multicols}{2}


\section*{Devoir maison n°\theds}\label{DM2}
{\centering \large Fonctions affines}
\addstarredchapter{Devoir maison n°\theds : Fonctions affines}

\begin{flushright}
\emph{\`A rendre pour le vendredi 3 f\'evrier}\end{flushright}



%\begin{prob}[Comparaison de tarifs]
Le tableau ci-dessous présente un extrait des tarifs des forfaits non bloqués pour téléphones portables, proposés par une société de téléphonie fictive.




\begin{center}
\begin{tabular}{|c|*{3}{p{2.5cm}|}}
		\hline
		Forfait & Min comprises dans le forfait	& Co\^ut du forfait (en \euro{}) & Par min de dépassement (en \euro{}) \\ \hline
		1 & 90 & 33	& 0,30	\\ \hline
		2 & 180 & 43 & 0,25 \\ \hline
		3 & 300 & 57 & 0,18 \\ \hline
		\end{tabular}
\end{center}



\noindent Pour les forfaits 1, 2 et 3 on désigne, respectivement, par $f_1(x)$, $f_2(x)$ et $f_3(x)$ le prix à payer en euros pour une durée totale de communications de $x$ minutes.

\begin{enumerate}
	\item
		\begin{enumerate}
			\item Exprimer $f_1(x)$ en fonction de $x$ lorsque $0\leqslant x \leqslant 90$ puis lorsque $x>90$.
			\item Dans un repère orthogonal, représenter graphquement la fonction $f_1$ pour $x$ compris entre 0 et 400.
		\end{enumerate}
	\item Exprimer $f_2(x)$ et $f_3(x)$ en fonction de $x$ et représenter sur le graphique précédent ces deux fonctions.
	\item Lire sur le graphique quel est le tarif le plus avantageurx en fonction de la durée mensuelle des communications.
	\item \'Ecrire un algorithme prenant comme argument une dur\'ee de communication et indiquant quel forfait est le plus avantageux pour cette dur\'ee ainsi que le prix total \`a payer.
\end{enumerate}
%\end{prob}


\sautcol

\section*{Devoir maison n°\theds}\label{DM2}
{\centering \large Fonctions affines}
%\addstarredchapter{Devoir maison n°\theds : Fonctions affines}

\begin{flushright}
\emph{\`A rendre pour le vendredi 3 f\'evrier}\end{flushright}



%\begin{prob}[Comparaison de tarifs]
Le tableau ci-dessous présente un extrait des tarifs des forfaits non bloqués pour téléphones portables, proposés par une société de téléphonie fictive.


\begin{center}
\begin{tabular}{|c|*{3}{p{2.5cm}|}}
		\hline
		Forfait & Min comprises dans le forfait	& Co\^ut du forfait (en \euro{}) & Par min de dépassement (en \euro{}) \\ \hline
		1 & 90 & 33	& 0,30	\\ \hline
		2 & 180 & 43 & 0,25 \\ \hline
		3 & 300 & 57 & 0,18 \\ \hline
		\end{tabular}
\end{center}


\noindent Pour les forfaits 1, 2 et 3 on désigne, respectivement, par $f_1(x)$, $f_2(x)$ et $f_3(x)$ le prix à payer en euros pour une durée totale de communications de $x$ minutes.

\begin{enumerate}
	\item
		\begin{enumerate}
			\item Exprimer $f_1(x)$ en fonction de $x$ lorsque $0\leqslant x \leqslant 90$ puis lorsque $x>90$.
			\item Dans un repère orthogonal, représenter graphquement la fonction $f_1$ pour $x$ compris entre 0 et 400.
		\end{enumerate}
	\item Exprimer $f_2(x)$ et $f_3(x)$ en fonction de $x$ et représenter sur le graphique précédent ces deux fonctions.
	\item Lire sur le graphique quel est le tarif le plus avantageurx en fonction de la durée mensuelle des communications.
	\item \'Ecrire un algorithme prenant comme argument une dur\'ee de communication et indiquant quel forfait est le plus avantageux pour cette dur\'ee ainsi que le prix total \`a payer.
\end{enumerate}
%\end{prob}

\end{multicols}

\end{landscape}

\setcounter{chapter}{\thechaptertemp} % remet le numéro de chapitre à ce qu'il était

\renewcommand{\headrulewidth}{0.5pt}
