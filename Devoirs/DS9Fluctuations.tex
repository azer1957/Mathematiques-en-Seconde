\cleardoublepage

\fancyhead{} % Delete current head settings

		\fancyhead[LE,LO]{\footnotesize \em Nom : }
		\fancyhead[RE,RO]{\scriptsize \em Lundi 22 mai -- 1h00}

\setcounter{ds}{9} %c'est le numéro du DS
\setcounter{chaptertemp}{\thechapter} %stocke le numéro du chapitre courant dans un compteur temporaire
\stepcounter{chapter} % remet tous les compteurs dépendant du chapitre à 0, dont les numéros d'exercice

%(et avance le compteur de 1 mais on s'en fout à cause de la ligne qui suit)
\setcounter{chapter}{\theds} % met le compteur de chapitre au numéro du ds

			 %(il sera remis à son numéro normal en fin de fichier)


\section*{Devoir surveillé n°\theds}\label{DS9}
{\centering \large Fluctuations}
\addstarredchapter{Devoir surveillé n°\theds : Fluctuations}

%\begin{small}

\medskip

\hrule

\begin{exo}[8 points]\label{ds10exo1}
Une urne contient 10 boules : \textbf{sept} rouges, \textbf{trois} noires. On tire une boule et on note sa couleur et on la remet dans l'urne.
\begin{enumerate}
\item \begin{enumerate}
       \item \`A chaque tirage, quelles sont les probabilit\'es $p(R)$ et $p(N)$ d'obtenir les \'ev\`enements suivants :
       \vspace{-1em}\begin{multicols}{2}\begin{description}
        \item[R :] \og{} La boule tir\'ee est rouge \fg{}
        \item[N :] \og{} La boule tir\'ee est noire \fg{}
       \end{description}\end{multicols}\vspace{-1em}
       \item\label{ds9q1} D\'eterminer pour chacune des couleurs l'intervalle de fluctuation au seuil de 95\,\% pour un \'echantillon de taille 50.
       \item Interpr\'eter ces intervalles de fluctuation.
      \end{enumerate}
\item On donne la table de nombres aléatoires entiers de 0 à 9 suivante : 
	\reinitrand[first=0, last=9]
\begin{center}
\rand\arabic{rand} \quad \rand\arabic{rand} \quad \rand\arabic{rand} \quad \rand\arabic{rand} \quad \rand\arabic{rand} \quad \rand\arabic{rand} \quad \rand\arabic{rand} \quad \rand\arabic{rand} \quad \rand\arabic{rand} \quad \rand\arabic{rand} \quad \rand\arabic{rand} \quad \rand\arabic{rand} \quad \rand\arabic{rand} \quad \rand\arabic{rand} \quad \rand\arabic{rand} \quad \rand\arabic{rand} \quad \rand\arabic{rand} \quad \rand\arabic{rand} \quad \rand\arabic{rand} \quad \rand\arabic{rand}

\rand\arabic{rand} \quad \rand\arabic{rand} \quad \rand\arabic{rand} \quad \rand\arabic{rand} \quad \rand\arabic{rand} \quad \rand\arabic{rand} \quad \rand\arabic{rand} \quad \rand\arabic{rand} \quad \rand\arabic{rand} \quad \rand\arabic{rand} \quad \rand\arabic{rand} \quad \rand\arabic{rand} \quad \rand\arabic{rand} \quad \rand\arabic{rand} \quad \rand\arabic{rand} \quad \rand\arabic{rand} \quad \rand\arabic{rand} \quad \rand\arabic{rand} \quad \rand\arabic{rand} \quad \rand\arabic{rand}

\rand\arabic{rand} \quad \rand\arabic{rand} \quad \rand\arabic{rand} \quad \rand\arabic{rand} \quad \rand\arabic{rand} \quad \rand\arabic{rand} \quad \rand\arabic{rand} \quad \rand\arabic{rand} \quad \rand\arabic{rand} \quad \rand\arabic{rand} \quad \rand\arabic{rand} \quad \rand\arabic{rand} \quad \rand\arabic{rand} \quad \rand\arabic{rand} \quad \rand\arabic{rand} \quad \rand\arabic{rand} \quad \rand\arabic{rand} \quad \rand\arabic{rand} \quad \rand\arabic{rand} \quad \rand\arabic{rand}

\rand\arabic{rand} \quad \rand\arabic{rand} \quad \rand\arabic{rand} \quad \rand\arabic{rand} \quad \rand\arabic{rand} \quad \rand\arabic{rand} \quad \rand\arabic{rand} \quad \rand\arabic{rand} \quad \rand\arabic{rand} \quad \rand\arabic{rand} \quad \rand\arabic{rand} \quad \rand\arabic{rand} \quad \rand\arabic{rand} \quad \rand\arabic{rand} \quad \rand\arabic{rand} \quad \rand\arabic{rand} \quad \rand\arabic{rand} \quad \rand\arabic{rand} \quad \rand\arabic{rand} \quad \rand\arabic{rand}

\rand\arabic{rand} \quad \rand\arabic{rand} \quad \rand\arabic{rand} \quad \rand\arabic{rand} \quad \rand\arabic{rand} \quad \rand\arabic{rand} \quad \rand\arabic{rand} \quad \rand\arabic{rand} \quad \rand\arabic{rand} \quad \rand\arabic{rand} \quad \rand\arabic{rand} \quad \rand\arabic{rand} \quad \rand\arabic{rand} \quad \rand\arabic{rand} \quad \rand\arabic{rand} \quad \rand\arabic{rand} \quad \rand\arabic{rand} \quad \rand\arabic{rand} \quad \rand\arabic{rand} \quad \rand\arabic{rand}
\end{center}
\begin{enumerate}
 \item D\'ecrire pr\'ecis\'ement comment simuler 50 tirages dans cette urne contenant les sept rouges et trois noires.
 \item Donner la liste des r\'esultats de vos 50 simulations.
 \item Calculer les fréquences obtenues pour chaque couleur.
 \item Ces fr\'equences sont-elles dans les intervalles de fluctuation d\'etermin\'es dans la question \ref{ds9q1} ?\\
	Conclure.
\end{enumerate}

\end{enumerate}
\end{exo}

\medskip

\hrule

\sautpage

\begin{exo}[12 points]
\emph{On arrondira tous les r\'esultats au centi\`eme.}\\

Dans un tr\`es gros lyc\'ee, la distribution des \'el\`eves de Seconde entre externes, demi-pensionnaires et internes est la suivante :
\begin{center}
 \begin{tabular}{c|c|c}
    Externes & Demi-pensionnaires & Internes \\ \hline
    404 & 1366 & 230
 \end{tabular}
\end{center}
\begin{enumerate}
  \item Quelles sont les proportions $p_E$, $p_D$ et $p_I$, respectivement d'externes, de demi-pensionnaires et d'internes ? 
  \item On constitue une classe en piochant $n=36$ \'el\`eves au hasard parmi les \'el\`eves de Seconde.\\
  On souhaite estimer l'intervalle de fluctuation au seuil de 95\,\% pour ces diff\'erentes proportions.
	\begin{enumerate}
	 \item Expliquer pourquoi on peut d\'eterminer les intervalles de fluctuations pour les externes, pour les demi-pensionnaires mais pas pour les internes.
	 \item \label{q1} D\'eterminer l'intervalle de fluctuation pour la proportion d'externes, qu'on appellera $I$, pour un \'echantillon de taille 36.\\
	 Pour la suite on pourra aussi utiliser, si besoin, l'intervalle de fluctuation au seuil de 95\,\% pour la proportion de \underline{demi-pensionnaires} pour un \'echantillon de taille $n=36$ suivant : $J=[0,516\,;\,0,850]$.
	\end{enumerate}
 
 \item Dans ce Lyc\'ee, il y a 36 \'el\`eves en Seconde 06 et leur distribution est la suivante :
\begin{center}
 \begin{tabular}{c|c|c}
    Externes & Demi-pensionnaires & Internes \\ \hline
    12 & 18 & 6
 \end{tabular}
\end{center}
\emph{On justifiera avec soin et arguments math\'ematiques ses r\'eponses aux questions suivantes.}
\begin{enumerate}
 \item La fr\'equence d'externes dans cette classe peut-elle \^etre due au hasard ?
 \item La fr\'equence de demi-pensionnaires dans cette classe est-elle repr\'esentative de la population globale des \'el\`eves de Seconde dans ce lyc\'ee ?
\end{enumerate}
 \item On fait l'hypoth\`ese que dans les \'el\`eves de Seconde de ce lyc\'ee, il y a la moiti\'e de filles et la moiti\'e de gar\c{c}ons.\\
 Dans cette m\^eme Seconde 06, il y a 11 filles et 25 gar\c{c}ons. \\
 Qu'en conclure ?
\end{enumerate}
%\end{enumerate}

\end{exo}

\medskip

\hrule

%\end{small}



\setcounter{chapter}{\thechaptertemp} % remet le numéro de chapitre à ce qu'il était



