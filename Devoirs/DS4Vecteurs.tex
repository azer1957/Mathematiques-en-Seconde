\cleardoublepage

\fancyhead{} % Delete current head settings

		\fancyhead[LE,LO]{\footnotesize \em Nom : }
		\fancyhead[RE,RO]{\scriptsize \em Vendredi 16 novembre 2016 -- 1h30}
		\fancyfoot{}

\setcounter{ds}{4} %c'est le numéro du DS	
\setcounter{chaptertemp}{\thechapter} %stocke le numéro du chapitre courant dans un compteur temporaire
\stepcounter{chapter} % avance le compteur de 1 et surtout remet tous les compteurs dépendant du chapitre à 0, dont les numéros d'exercice
\setcounter{chapter}{\theds} % met le compteur de chapitre au numéro du ds
		
    
\section*{Devoir surveill\'e n°\theds}\label{DS4}
{\centering \large Vecteurs}
\addstarredchapter{Devoir surveill\'e n°\theds : Vecteurs} 



\begin{exo}[3 points]\label{ds4exo1}
Sur la figure ci-dessous, $ABCD$ est un parall\'elogramme et $I$, $J$, $K$ et $L$ sont les milieux respectifs des segments $[AB]$, $[BC]$, $[CD]$ et $[DA]$.

\begin{center}
\psset{xunit=1cm,yunit=1cm}
\def\xmin{-2.1} \def\xmax{12.1} \def\ymin{-1.1} \def\ymax{3.1}
\begin{pspicture*}(\xmin,\ymin)(\xmax,\ymax)
\psgrid[gridlabels=0pt,gridwidth=.3pt, gridcolor=gray, subgridwidth=.3pt, subgridcolor=gray, subgriddiv=1](\xmin,\ymin)(\xmax,\ymax)

\psline[linewidth=1.2pt](0,0)(2,2)(10,2)(8,0)(0,0)
\psdots(0,0)(1,1)(2,2)(6,2)(10,2)(9,1)(8,0)(4,0)

\uput[ul](2,2){$A$}
\uput[ur](10,2){$B$}
\uput[dr](8,0){$C$}
\uput[dl](0,0){$D$}
\uput[ul](1,1){$L$}
\uput[u](6,2){$I$}
\uput[dr](9,1){$J$}
\uput[d](4,0){$K$}

\end{pspicture*}
\end{center}

Compl\'eter les \'egalit\'es suivantes \`a l'aide des points de la figure. \emph{Aucune justification n'est attendue.}

\vspace{-1em}\begin{multicols}{2}
\begin{enumerate}
 \item $\V{AL}+\V{KJ}=\V{A\ldots}$
 \item $\V{LJ}-\V{AC}=\V{D\ldots}$
 \item $\V{BD}+\V{CJ}=\V{\ldots D}$
 \item $\V{AK}+\V{DL}+\V{BI}=\ldots\ldots$
\end{enumerate}
\end{multicols}\vspace{-1em}

\end{exo}

\medskip

\hrule

\begin{exo}[4 points]
 En utilisant les points de la figure, indiquer :
 \begin{multicols}{2}\begin{enumerate}
  \item un vecteur \'egal \`a $\V{DE}$ : \dotfill \\.\dotfill
  \item un vecteur oppos\'e \`a $\V{AF}$ : \dotfill \\.\dotfill
  \item un vecteur \'egal \`a $\V{FE}$ d'origine $A$ : \dotfill \\.\dotfill
  \item un vecteur oppos\'e \`a $\V{CD}$ et d'extr\'emit\'e $B$ :\\ .\dotfill\\.\dotfill
  \item deux vecteurs de m\^eme norme (longueur) mais de directions diff\'erentes :\dotfill \\ .\dotfill\\ .\dotfill
 \end{enumerate}
 
\begin{center}
\psset{xunit=1cm,yunit=1cm}
\def\xmin{-1.1} \def\xmax{7.1} \def\ymin{-1.1} \def\ymax{6.1}
\begin{pspicture*}(\xmin,\ymin)(\xmax,\ymax)
\psgrid[gridlabels=0pt,gridwidth=.3pt, gridcolor=gray, subgridwidth=.3pt, subgridcolor=gray, subgriddiv=1](\xmin,\ymin)(\xmax,\ymax)

\psline[linewidth=1.2pt](2,1)(0,0)(1,3)(2,1)(4,2)(1,3)(5,5)(4,2)(6,3)(5,5)
\psdots(0,0)(1,3)(2,1)(4,2)(5,5)(6,3)

\uput[dl](0,0){$A$}
\uput[dr](2,1){$B$}
\uput[ul](1,3){$F$}
\uput[dr](4,2){$C$}
\uput[dr](6,3){$D$}
\uput[ur](5,5){$E$}

\end{pspicture*}
\end{center} 
\end{multicols}
\end{exo}


\medskip

\hrule

\begin{exo}[3 points]
 Simplifier au maximum les \'ecritures des vecteurs suivants :
 \begin{enumerate}
  \item $\V{u}=\V{AB}-\V{AC}+\V{BC}-\V{BA}=\dotfill \\ .\dotfill \\ .\dotfill $
  \item $\V{v}=\V{OA}-\V{OB}+\V{AC}=\dotfill \\ .\dotfill \\ .\dotfill $
  \item $\V{w}=\V{MA}-\V{MB}+\V{MC}-\V{MD}=\dotfill \\ .\dotfill \\ .\dotfill $
 \end{enumerate}

 
\end{exo}

\medskip

\hrule

\medskip

\emph{Les exercices suivants sont \`a faire sur une feuille et plus sur l'\'enonc\'e.}

\medskip

\hrule

\begin{exo}[3,5 points]
 $EFGH$ est un parall\'elogramme de centre $O$.
 \begin{enumerate}
  \item Construire les points $S$ et $T$ tels que $\V{OT}=\V{OE}+\V{OF}$ et $\V{OS}=\V{OG}+\V{OH}$.
  \item D\'emontrer que $\V{OT}+\V{OS}=\V{0}$ ; que peut-on en d\'eduire ?
 \end{enumerate}

\end{exo}

\medskip

\hrule

\begin{exo}[6,5 points]
 Soit $ABCD$ un parall\'elogramme. Soit $E$ et $F$ les points d\'efinis par $V{BE}=2\V{AB}$ et $\V{AF}=\frac{3}{2}\V{AD}$.
 \begin{enumerate}
  \item Faire une figure.
  \item Exprimer $\V{CE}$ en fonction de $\V{AB}$ et $\V{BC}$.
  \item Exprimer $\V{CF}$ en fonction de $\V{AB}$ et $\V{BC}$.
  \item D\'emontrer que les points $C$, $E$ et $F$ sont align\'es.
 \end{enumerate}
\end{exo}

\medskip

\hrule

\begin{exo}
 \emph{Cet exercice est en bonus ; il est hors bar\`eme.}\\
 Sur la figure ci-contre, $ABC$ est un triangle.
 \begin{enumerate}
  \item Placer sur cette figure les points $D$ et $E$ tels que $\V{AD}=3\V{AC}$ et $\V{AE}=\V{AB}+\V{AC}$.
  \item Placer sur cette m\^eme figure le point $F$ tel que $\V{AF}=3\V{BF}$.\\ \emph{Conseil : exprimer $\V{AF}$ en fonction de $\V{AB}$.}
  \item D\'emontrer que $(DE)$ et $(EF)$ sont parall\`eles.
 \end{enumerate}
 
 \begin{center}
\psset{xunit=0.5cm,yunit=0.5cm}
\def\xmin{-0.1} \def\xmax{22.1} \def\ymin{-0.1} \def\ymax{14.1}
\begin{pspicture*}(\xmin,\ymin)(\xmax,\ymax)
\psgrid[gridlabels=0pt,gridwidth=.3pt, gridcolor=gray, subgridwidth=.3pt, subgridcolor=gray, subgriddiv=1](\xmin,\ymin)(\xmax,\ymax)

\psline[linewidth=1.2pt](2,7)(8,5)(10,11)(2,7)
\psdots(8,5)(10,11)(2,7)

\uput[l](2,7){$A$}
\uput[d](8,5){$C$}
\uput[u](10,11){$B$}


\end{pspicture*}
\end{center} 

\end{exo}






\setcounter{chapter}{\thechaptertemp} % remet le numéro de chapitre à ce qu'il était
