\cleardoublepage

\fancyhead{} % Delete current head settings

		\fancyhead[LE,LO]{\footnotesize \em Nom : }
		\fancyhead[RE,RO]{\scriptsize \em Vendredi 14 octobre 2016 -- 1h00}
		\fancyfoot{}

\setcounter{ds}{2} %c'est le numéro du DS	
\setcounter{chaptertemp}{\thechapter} %stocke le numéro du chapitre courant dans un compteur temporaire
\stepcounter{chapter} % avance le compteur de 1 et surtout remet tous les compteurs dépendant du chapitre à 0, dont les numéros d'exercice
\setcounter{chapter}{\theds} % met le compteur de chapitre au numéro du ds
		
    
\section*{Devoir surveill\'e n°\theds}\label{DS2}
{\centering \large G\'en\'eralit\'es sur les fonctions}
\addstarredchapter{Devoir surveill\'e n°\theds : G\'en\'eralit\'es sur les fonctions} 

\begin{exo}[9 points]\label{ds2exo1}
On donne sur la figure ci-dessous les courbes repr\'esentatives de deux fonctions $f$ et $g$, nomm\'ees, respectivement, $\mathcal{C}_f$ et $\mathcal{C}_g$, d\'efinies toutes deux sur l'intervalle $[-3,5\,;\,3,5]$.

\begin{center}
    \def\xmin{-4.1} \def\xmax{4.1} \def\ymin{-4.1} \def\ymax{7.1}
\psset{xunit=2cm,yunit=0.5cm}
\begin{pspicture*}(\xmin,\ymin)(\xmax,\ymax)
\psset{unit=0.5cm}
\psgrid[griddots=10,gridlabels=0pt,gridwidth=.5pt, gridcolor=black, subgridwidth=.3pt, subgridcolor=black, subgriddiv=1](0,0)(-16,\ymin)(16,\ymax)
\psset{xunit=2cm,yunit=0.5cm}
\psaxes[labels=all,labelsep=1pt, Dx=1,Dy=1]{->}(0,0)(\xmin,\ymin)(\xmax,\ymax)
\psdots(0,0)(1,0)(0,1)%
\uput[dl](0,0){$O$}
\uput[u](1,0){$I$}
\uput[r](0,1){$J$}
\uput[ul](\xmax,0){$x$}
\uput[dr](0,\ymax){$y$}

\psplot[plotpoints=200,algebraic=true]{-3.5}{3.5}{0.25*(x-1)*(x+3)*(x-2)}
\psplot[plotpoints=200,algebraic=true]{-3.5}{3.5}{-0.25*(x-1)*(x-1)+4}
\psdots(-3.5,-3.09375)(-3.5,-1.0625)(3.5,6,09375)(3.5,2.4375)(-3,0)(-2,3)(-1,3)(1,0)(1,4)(2,0)(3,3)

\uput[u](1,4){$\mathcal{C}_g$}
\uput[ul](-2,3){$\mathcal{C}_f$}
\end{pspicture*}\end{center}

%\medskip

\textsc{\textbf{Partie A}} (2,5 points)\\
Des phrases sont propos\'ees ci-dessous.\\
Indiquer si elles sont vraies ou fausses et, si elles sont fausses, les corriger pour qu'elles deviennent vraies.

\begin{multicols}{2}
 \begin{enumerate}
  \item L'image de $-2$ par $g$ est 3\\
	.\dotfill.\\
	.\dotfill.
  \item $1,5$ a trois ant\'ec\'edents par $f$\\
	.\dotfill.\\
	.\dotfill. \sautcol
  \item 3 est un ant\'ec\'edent de $-2$ par $f$\\
	.\dotfill.\\
	.\dotfill. 
  \item 0 a pour image 2 par $f$\\
	.\dotfill.\\
	.\dotfill.  
 \end{enumerate} 
\end{multicols}


\medskip

\textsc{\textbf{Partie B}} (4,5 points)\\
Avec la pr\'ecision persmise par le graphique, r\'esoudre les \'equations et in\'equations suivantes.

\begin{multicols}{2}
 \begin{enumerate}
  \item $f(x)=0$\\
	.\dotfill.\\
	.\dotfill.
  \item $g(x)>0$\\
	.\dotfill.\\
	.\dotfill.
  \item $f(x)\geqslant 3$\\
	.\dotfill.\\
	.\dotfill. \sautcol
  \item $g(x)< 3$\\
	.\dotfill.\\
	.\dotfill. 
  \item $f(x)=g(x)$\\
	.\dotfill.\\
	.\dotfill.  
  \item $f(x)>g(x)$\\
	.\dotfill.\\
	.\dotfill.
 \end{enumerate} 
\end{multicols}

\medskip

\textsc{\textbf{Partie C}} (2 points)\\
D\'eterminer graphiquement le signe de $f(x)$ selon les valeurs de $x$. \emph{On pourra pr\'esenter sa r\'eponse sous la forme d'un tableau.}
.\dotfill.\\
.\dotfill.\\
.\dotfill.\\
.\dotfill.


\end{exo}

%\medskip

%\hrule

\begin{exo}[3 points]\label{ds2exo2}
Soit $f$ la fonction d\'efinie sur $[-2\,;\,4]$ par $f : x\longmapsto -x^2+2x+3$.
\begin{enumerate}
 \item Compl\'eter le tableau de valeurs ci-dessous :
 %\vspace{-1em}
 \begin{center}
  \begin{tabular}{c*{7}{|>{\centering} m{1.5cm}}}
   $x$ & $-2$&$-1$ & 0 & 1 &2 & 3 & 4  \tabularnewline \hline
   &&&&&&& \tabularnewline
   $f(x)$ &$-5$&&&&&& %\tabularnewline
   %&&&&&&& 
  \end{tabular}
 \end{center}
 \item Tracer la courbe repr\'esentative de $f$ dans le rep\`ere ci-dessous :
 %\vspace{-1em}
 \begin{center}
    \def\xmin{-3.1} \def\xmax{5.1} \def\ymin{-6.1} \def\ymax{6.1}
\psset{xunit=2cm,yunit=0.5cm}
\begin{pspicture*}(\xmin,\ymin)(\xmax,\ymax)
\psset{unit=0.5cm}
\psgrid[griddots=10,gridlabels=0pt,gridwidth=.5pt, gridcolor=black, subgridwidth=.3pt, subgridcolor=black, subgriddiv=1](0,0)(-12,\ymin)(20,\ymax)
\psset{xunit=2cm,yunit=0.5cm}
\psaxes[labels=all,labelsep=1pt, Dx=1,Dy=1]{->}(0,0)(\xmin,\ymin)(\xmax,\ymax)
\psdots(0,0)(1,0)(0,1)%
\uput[dl](0,0){$O$}
\uput[u](1,0){$I$}
\uput[r](0,1){$J$}
\uput[ul](\xmax,0){$x$}
\uput[dr](0,\ymax){$y$}

%\psplot[plotpoints=200,algebraic=true]{-2}{4}{-x^2+2*x+3}

\end{pspicture*}\end{center}
\end{enumerate}

 
\end{exo}

\medskip

\hrule

\begin{exo}[3 points]\label{ds2exo3}
Soit $f$ et $g$ deux fonctions d\'efinies sur $\R$ par 
$f:x\longmapsto x^2-4$ et $g:x\longmapsto x^3 -4x$.\\
\`A l'aide de la calculatrice r\'esoudre l'in\'equation $f(x)\geqslant g(x)$.\\
\emph{On ne demande aucune justification.}\\
.\dotfill.\\
.\dotfill.\\
.\dotfill.
 
\end{exo}

\medskip

\hrule


\begin{exo}[5 points]\label{ds3exo1}



On donne l'algorithme ci-contre.





\begin{enumerate}\vspace{-1em}\begin{multicols}{2}
 \item Le faire fonctionner avec les valeurs indiqu\'ees et compl\'eter le tableau ci-dessous.
 \begin{center}
  \begin{tabular}{c|c|c|c}
   $A$ & $B$ & $C$ & Sortie de l'algorithme \\ \hline
   & & & \\
   3 & $-1$ & 27 & \\ \hline
  % & & & \\ \hline
   & & & \\
   12 & 7 & 2 & \\ \hline
   %& & & \\ \hline
   & & & \\
   4,5 & 7,5 & 1,5 & 
  % & & & 
  \end{tabular}

 \end{center}
 \begin{algo}
\begin{small}
\begin{verbatim}
   ENTREES
      A, B, C : nombres
   TRAITEMENT
      SI A>B ALORS M prend la valeur A
             SINON M prend la valeur B
      SI C>M ALORS M prend la valeur C
   SORTIE
      M
\end{verbatim}
\end{small}
\end{algo}
\end{multicols}

 \item Quel est le but de cet algorithme ?\\
 .\dotfill.\\
 .\dotfill.\\
 .\dotfill.
\end{enumerate}


\end{exo}
\hrule


%\hrulefill


 



%\FloatBarrier






\setcounter{chapter}{\thechaptertemp} % remet le numéro de chapitre à ce qu'il était
