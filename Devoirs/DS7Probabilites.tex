\cleardoublepage

\fancyhead{} % Delete current head settings

		\fancyhead[LO]{\footnotesize \em Nom : }
		\fancyhead[RE,RO]{\scriptsize \em Lundi 27 mars 2017 -- 1h00}
		\fancyfoot{}

\setcounter{ds}{7} %c'est le numéro du DS	
\setcounter{chaptertemp}{\thechapter} %stocke le numéro du chapitre courant dans un compteur temporaire
\stepcounter{chapter} % avance le compteur de 1 et surtout remet tous les compteurs dépendant du chapitre à 0, dont les numéros d'exercice
\setcounter{chapter}{\theds} % met le compteur de chapitre au numéro du ds
		
    
\section*{Devoir surveill\'e n°\theds}\label{DS7}
{\centering \large Probabilit\'es}
\addstarredchapter{Devoir surveill\'e n°\theds : Probabilit\'es} 

\medskip

\hrule

\begin{exo}\label{ds7exo1}
On tire au hasard une carte dans un jeu de 52 cartes : 13 cartes de pique (couleur noire), 13 de tr\`efle (couleur noire), 13 de c\oe{}ur (couleur rouge), 13 de carreau (couleur rouge) ; ces 13 cartes sont celles num\'erot\'ees de 1 (as) \`a 10 auxquelles s'ajoutent les 3 figures : valet, dame, roi.
\begin{enumerate}
 \item Quelle est la probabilit\'e de tirer un tr\`efle ?
 \item Quelle est la probabilit\'e de tirer une carte noire ?
 \item Quelle est la probabilit\'e de ne pas tirer un carreau ?
 \item Quelle est la probabilit\'e de tirer une figure (roi, dame ou valet) ?
 \item Quelle est la probabilit\'e de tirer un as ?
 \item Quelle est la probabilit\'e de ne pas tirer un valet noir ?
\end{enumerate}
\end{exo}

\hrule

\begin{exo}
On lance deux d\'es cubiques \'equilibr\'es num\'erot\'es de 1 \`a 6. On note alors le plus grand des deux num\'eros sortis.
\begin{enumerate}
 \item Utiliser un tableau \`a double entr\'ee pour mod\'eliser la situation.
 \item Quel est l'univers $\Omega$ de toutes les issues possibles ?
 \item \'Etablir la loi de probabilit\'e de l'exp\'erience.
\end{enumerate}

\end{exo}

\hrule


\begin{exo}
La porte d'entr\'ee d'un immeuble est muni d'un clavier de trois touches marqu\'ees par les lettres $A$, $B$ et $C$.\\
Le code qui d\'eclenche l'ouverture de la porte est form\'e d'une s\'erie de deux lettres distinctes ou non.
\begin{enumerate}
 \item Recopier et compl\'eter l'arbre suivant qui d\'enombre l'ensemble des codes possibles :
 \begin{center}
  %\usepackage{pstricks,pst-plot,pst-text,pst-tree,pst-eps,pst-fill,pst-node,pst-math}
\psset{nodesep=0mm,levelsep=20mm,treesep=10mm}
\pstree[treemode=R]{\Tdot}
{
\pstree
{\Tdot~[tnpos=a]{$A$}\taput{\small $$}}
{
\Tdot~[tnpos=r]{$A$}\taput{\small $$}
\Tdot~[tnpos=r]{$B$}\taput{\small $$}
\Tdot~[tnpos=r]{$C$}\tbput{\small $$}
}
\pstree
{\Tdot~[tnpos=a]{$B$}\taput{\small $$}}
{
\Tdot~[tnpos=r]{$\ldots$}\taput{\small $$}
\Tdot~[tnpos=r]{$\ldots$}\taput{\small $$}
\Tdot~[tnpos=r]{$\ldots$}\tbput{\small $$}
}
\Tdot~[tnpos=r]{$\ldots$}\tbput{\small $$}
}


 \end{center}
 \item D\'eterminer le nombre de codes diff\'erents possibles.
 \item D\'eterminer la probabilit\'e de chacun des \'ev\`enements suivants :
      \begin{description}
       \item[X :] Le code se termine par $A$.
       \item[Y :] Le code est form\'e de deux lettres diff\'erentes.
       \item[Z :] Le code comporte au moins une fois la lettre $A$.
      \end{description}
\end{enumerate}

\end{exo}

\hrule

\sautpage

\begin{exo}
 Une campagne de pr\'evention routi\`ere s'int\'eresse aux d\'efauts constat\'es sur le freinage et sur l'\'eclairage de 400 v\'ehicules :
 \begin{itemize}
  \item 60 des 400 v\'ehicules pr\'esentent un d\'efaut de freinage $F$ (dont certains pr\'esentent aussi un autre d\'efaut).
  \item 140 des 400 v\'ehicules pr\'esentent un d\'efaut d'\'eclairage $E$ (dont certains pr\'esentent aussi un autre d\'efaut).
  \item 45 v\'ehicules pr\'esentent \`a la fois un d\'efaut de freinage et un d\'efaut d'\'eclairage.
 \end{itemize}
\begin{enumerate}
 \item Recopier puis compl\'eter le diagramme ci-dessous avec des nombres pour repr\'esenter la situation.
 \begin{center}
  \psset{xunit=1cm,yunit=1cm}
\def\xmin{-0} \def\xmax{10.6} \def\ymin{-0} \def\ymax{4}
\begin{pspicture*}(\xmin,\ymin)(\xmax,\ymax)
%\psset{xunit=1cm,yunit=1cm}
%\psgrid[griddots=7,gridlabels=0pt,gridwidth=.3pt, gridcolor=black, subgridwidth=.3pt, subgridcolor=black, subgriddiv=1](0,0)(\xmin,\ymin)(\xmax,\ymax)
%\psset{xunit=1cm,yunit=1cm}
%\psaxes[labels=all,labelsep=1pt, Dx=1,Dy=1]{-}(0,0)(\xmin,\ymin)(\xmax,\ymax)

\psellipse[](3,2)(2,1)
\psellipse[](5.75,2.25)(2,1)
\psellipse[](4.5,2)(4.5,1.5)

\rput(1.75,2){$F$}
\rput(7,2.25){$E$}
\end{pspicture*}
 \end{center}
 \item On choisit un v\'ehicule au hasard parmi ceux qui ont \'et\'e examin\'es. Quelle est la probabilit\'e que :
 \begin{enumerate}
  \item le v\'ehicule pr\'esente un d\'efaut de freinage mais pas de d\'efaut d'\'eclairage ?
  \item le v\'ehicule pr\'esente un d\'efaut d'\'eclairage mais pas de d\'efaut de freinage ?
  \item le v\'ehicule ne pr\'esente aucun des deux d\'efauts ?
  \item le v\'ehicule pr\'esente au moins un des deux d\'efauts ?
 \end{enumerate}

\end{enumerate}

\end{exo}

\hrule

\begin{exo}
 Voici les r\'esultats d'un saondage effectu\'e en 1999 aupr\`es de \np{2000} personnes, \`a propos d'Internet :
 \begin{itemize}
  \item 40\,\% des personnes interrog\'ees d\'eclarent \^etre int\'eress\'ees par Internet ;
  \item 35\,\% des personnes interrog\'ees ont moins de 30 ans et, parmi celles-ci, quatre cinqui\`emes d\'eclarent \^etre int\'eress\'ees par Internet ;
  \item 30\,\% des personnes interrog\'ees ont plus de 60 ans et, parmi celles-ci, 85\,\% ne sont pas int\'eress\'ees par Internet.
 \end{itemize}
\begin{enumerate}
 \item Reproduire et compl\'eter le tableau suivant :
 \begin{center}
  \begin{tabular}{c|c|c|c}
   & int\'eress\'ees par Internet & non int\'eress\'ees par Internet & total \\ \hline
   moins de 30 ans &&& \\ \hline
   de 30 \`a 60 ans &&& \\ \hline
   plus de 60 ans &&& \\ \hline
   total &&& \np{2000}
  \end{tabular}

 \end{center}
 \item On choisit au hasard une personne parmi les \np{2000} interrog\'ees. On suppose que toutes les personnes ont la m\^eme probabilit\'e d'\^etre choisies. On consid\`ere les \'ev\`enements :
 \begin{description}
  \item[A :] \og la personne interrog\'ee a moins de 30 ans \fg{} ;
  \item[B :] \og la personne interrog\'ee est int\'eress\'ee par Internet \fg{}.
 \end{description}
 \begin{enumerate}
  \item Calculer les probabilit\'es $p(A)$ et $p(B)$.
  \item D\'efinir par une phrase l'\'ev\`enement $\overline{A}$ puis calculer $p(\overline{A})$.
  \item D\'efinir par une phrase l'\'ev\`enement $A\cap B$ puis calculer $p(A\cap B)$. En d\'eduire $p(A\cup B)$.
 \end{enumerate}
 
 \item On sait maintenant que la personne interrog\'ee est int\'eress\'ee par Internet.\\
 Quelle est la probabilit\'e qu'elle ait plus de 30 ans ?
\end{enumerate}

\end{exo}








\setcounter{chapter}{\thechaptertemp} % remet le numéro de chapitre à ce qu'il était
