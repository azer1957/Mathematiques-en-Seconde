\chapter{Statistiques discr\`etes} \label{statistiques1}
\minitoc

\fancyhead{} % efface les ent\^etes pr\'ec\'edentes
\fancyhead[LE,RO]{\footnotesize \em \rightmark} % section en ent\^ete
\fancyhead[RE,LO]{\scriptsize \em Seconde} % classe et ann\'ee en ent\^ete

    \fancyfoot{}
		\fancyfoot[RE]{\scriptsize \em \href{http://perpendiculaires.free.fr/}{http://perpendiculaires.free.fr/}}
		\fancyfoot[LO]{\scriptsize \em David ROBERT}
    \fancyfoot[LE,RO]{\textbf{\thepage}}

%\sautpage


\section{Vocabulaire}


\begin{definition}
Une s\'erie statistique est un ensemble d'observations collect\'ees et on a les d\'efinitions suivantes :
\begin{itemize}
	\item \emph{Population} : C'est l'ensemble sur lequel porte une \'etude statistique ;
		  %si elle est trop grande, on s'int\'eresse \`a un \emph{\'echantillon} de cette population ;
	\item \emph{Individu} : C'est un \'el\'ement de la population ;
	\item \emph{Caract\`ere} : C'est ce qu'on observe chez l'individu ;
	\item \emph{Modalit\'e} : Ce sont les diff\'erentes valeurs prises par le caract\`ere ;
	\item La s\'erie statistique est dite \emph{quantitative} quand les modalit\'es sont des nombres %(nombre de fr\`eres et soeurs, dimensions d'une pi\`ece)
		et \emph{qualitative} sinon ; %(candidat pour lequel un individu \`a l'intention de voter)
	\item Dans le cas d'une s\'erie quantitative, celle-ci est dite \emph{discr\`ete} si les modalit\'es sont limit\'ees \`a un ensemble fini de valeurs %(le nombre de fr\`eres et soeurs ne peut \^etre qu'un \'el\'ement de l'ensemble $\{0\,;\,1\,;\,\ldots\,;\,10\}$)
	    et \emph{continue} si les modalit\'es peuvent prendre n'importe quelle valeur dans un intervalle. %(la taille d'un individu)
\end{itemize}
\end{definition}

\begin{exemples*}~
\begin{itemize}
 \item On peut s'int\'eresser \`a une classe (population),
comportant des \'el\`eves (individus) et observer leur nombre de fr\`eres et s\oe{}urs (caract\`ere)
qui peuvent \^etre 0, 1, 2, \ldots (modalit\'es),
ces donn\'ees formant alors une s\'erie statistique quantitative discr\`ete.
 \item On peut s'int\'eresser \`a une cha\^ine d'usine produisant des bras de suspension pour voiture (population),
et observer sur chaque pi\`ece (individu) ses dimensions exactes (caract\`ere)
qui peuvent varier entre 500 et 750 mm (modalit\'es),
ces donn\'ees formant alors une s\'erie statistique quantitative continue.
 \item On peut s'int\'eresser \`a la population fran\c{c}aise (population)
comportant des individus (individus) et estimer leur intention de vote (caract\`ere) pouvant \^etre n'importe
lequel des candidats se pr\'esentant (modalit\'es),
ces donn\'ees formant alors une s\'erie statistique qualitative.
\end{itemize}
\end{exemples*}

\begin{definition}
On a aussi :
\begin{itemize}
	\item \emph{Effectif d'une valeur} : C'est le nombre de fois que la valeur d'un caract\`ere (la modalit\'e) revient dans la s\'erie ;
	\item \emph{Fr\'equence d'une valeur} : C'est l'effectif de la modalit\'e divis\'e par l'effectif total ; elle est comprise entre 0 et 1.
	\item \emph{Classes de valeurs} : s'il y a trop de valeurs diff\'erentes, elles sont rang\'ees par \emph{classe} (intervalle), l'effectif de la classe \'etant alors le nombre de modalit\'es appartenant \`a cet intervalle.
\end{itemize}
\end{definition}

% Pour faire parler ces (souvent longues) s\'eries, il est n\'ecessaire de les r\'esumer : on produit alors \emph{des} statistiques. Tout r\'esum\'e met en \'evidence certaines caract\'eristiques de la s\'erie mais engendre une \emph{perte d'information}, toutes les donn\'ees n'\'etant plus accessibles.

% Le r\'esum\'e peut \^etre un graphique : en Seconde vous avez vu le \emph{diagramme en bâtons} et l'\emph{histogramme} (pour des s\'eries rang\'ees en classes). Nous en verrons deux autres cette ann\'ee.

% Il peut aussi \^etre num\'erique dans le cas d'une s\'erie satistique quantitative. Ces r\'esum\'es num\'eriques sont de deux types : les mesures centrales et les mesures de dispersion.

%\sautpage

\section{Mesures centrales}

\begin{encadrer} \begin{Large}Elles visent \`a r\'esumer la s\'erie par une seule valeur repr\'esentative, d'une certaine mani\`ere, de toutes les valeurs de la s\'erie.                                                                                                                                                \end{Large}\end{encadrer}

\subsection{Mode}

\begin{definition}[Mode]
Le \emph{mode} d'une s\'erie statistique est la donn\'ee la plus fr\'equente de la s\'erie.
\end{definition}

\begin{rmqs}~
\begin{itemize}
	\item S'il y a plusieurs donn\'ees arrivant \`a \'egalit\'e, il y a plusieurs modes.
	\item Si les donn\'ees sont rang\'ees en classe, on parle de \emph{classe modale}.
	\item Le mode est d\'efini aussi bien pour les s\'eries quantitatives que qualitatives.
\end{itemize}
\end{rmqs}

Le mode est un r\'esum\'e sommaire d'une s\'erie qui fournit un type d'information assez limit\'e. Il pourra int\'eresser un publicitaire.

\subsection{Moyenne arithm\'etique}

\begin{definition}[Moyenne arithm\'etique]
La \emph{moyenne arithm\'etique} d'une s\'erie statistique quantitative $S=\{x_1,x_2,\ldots,x_n\}$ est le nombre,
not\'e $\overline{x}$ : \[\overline{x}=\frac{x_1+x_2+\ldots+x_n}{n}\]
\end{definition}


\begin{rmq} De la d\'efinition, on peut d\'eduire que $n\overline{x}=x_1+x_2+\ldots+x_n$, ce qui peut s'interpr\'eter de la mani\`ere suivante : \og La somme de toutes les valeurs de la s\'erie est inchang\'ee si on remplace chaque valeur par $\overline{x}$ \fg. 
%\item Si la s\'erie $S$ comporte $n$ donn\'ees selon $p$ modalit\'es $x_1, x_2, \ldots, x_p$ d'effectifs respectifs (ou de fr\'equences respectives) $n_1$, $n_2$, \ldots, $n_p$, alors $\overline{x}=\frac{n_1x_1+n_2x_2+\ldots+n_px_p}{\underbrace{n_1+n_2+\ldots+n_p}_{\text{effectif total}}}$
\end{rmq}

 La moyenne a des avantages calculatoires : si l'on conna\^it les moyennes et les effectifs de deux s\'eries (ou deux sous-s\'eries), on peut obtenir la moyenne de la s\'erie constitu\'ee l'agr\'egation de ces deux s\'eries. Elle a le d\'efaut d'\^etre sensible aux valeurs extr\^emes.

\subsection{M\'ediane}

\begin{definition*}[M\'ediane dans le cas g\'en\'eral]
On appelle \emph{m\'ediane} d'une s\'erie statistique quantitative tout nombre $m$ tel que :
\begin{itemize}
	\item la moiti\'e au moins des valeurs de la s\'erie est inf\'erieure \`a $m$
	\item la moiti\'e au moins des valeurs de la s\'erie est sup\'erieure \`a $m$
\end{itemize}
\end{definition*}

\begin{rmqs}~
\begin{itemize}
	\item Rappel : math\'ematiquement \og inf\'erieur \fg{} et \og sup\'erieur \fg{} signifient, en fran\c{c}ais, \og inf\'erieur ou \'egal \fg{} et \og sup\'erieur ou \'egal \fg.
	\item On admettra qu'un tel nombre existe toujours.
	\item La m\'ediane partage la s\'erie en deux sous-s\'eries ayant \emph{quasiment} le m\^eme effectif ; \emph{quasiment} car si plusieurs valeurs de la s\'erie sont \'egales \`a la m\'ediane, les donn\'ees inf\'erieures \`a la m\'ediane et les donn\'ees sup\'erieures \`a la m\'ediane ne seront pas forc\'ement en nombre \'egal.
	\item Il faut comprendre la m\'ediane comme \og la valeur du milieu \fg.
\end{itemize}
\end{rmqs}

Plusieurs valeurs peuvent parfois convenir pour la m\'ediane, aussi convient-on de prendre, dans le cadre scolaire\footnote{Les statisticiens, eux, prennent n'importe quel nombre convenant parmi les m\'edianes possibles ; sur des s\'eries de grande taille, ils ont tous le m\^eme ordre de grandeur}, les valeurs, uniques, suivantes :

\begin{definition}[M\'ediane dans le cadre scolaire]
Soit une s\'erie statistique quantitative comportant $n$ donn\'ees : $S=\{x_1,x_2,\ldots,x_i,\ldots,x_n\}$ telles que $x_1\leqslant x_2\leqslant \ldots \leqslant x_n$.
\begin{itemize}
	\item Si $n$ est impair, la $\frac{n+1}{2}^{\text{i\`eme}}$ donn\'ee de la s\'erie est la m\'ediane.
	\item Si $n$ est pair, tout nombre compris entre le $\frac{n}{2}^{\text{i\`eme}}$ \'el\'ement de la s\'erie et le suivant est \textbf{une} m\'ediane ; dans le cadre scolaire \textbf{la} m\'ediane sera la moyenne des deux donn\'ees centrales de la s\'erie :
	      \[m=\frac{\left(\frac{n}{2}\right)^{\text{i\`eme}}+\left(\frac{n}{2}+1\right)^{\text{i\`eme}}}{2}\]
\end{itemize}
\end{definition}

C'est cette m\'ediane qui sera attendue syst\'ematiquement dans les exercices et les \'evaluations.

\medskip

La m\'ediane a l'avantage de ne pas \^etre influenc\'ee par les valeurs extr\^emes. Elle n'a aucun avantage pratique dans les calculs, puisque, pour conna\^itre la m\'ediane d'une s\'erie constitu\'ee de l'agr\'egation de deux s\'eries, il faut n\'ecessairement re-ordonner la nouvelle s\'erie pour trouver sa m\'ediane, qui n'aura pas de lien avec les deux m\'edianes des deux s\'eries initiales.

%\sautpage

\section{Mesures de dispersion}

 \begin{encadrer}\begin{Large}Elles visent \`a indiquer comment les donn\'ees de la s\'erie statistique sont dispers\'ees par rapport aux mesures centrales. \end{Large}\end{encadrer}

\subsection{Valeurs extr\^emes}

\begin{definition}
Les valeurs extr\^emes d'une s\'erie quantitative sont ses valeurs \emph{minimale} et \emph{maximale} et l'\emph{\'etendue} est la diff\'erence entre les valeurs extr\^emes de la s\'erie.
\end{definition}

\subsection{Quartiles}

\begin{definition*}[Quartiles dans le cas g\'en\'eral]
Soit $S$ une s\'erie statistique quantitative.

\begin{itemize}
%\renewcommand{\labelitemii}{$-$}
	\item On appelle \emph{premier quartile}, not\'e $Q_1$, tout r\'eel tel que
			\begin{itemize}
				\item au moins 25\% des valeurs de la s\'erie ont une valeur inf\'erieure ou \'egale \`a $Q_1$
			\item
			au moins 75\% des valeurs de la s\'erie ont une valeur sup\'erieure ou \'egale \`a $Q_1$
			\end{itemize}
	\item On appelle \emph{deuxi\`eme quartile}, not\'e $Q_2$, tout r\'eel tel que
			\begin{itemize}
				\item au moins 50\% des valeurs de la s\'erie ont une valeur inf\'erieure ou \'egale \`a $Q_2$
			\item
			au moins 50\% des valeurs de la s\'erie ont une valeur sup\'erieure ou \'egale \`a $m$
			\end{itemize}
		\item On appelle \emph{troisi\`eme quartile}, not\'e $Q_3$, tout r\'eel tel que
			\begin{itemize}
				\item au moins 75\% des valeurs de la s\'erie ont une valeur inf\'erieure ou \'egale \`a $Q_3$
			\item
			au moins 25\% des valeurs de la s\'erie ont une valeur sup\'erieure ou \'egale \`a $Q_3$
			\end{itemize}
      
	
\end{itemize}
\end{definition*}

\begin{rmqs}~
\begin{itemize}
 \item $Q_2$ est, par d\'efinition, la m\'ediane de la s\'erie.
 \item On admettra que de tels nombres existent toujours.
 \item La m\'ediane partage une s\'erie en deux sous-s\'eries ayant quasiment le m\^eme effectif (environ 50\,\%) ; les premier, troisi\`eme quartiles et la m\'ediane partageront une s\'erie en quatre sous-s\'eries ayant quasiment le m\^eme effectif (environ 25\,\%). 
\end{itemize}

 
\end{rmqs}

Comme pour la m\'ediane, selon le nombre $n$ de donn\'ees dans la s\'erie, il y a parfois plusieurs possibilit\'es. Aussi on convient de prendre, dans le cadre scolaire\footnote{Ce sont aussi ces quartiles que prennent les statisticiens}, syst\'ematiquement les nombres suivants :

\begin{definition}[Quartiles dans le cadre scolaire]
 Soit $S$ une s\'erie statistique quantitative dont les donn\'ees sont ordonn\'ees dans l'ordre croissant. On appelle :
 \begin{itemize}
  \item \emph{premier quartile}, not\'e $Q_1$, \textbf{la premi\`ere valeur de la s\'erie} telle qu'au moins 25\,\% des valeurs de la s\'erie ont une valeur inf\'erieure ou \'egale \`a $Q_1$ ;
  \item \emph{troisi\`eme quartile}, not\'e $Q_3$, \textbf{la premi\`ere valeur de la s\'erie} telle qu'au moins 75\,\% des valeurs de la s\'erie ont une valeur inf\'erieure ou \'egale \`a $Q_3$.
 \end{itemize}
\end{definition}

Ce sont ces quartiles qui seront attendus syst\'ematiquement dans les exercices et les \'evaluations.

\begin{rmq}
 Si l'on adopte le m\^eme type de d\'efinition pour le deuxi\`eme quartile on ne tombe pas forc\'ement sur la valeur de la m\'ediane telle que d\'efinie dans le cadre scolaire.\\ Par exemple la s\'erie $S=\{1\,;\,2\,;\,3\,;\,4\}$ a pour m\'ediane $m=\frac{2+3}{2}=2,5$ et pour deuxi\`eme quartile $Q_2=2$ car c'est la premi\`ere valeur de la s\'erie telle que au moins 50\,\% des valeurs de la s\'erie lui sont inf\'erieures.
\end{rmq}

%\sautpage

La propri\'et\'e suivante permet de trouver ais\'ement $Q_1$ et $Q_3$ :

\begin{prop}
Soit une s\'erie statistique quantitative comportant $n$ donn\'ees : $S=\{x_1,x_2,\ldots,x_i,\ldots,x_n\}$ telles que $x_1\leqslant x_2\leqslant\ldots \leqslant x_n$. Alors :
\begin{itemize}
	\item La donn\'ee de rang $\frac{1}{4}n$ (ou sa valeur approch\'ee par exc\`es \`a l'entier sup\'erieur si $\frac{1}{4}n$ n'est pas un entier) convient toujours comme premier quartile.
	\item La donn\'ee de rang $\frac{3}{4}n$ (ou sa valeur approch\'ee par exc\`es \`a l'entier sup\'erieur si $\frac{3}{4}n$ n'est pas un entier) convient toujours comme troisi\`eme quartile.
	\end{itemize}
\end{prop}

 On l'admettra.

\sautpage

\begin{exemples*} \label{statsexemple}~
\begin{itemize}
 \item S'il y a $n=29$ donn\'ees dans la s\'erie, rang\'ees dans l'ordre croissant :
\begin{itemize}
	\item $\frac{1}{4}\times29=7,25$ donc la huiti\`eme (valeur approch\'ee par exc\`es de 7,25) donn\'ee de la s\'erie convient comme premier quartile ;
	\item $\frac{3}{4}\times29=21,75$
	donc la vingt-deuxi\`eme (valeur approch\'ee par exc\`es de 21,75) donn\'ee de la s\'erie convient comme troisi\`eme quartile.
	\end{itemize}
 \item S'il y a $n=64$ donn\'ees dans la s\'erie, rang\'ees dans l'ordre croissant : 
\begin{itemize}
	\item $\frac{1}{4}\times64=16$
	donc la seizi\`eme donn\'ee de la s\'erie convient comme premier quartile ;
	\item $\frac{3}{4}\times64=48$
	donc la quarante huiti\`eme donn\'ee de la s\'erie convient comme troisi\`eme quartile.
	\end{itemize}
\end{itemize}



\end{exemples*}

\subsection{Interquartiles}

Une fois les premier et troisi\`eme quartiles disponibles, on d\'efinit l'\'ecart et l'intervalle interquartiles de la mani\`ere suivante :

\begin{definition}
 Soit $S$ une s\'erie statistique quantitative et $Q_1$ et $Q_3$ ses premier et troisi\`eme quartiles. On appelle :
 \begin{itemize}
  \item \emph{\'ecart interquartile} la diff\'erence $Q_3 - Q_1$ ;
  \item \emph{intervalle interquartile} l'intervalle $[Q_1 \,;\, Q_3]$.
 \end{itemize}

\end{definition}


%\sautpage




%%%%%%%%%%%%%%%%%%%%%%%%%%%%%%%%%%%%%%%%%%%%%%%%%%


\section{Repr\'esentations graphiques}

Si les mesures centrales et les mesures de dispersion ont pour but de r\'esumer une s\'erie statistique en quelques nombres, les repr\'esentations graphiques, elles, visent \`a la visualiser.

\subsection{Diagramme \`a bâtons}

On consid\`ere la s\'erie :
%\vspace{-1em}
\begin{footnotesize}\begin{center}
\begin{tabular}{|*{22}{c|}}\hline
Valeurs $x_i$ & 0 & 1 & 2& 3 & 4& 5& 6& 7 & 8 & 9  & 10 & 11& 12& 13& 14& 15& 16& 17& 18 & 19& 20 \\ \hline
Effectifs $n_i$ & 3 & 5 & 6& 5 & 6& 7& 7& 10& 13& 20 & 25	& 21& 23& 12& 10& 5 & 7 & 5 & 3  & 2 & 1\\ \hline
\end{tabular}
\end{center}\end{footnotesize}

 On obtient le diagramme \`a bâtons de la figure \ref{batons}, \vpageref{batons}.

\begin{figure}[!hbtp]
\centering
\caption{Diagramme en bâtons}\label{batons}
\def\xmin{-1} \def\xmax{20.6} \def\ymin{-5.6} \def\ymax{25.9}
\psset{xunit=0.75cm,yunit=0.15cm}
\begin{pspicture*}(\xmin,\ymin)(\xmax,\ymax)
%\psgrid[griddots=7,gridlabels=0pt,gridwidth=.3pt, gridcolor=black, subgridwidth=.3pt, subgridcolor=black, subgriddiv=1](0,0)(\xmin,\ymin)(\xmax,\ymax)
\psaxes[labels=all,labelsep=1pt,Dx=1,Dy=5]{->}(\xmax,\ymax)
\psline[linewidth=1.5pt](0,0)(0,3)
\psline[linewidth=1.5pt](1,0)(1,5)
\psline[linewidth=1.5pt](2,0)(2,6)
\psline[linewidth=1.5pt](3,0)(3,5)
\psline[linewidth=1.5pt](4,0)(4,6)
\psline[linewidth=1.5pt](5,0)(5,7)
\psline[linewidth=1.5pt](6,0)(6,7)
\psline[linewidth=1.5pt](7,0)(7,10)
\psline[linewidth=1.5pt](8,0)(8,13)
\psline[linewidth=1.5pt](9,0)(9,20)
\psline[linewidth=1.5pt](10,0)(10,25)
\psline[linewidth=1.5pt](11,0)(11,21)
\psline[linewidth=1.5pt](12,0)(12,23)
\psline[linewidth=1.5pt](13,0)(13,12)
\psline[linewidth=1.5pt](14,0)(14,10)
\psline[linewidth=1.5pt](15,0)(15,5)
\psline[linewidth=1.5pt](16,0)(16,7)
\psline[linewidth=1.5pt](17,0)(17,5)
\psline[linewidth=1.5pt](18,0)(18,3)
\psline[linewidth=1.5pt](19,0)(19,2)
\psline[linewidth=1.5pt](20,0)(20,1)
\end{pspicture*}

\end{figure}

\subsection{Diagrammes bas\'es sur la fr\'equence}

 Les s\'eries statistiques peuvent aussi \^etre repr\'esent\'ees en diagrammes circulaires, semi-circulaires, rectangulaires, etc.
L'aire de chaque modalit\'e devra \^etre proportionnelle \`a l'effectif de cette modalit\'e.
Les fr\'equences permettent d'obtenir assez facilement la part du diagramme qui devra \^etre consacr\'ee \`a chaque modalit\'e.

 Ainsi si on consid\`ere la s\'erie suivante :
\begin{small}\begin{center}
\begin{tabular}{|*{5}{c|}}\hline
Donn\'ees & Blonds 	& Bruns & Ch\^atains & Roux  \\ \hline
Effectifs $n_i$ & 25		& 57		&91		&23 \\ \hline
\end{tabular}
\end{center}            \end{small}

 On a alors :
\begin{small}\begin{center}
\begin{tabular}{|>{\centering}m{4cm}|*{5}{c|}}\hline
$x_i$ & Blonds 	& Bruns & Ch\^atains & Roux & Total \\ \hline
$n_i$ & 29		& 57		&91		&23 & 200\\ \hline
Fr\'equence $f_i$ & $\delair{\frac{29}{200}=0,145}$ & 0,285 & 0,455 & 0,115 & 1 \\ \hline
Part d'un diagramme circulaire & $0,145\times 360 = 52,2\,\degre$ & 102,6\,\degre & 163,8\,\degre & 41,4\,\degre & 360\,\degre \\ \hline
Part d'un diagramme semi-circulaire & $0,145\times 180 = 26,1\,\degre$ & 51,3\,\degre & 81,9\,\degre & 20,7\,\degre & 180\,\degre \\ \hline
Part d'un rectangle de 10\,cm & $0,145\times 10 = 1,45$\,cm & 2,85\,cm &4,55\,cm & 1,15\,cm & 10\,cm \\ \hline
Fr\'equence en pourcentage & $0,145=\frac{14,5}{100}= 14,5$\,\% & 28,5\,\% &45,5\,\% & 11,5\,\% & 100\,\% \\ \hline
\end{tabular}
\end{center}            \end{small}

 On obtient les diagrammes de la figure %ci-dessous.%
 \ref{statssemicirculaires}, \vpageref{statssemicirculaires}.

\begin{figure}[!h]
 \centering
 \caption{Diagrammes circulaire, semi-circulaire et rectangulaire}\label{statssemicirculaires}
 
 \begin{multicols}{2}
%\centering
\def\xmin{-3.25} \def\xmax{3.25} \def\ymin{-3.25} \def\ymax{3.25}
\psset{xunit=1cm,yunit=1cm}
\begin{pspicture*}(\xmin,\ymin)(\xmax,\ymax)
\pswedge(0,0){3}{0}{52.2}
\pswedge(0,0){3}{52.2}{154.8}
\pswedge(0,0){3}{154.8}{318.6}
\pswedge(0,0){3}{318.6}{360}
\rput(1.5;26.1){Blonds}
\rput(1.5;103.5){Bruns}
\rput(1.5;236.7){Ch\^atains}
\rput(1.5;339.3){Roux}
\end{pspicture*}
%\centering
\def\xmin{-4.25} \def\xmax{4.25} \def\ymin{-0.25} \def\ymax{4.25}
\psset{xunit=1cm,yunit=1cm}
\begin{pspicture*}(\xmin,\ymin)(\xmax,\ymax)
\pswedge(0,0){4}{0}{26.1}
\pswedge(0,0){4}{26.1}{77.4}
\pswedge(0,0){4}{77.4}{159.3}
\pswedge(0,0){4}{159.3}{180}
\rput(2;13.05){Blonds}
\rput(2;51.75){Bruns}
\rput(2;118.35){Ch\^atains}
\rput(2;169.65){Roux}
\end{pspicture*}
\end{multicols}
%\vspace{-2em}
%\begin{figure}[!hbtp]


%\begin{center}
\def\xmin{-0.25} \def\xmax{10.25} \def\ymin{-0.25} \def\ymax{1.75}
\psset{xunit=1cm,yunit=1cm}
\begin{pspicture*}(\xmin,\ymin)(\xmax,\ymax)
\pspolygon[linewidth=1pt](0,0)(0,1.5)(1.45,1.5)(1.45,0)
\pspolygon[linewidth=1pt](1.45,0)(1.45,1.5)(4.3,1.5)(4.3,0)
\pspolygon[linewidth=1pt](4.3,0)(4.3,1.5)(8.85,1.5)(8.85,0)
\pspolygon[linewidth=1pt](8.85,0)(8.85,1.5)(10,1.5)(10,0)
\rput(0.725,0.75){Blonds}
\rput(2.875,0.75){Bruns}
\rput(6.575,0.75){Ch\^atains}
\rput(9.425,0.75){Roux}
\end{pspicture*}                %\end{center}
 
 
\end{figure}



%\FloatBarrier
\sautpage


\subsection{Diagramme en boite}

 On peut repr\'esenter graphiquement les valeurs extr\^emes, les quartiles et la m\'ediane par un \emph{diagramme en boite}, appel\'e aussi \emph{boite \`a moustaches}, con\c{c}u de la mani\`ere suivante :
\begin{itemize}
	\item au centre une boite allant du premier au troisi\`eme quartile, s\'epar\'ee en deux par la m\'ediane ;
	\item de chaque côt\'e une moustache allant du minimum au premier quartile pour l'une, et du troisi\`eme quartile au maximum pour l'autre.
\end{itemize}

La figure \ref{chap4moustaches}, \vpageref{chap4moustaches} en est un exemple.

\begin{figure}[!h]
 \centering
 \caption{Exemple de diagramme en boite}\label{chap4moustaches}

 \psset{xunit=0.6cm,yunit=0.6cm}
\begin{pspicture}(-0.3,0)(20.5,3)
\psaxes[labels=all,labelsep=1pt, Dx=1,Dy=1]{->}(0,0)(0,0)(20.5,0)

\psline{*-}(2,2)(5,2)(5,2.5)(7,2.5)(7,1.5)(5,1.5)(5,2)
\psline{*-}(18,2)(15,2)(15,2.5)(7,2.5)(7,1.5)(15,1.5)(15,2)
\uput[d](2,2){min}
\uput[d](18,2){max}
\uput[d](5,1.5){$Q_1$}
\uput[d](7,1.5){$m$}
\uput[d](15,1.5){$Q_3$}

\end{pspicture}

\end{figure}

\medskip

Ce type de diagramme permet une interpr\'etation visuelle et rapide de la dispersion des s\'eries statistiques. Il
permet \'egalement d'appr\'ecier des diff\'erences entre des s\'eries (lorsqu'elles ont des ordres de grandeurs
comparables).

\FloatBarrier

\begin{rmqs}~
\begin{itemize}
	\item La hauteur des boites est arbitraire (on les fait parfois proportionnelles \`a l'effectif total de la s\'erie).
	\item La boite contient les 50\% des donn\'ees centrales.
	%\item On coupe parfois les moustaches de part et d'autre \`a la hauteur du premier et neuvi\`eme d\'ecile ; on fait alors appara\^itre les minimum et maximum par un point.
\end{itemize}
\end{rmqs}



%%%%%%%%%%%%%%%%%%%%%%%%%%%%%%%%%%%%%%%%%%%%%%%%%
%\FloatBarrier


%\sautpage


\section{Exercices}

%\begin{multicols}{2}
\begin{exo}
Dans chaque cas, calculer la moyenne, le mode et la m\'ediane de la s\'erie, et conseiller le narrateur sur la meilleure strat\'egie pour minimiser son r\'esultat aupr\`es de ses parents qui ne connaissent rien en statistique :
\begin{enumerate}
	\item \og{} Je n'ai eu que 8 sur 20 au contrôle de statistiques. Nous sommes 10 en
classe. La meilleure note est 19. Ensuite il y a un 10, quatre 9, un 8 (moi) et trois 2. \fg
	\item \og{} Encore un 8 ! Cette fois les notes sont 2, 3, 4, 5, 7, 8  (moi), 9, 9, 18, 19. \fg
\item \og{} Toujours un 8 ! Cette fois il y a eu trois 7 et un 19, 18, 12, 11, 10, 8(moi) et 2. \fg
\end{enumerate}
\end{exo}


\begin{exo}
On donne la s\'erie suivante : \begin{center}
11, 12,  13, 4, 17, 5, 13, 13, 5, 6, 6, 10, 10, 8, 9, 9, 11, 11, 14, 5, 14, 9, 9, 15, 7, 8, 15.                                                                                                                              \end{center}
\begin{enumerate}
	\item D\'eterminer la moyenne de la s\'erie.
	\item D\'eterminer la m\'ediane et les quartiles de la s\'erie.
	\item Repr\'esenter le diagramme en bo\^ite correspondant.
	\item Quel est l'\'ecart interquartile de la s\'erie ?
	\item Quel est l'intervalle interquartile de la s\'erie ?
\end{enumerate}
\end{exo}

\sautpage

\begin{exo}
 Dans une classe, les notes sont les suivantes : \begin{center}
0, 1, 2, 3, 4, 5, 6, 7, 8, 9, 10, 11, 12, 13, 14, 15, 16, 17, 18, 19, 20.                                                                                                                          \end{center}
\begin{enumerate}
	\item D\'eterminer la moyenne de la s\'erie.
	\item D\'eterminer la m\'ediane et les quartiles de la s\'erie.
	\item Repr\'esenter le diagramme en bo\^ite correspondant.
	\item Que remarque-t-on sur ce diagramme ? Pouvait-on s'y attendre ?
\end{enumerate}
\emph{On gardera en t\^ete l'allure de cette s\'erie \og canonique \fg{} o\`u les donn\'ees sont parfaitement r\'eparties qui pourra servir de r\'ef\'erence pour d\'ecrire les autres s\'eries : plus on s'\'eloigne de ce cas particulier, plus on pourra parler \og d'irr\'egularit\'e \fg{} de dispersion.}
\end{exo}

%\sautpage

\begin{exo}
Soit $S_1$, $S_2$ et $S_3$ les trois s\'eries de notes suivantes :
\begin{itemize}
 \item $S_1 =\{2\,;\,2\,;\,4\,;\,4\,;\,6\,;\,10\,;\,14\,;\,16\,;\,16\,;\,18\,;\,18\}$ ;
 \item $S_2 =\{2\,;\,8\,;\,9\,;\,10\,;\,10\,;\,10\,;\,11\,;\,12\,;\,18\}$ ;
 \item $S_3 =\{2\,;\,5\,;\,5\,;\,5\,;\,5\,;\,6\,;\,6\,;\,7\,;\,9\,;\,10\,;\,11\,;\,13\,;\,14\,;\,14\,;\,15\,;\,15\,;\,15\,;\,15\,;\,18\}$.
\end{itemize}


\begin{enumerate}
 \item D\'eterminer les mesures centrales et les mesures de dispersion les plus adapt\'ees pour d\'ecrire les diff\'erences entre ces trois s\'eries.
 \item Construire les diagrammes qui vous semblent les plus adapt\'es.
 \item Commenter.
\end{enumerate}
\end{exo}%\end{multicols}

\begin{exo}
Un entomologiste a fait des relev\'es sur la taille de 50 courtili\`eres adultes dont voici les r\'esultats :
\begin{center}
33, 35, 36, 36, 37, 37, 37, 38, 38, 38, 39, 39, 39, 39, 40, 40, 40, 40, 40, 41, 41, 41, 41, 41, 41, 41, 42, 42, 42, 42, 42, 42, 43, 43, 43, 43, 44, 44, 44, 44, 45, 45, 45, 46, 46, 47, 47, 48, 48, 50.\end{center}
\begin{enumerate}
	\item Organiser les relev\'es dans le tableau d'effectifs suivant :\\
\footnotesize \begin{tabular}{|m{2.5cm}|*{18}{c|}}\hline
Valeur & 33&34&35&36&37&38&39&40&41&42&43&44&45&46&47&48&49&50\\ \hline
Effectif&&&&&&&&&&&&&&&&&&\\ \hline
Effectif cumul\'e croissant&&&&&&&&&&&&&&&&&&\\\hline
\end{tabular} \normalsize
\item Repr\'esenter les donn\'ees par un diagramme \`a bâtons. Un diagramme circulaire serait-il int\'eressant ?
\item Calculer la moyenne de la s\'erie. D\'eterminer sa m\'ediane.
D\'eterminer les premier et troisi\`eme quartiles.
\item Construire le diagramme en bo\^ite correspondant.
\item Interpr\'eter les r\'esultats obtenus.
\end{enumerate}
\end{exo}

\begin{exo}
Dans une classe de 30 \'el\`eves, la moyenne des 20 filles est 11,5 et la moyenne des 10 gar\c{c}ons est 8,5. Donner la moyenne de classe en prouvant la validit\'e de votre calcul.
\end{exo}

\sautpage

\begin{exo}
On a relev\'e le prix de vente d'un CD et le nombre de CD vendus chez diff\'erents fournisseurs. Les r\'esultats forment une s\'erie statistique \`a une variable donn\'ee par le tableau ci-dessous.
\begin{center}
\begin{tabular}{|*{6}{c|}}\hline
Prix de vente (en \euro) & 15& 16&17&18&19\\ \hline
Nombre de CD vendus & 83&48&32&20&17 \\ \hline
\end{tabular}
\end{center}
\begin{enumerate}
	\item Quelles sont les diff\'erentes valeurs de la s\'erie.
	\item Donner la fr\'equence correspondant \`a chacune de ces valeurs.
	\item Donner la moyenne et la m\'ediane de la s\'erie. Que repr\'esentent ces nombres ?
	\item Repr\'esenter la s\'erie par un diagramme semi-circulaire.
\end{enumerate}
\end{exo}



\begin{exo}\label{chap4entreprises}
Dans deux entreprises $A$ et $B$, les employ\'es sont
class\'es en deux cat\'egories : ouvriers et cadres.\\
Le tableau \ref{chap4entreprisestable}, \vpageref{chap4entreprisestable}, donne la
r\'epartition des employ\'es en fonction de leur
cat\'egorie professionnelle et de leur salaire
mensuel net, en euros.

\begin{table}[!h]
 \centering
 \caption{Salaires des entreprise $A$ et $B$ de l'exercice \ref{chap4entreprises}}\label{chap4entreprisestable}

 %\medskip
 
 \begin{multicols}{2}
 \textbf{Entreprise $A$}

\vskip 0.5em

\begin{tabular}{|*{4}{c|}}
\hline
Salaires & 1\,500 & 2\,500 & 3\,500 \\ \hline
Ouvriers & 114 & 66 & 0 \\ \hline
Cadres & 0 & 8 & 12 \\ \hline
\end{tabular}

\sautcol

\textbf{Entreprise $B$}

\vskip 0.5em

\begin{tabular}{|*{4}{c|}}
\hline
Salaires & 1\,500 & 2\,500 & 3\,500 \\ \hline
Ouvriers & 84 & 42 & 0 \\ \hline
Cadres & 0 & 12 & 12 \\ \hline
\end{tabular}
\end{multicols}
 
 \end{table}


\begin{enumerate}
	\item
\begin{enumerate}
	\item Calculer la moyenne des salaires de tous
les employ\'es de l'entreprise $A$
\item Calculer la moyenne des salaires des
ouvriers de l'entreprise $A$
\item Calculer la moyenne des salaires des
cadres de l'entreprise $A$
\end{enumerate}
\item Faire les m\^emes calculs pour l'entreprise $B$
\item Le P.D.G. de l'entreprise $A$ dit \`a celui de l'entreprise $B$ : \og Mes employ\'es sont mieux pay\'es que les vôtres. \fg

\og Faux \fg r\'epond ce dernier, \og mes ouvriers sont mieux pay\'es et mes cadres \'egalement. \fg\\
Expliquer ce paradoxe.
\end{enumerate}



\end{exo}



\begin{exo}
  Le recensement de 1\,999 a permis d'observer l'existence en France de dix communes
de plus de $200\, 000$ habitants. Le tableau ci-dessous donne la liste de ces communes et
de leurs populations en milliers d'habitants.

\begin{tabular}{cc}
 \begin{minipage}[l]{0.70\linewidth}

\begin{enumerate}
	\item Quelle est l'\'etendue de cette s\'erie statistique ? Que devient l'\'etendue quand on
retire Paris de la liste ?
\item Comparer moyenne et m\'ediane des populations de ces dix villes. Que constate-ton
? Comment expliquer ceci ?
\item Faire de m\^eme si on retire Paris de la liste. Que constatez-vous maintenant ?
\item Que choisiriez-vous entre \og moyenne \'etendue \fg et \og m\'ediane \'etendue \fg pour
r\'esumer cette s\'erie statistique ? Expliquez votre choix.
\end{enumerate}
 \end{minipage}
 &
 \begin{minipage}[r]{0.30\linewidth}
  \begin{center}
\begin{tabular}{|c|c|}
\hline
Paris & $2\,116$ \\ \hline
Marseille & 798 \\ \hline
Lyon & 445 \\ \hline
Toulouse & 391 \\ \hline
Nice & 341 \\ \hline
Nantes & 269 \\ \hline
Strasbourg & 264 \\ \hline
Montpellier & 225 \\ \hline
Bordeaux & 215 \\ \hline
Rennes & 206 \\ \hline
\end{tabular}
\end{center}
 \end{minipage}
\end{tabular}



\end{exo}
%\end{multicols}

\sautpage

\begin{exo}\label{act3}
Les tableaux dont il est question dans cet exercice sont ceux de la page \vpageref{tableact3}.
\begin{enumerate}
	\item Le tableau 1 pr\'esente la r\'epartition des salaires mensuels dans une entreprise (\emph{source : DoC TICE-MEN}).\\
Calculer la moyenne des salaires de cette entreprise et d\'eterminer la m\'ediane.
\item Une erreur a \'et\'e commise dans les relev\'es : les effectifs correspondant aux salaires de 1\,100\,\euro{} et 1\,400\,\euro{} ont
\'et\'e permut\'es. Les donn\'ees exactes sont celles du tableau 2.\\
Comparer les moyenne et m\'ediane de cette s\'erie \`a celles de la pr\'ec\'edente. Les variations \'etaient-elles
pr\'evisibles ?
\item R\^evons $\ldots$ 35 personnes ont un salaire de 3\,400\,\euro{} et l'effectif total est inchang\'e.
Utiliser le tableau 3 pour imaginer une r\'epartition des effectifs telle que la m\'ediane ne soit pas modifi\'ee :
comment va varier la moyenne ?\\
Quelle r\'epartition des effectifs respectant ces contraintes donnera la moyenne des salaires la plus \'elev\'ee
(calculer alors cette moyenne) ?
Quelle r\'epartition des effectifs respectant ces contraintes donnera la moyenne des salaires la moins \'elev\'ee
(calculer alors cette moyenne) ?\\
Mathias pr\'etend qu'avec 35 personnes ayant un salaire de 3\,400\,\euro{}, la moyenne est obligatoirement plus
\'elev\'ee. A-t-il raison ?
\end{enumerate}

\begin{table}[!h]
\centering
\caption{Donn\'ees de l'exercice \ref{act3}}\label{tableact3}


\begin{multicols}{3}
\centering
\footnotesize
\textbf{Tableau 1}
\vskip 0.5em
\begin{tabular}{|*{3}{m{1.3cm}|}}
\hline
Salaire en \euro{} & Effectif & Effectif cumul\'e croissant \\ \hline
$1\,100$ &18 & 18\\ \hline
$1\,200$ & 15& 33\\ \hline
$1\,300$ & 20&53 \\ \hline
$1\,400$ & 10&63 \\ \hline
$1\,500$ & 25&88 \\ \hline
$1\,600$ & 12&100 \\ \hline
$1\,700$ & 4&104 \\ \hline
$1\,800$ & 5&109 \\ \hline
$1\,900$ & 3&112 \\ \hline
$2\,000$ & 2&114 \\ \hline
$2\,100$ & 6&120 \\ \hline
$2\,200$ & 7&127 \\ \hline
$2\,300$ & 0&127 \\ \hline
$2\,400$ & 2&129 \\ \hline
$2\,500$ & 0&129 \\ \hline
$2\,600$ & 3&132 \\ \hline
$2\,700$ & 0&132 \\ \hline
$2\,800$ & 3&135 \\ \hline
$2\,900$ & 0&135 \\ \hline
$3\,000$ & 0&135 \\ \hline
$3\,100$ & 3&138 \\ \hline
$3\,200$ & 0&138 \\ \hline
$3\,300$ & 5&143 \\ \hline
$3\,400$ & 8&151  \\ \hline
\end{tabular}

\sautcol

\textbf{Tableau 2}
\vskip 0.5em
\begin{tabular}{|*{3}{m{1.3cm}|}}
\hline
Salaire en \euro{} & Effectif & Effectif cumul\'e croissant \\ \hline
$1\,100$ &\textbf{10}& \\ \hline
$1\,200$ & 15& \\ \hline
$1\,300$ & 20& \\ \hline
$1\,400$ & \textbf{18}& \\ \hline
$1\,500$ & 25& \\ \hline
$1\,600$ & 12& \\ \hline
$1\,700$ & 4& \\ \hline
$1\,800$ & 5& \\ \hline
$1\,900$ & 3& \\ \hline
$2\,000$ & 2& \\ \hline
$2\,100$ & 6& \\ \hline
$2\,200$ & 7& \\ \hline
$2\,300$ & 0& \\ \hline
$2\,400$ & 2& \\ \hline
$2\,500$ & 0& \\ \hline
$2\,600$ & 3& \\ \hline
$2\,700$ & 0& \\ \hline
$2\,800$ & 3& \\ \hline
$2\,900$ & 0& \\ \hline
$3\,000$ & 0& \\ \hline
$3\,100$ & 3& \\ \hline
$3\,200$ & 0& \\ \hline
$3\,300$ & 5& \\ \hline
$3\,400$ & 8&  \\ \hline
\end{tabular}

\sautcol

\textbf{Tableau 3}
\vskip 0.5em
\begin{tabular}{|*{3}{m{1.3cm}|}}
\hline
Salaire en \euro{}& Effectif & Effectif cumul\'e croissant \\ \hline
$1\,100$ & & \\ \hline
$1\,200$ & & \\ \hline
$1\,300$ & & \\ \hline
$1\,400$ & & \\ \hline
$1\,500$ & & \\ \hline
$1\,600$ & & \\ \hline
$1\,700$ & & \\ \hline
$1\,800$ & & \\ \hline
$1\,900$ & & \\ \hline
$2\,000$ & & \\ \hline
$2\,100$ & & \\ \hline
$2\,200$ & & \\ \hline
$2\,300$ & & \\ \hline
$2\,400$ & & \\ \hline
$2\,500$ & & \\ \hline
$2\,600$ & & \\ \hline
$2\,700$ & & \\ \hline
$2\,800$ & & \\ \hline
$2\,900$ & & \\ \hline
$3\,000$ & & \\ \hline
$3\,100$ & & \\ \hline
$3\,200$ & & \\ \hline
$3\,300$ & & \\ \hline
$3\,400$ & \textbf{35}&\textbf{151} \\ \hline
\end{tabular}

\normalsize
\end{multicols}
\end{table}
\end{exo}

\sautpage

\begin{multicols}{2}

\begin{exo}
 Des salari\'es d'une entreprise se sont r\'eunis dans un restaurant o\`u ils sont les seuls clients pour f\^eter le changement de leur grille de salaire : d\'esormais ils touchent tous 1\,700\,\euro{} par mois.
 \begin{enumerate}
  \item Quelle est la moyenne et la m\'ediane de la s\'erie constitu\'ee par leurs salaires.
  \item Bill Gates, qui a un salaire bien plus grand, entre dans le restaurant : que devient la moyenne et la m\'ediane de la s\'erie constitu\'ee par leurs salaires et celui de Bill Gates.
 \end{enumerate}

\end{exo}

\begin{exo}
 D'apr\`es Wikip\'edia, le salaire m\'edian des salari\'es de 25 \`a 55 ans en France en 2008 \'etait de 1\,655\,\euro{} nets et le salaire moyen de 2\,069\,\euro{} nets.\\
 Conjecturer ce qui peut expliquer cette diff\'erence.
\end{exo}

\begin{exo}
Trois s\'eries statistiques, comportant 10 donn\'ees chacune, ont les param\`etres
suivants :
\begin{itemize}
	\item S\'erie A : Minimum 10 ; maximum 50 ; moyenne 28 ; m\'ediane 20.
	\item S\'erie B : Minimum 10 ; maximum 50 ; moyenne 30 ; m\'ediane 30.
	\item S\'erie C : Minimum 10 ; maximum 50 ; moyenne 21,5 ; m\'ediane 25.
\end{itemize}
Conjecturer pour chacune de ces s\'eries comment peuvent \^etre r\'eparties les
donn\'ees.
\end{exo}

\sautcol

%\sautpage

\begin{exo}\label{chap4troisclassesenonce}
Le tableau ci-dessous %\ref{chap4troisclasses}, \vpageref{chap4troisclasses}, 
est le relev\'e de trois s\'eries de notes obtenues en math\'ematiques dans des classes de
seconde (\`a effectifs tr\`es r\'eduits) lors d'un contrôle sur les statistiques.

%\begin{table}[!h]
% \centering
% \caption{S\'eries de l'exercice \ref{chap4troisclassesenonce}}\label{chap4troisclasses}

% \medskip
\begin{center}
\begin{small}
\begin{tabular}{|c|c|c|}
\hline
Classe 1 & Classe 2 & Classe 3 \\ \hline
2&2&8\\ \hline
5&3&8\\ \hline
6&4&9\\ \hline
7&4&9\\ \hline
9&4&10\\ \hline
10&9&10\\ \hline
10&10&10\\ \hline
12&12&12\\ \hline
13&12&13\\ \hline
14&12&15\\ \hline
15&12&15\\ \hline
15&12&16\\ \hline
16&13&17\\ \hline
19&13&18\\ \hline
\end{tabular}             \end{small}\end{center}
%\end{table}

\begin{enumerate}
	\item D\'eterminer la note m\'ediane de chaque classe.
	\item \textbf{SANS LA CALCULER}, conjecturer pour chaque s\'erie si la moyenne sera
sup\'erieure, inf\'erieure ou proche de la m\'ediane.
\end{enumerate}



\end{exo}
\end{multicols}

%\sautpage

\begin{exo}
Le tableau suivant donne les effectifs des notes obtenues dans une classe en Math\'ematiques et en Histoire-G\'eographie :
\begin{center}
\begin{small}\begin{tabular}{|*{22}{c|}}
\hline
Notes & 0 & 1 & 2 & 3 & 4 & 5 & 6 & 7&  8 & 9 &10 &11 &12& 13 &14& 15 &16& 17& 18 &19 &20 \\ \hline
Maths& 0& 0 &0 &0 &1& 0& 1 &1 &3 &4& 4& 1& 3& 2&2 &1 &1& 0& 0& 0& 0\\ \hline
H.-G. &0 &1& 0& 0 &2 &0 &1 &2 &1& 1& 4& 2& 2& 0 &3& 2 &1 &0& 1& 0& 1\\ \hline
\end{tabular}\end{small}
\end{center}
 Le but de l'exercice est de comparer la dispersion des notes en Maths et en Histoire-G\'eographie.

\begin{enumerate}
	\item Calculer $\overline{x}$ et $\overline{x'}$ les moyennes respectives de Maths et d'Histoire-G\'eographie.
	\item Calculer m\'ediane $m_e$ et quartiles $Q_1$ et $Q_3$ en Maths.
	\item Calculer m\'ediane $m'_e$ et quartiles $Q'_1$ et $Q'_3$ en Histoire-G\'eographie.
	\item Repr\'esenter les diagrammes en bo\^ite des notes en Maths et en Histoire-G\'eographie.
	\item Interpr\'eter les r\'esultats obtenus.
\end{enumerate}

\end{exo}


%\begin{exo}
%Le tableau suivant donne les effectifs des notes obtenues dans une classe en
%Math\'ematiques et en Sciences de la vie et de la terre :
%\begin{center}
%\begin{tabular}{|*{22}{c|}}
%\hline
%Note & 0 &1 &2 &3 &4 &5 &6 &7 &8 &9 &10 &11 &12 &13 &14 &15 &16 &17 &18 &19 &20\\ \hline
%Maths& 0 &1 &0 &2 &0 &1 &3 &0 &5 &0 &2 &1 &1 &0 &0 &0 &2 &1 &1 &0 &2\\ \hline
%SVT& 0 &1& 0 &0 &0 &0 &0& 4& 4& 1 &0 &3 &1 &4 &2 &0 &1 &0 &0 &0 &1\\ \hline
%\end{tabular}
%\end{center}%\vspace{-1em}
%\begin{enumerate}
%	\item Calculer $\overline{x}$ et $\overline{x'}$ les moyennes respectives de Maths et de SVT.
%	\item Calculer m\'ediane $m_e$ et quartiles $Q_1$ et $Q_3$ en Maths.
%	\item Calculer m\'ediane $m'_e$ et quartiles $Q'_1$ et $Q'_3$ en SVT.
%	\item Interpr\'eter les r\'esultats obtenus.
%\end{enumerate}
%\end{exo}

%\end{multicols}

%\sautpage


