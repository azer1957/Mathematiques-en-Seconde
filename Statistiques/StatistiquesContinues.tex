\chapter{Statistiques continues} \label{statistiquescontinues}
\minitoc

\fancyhead{} % efface les entêtes précédentes
\fancyhead[LE,RO]{\footnotesize \em \rightmark} % section en entête
\fancyhead[RE,LO]{\scriptsize \em Seconde} % classe et année en entête

    \fancyfoot{}
		\fancyfoot[RE]{\scriptsize \em \href{http://perpendiculaires.free.fr/}{http://perpendiculaires.free.fr/}}
		\fancyfoot[LO]{\scriptsize \em David ROBERT}
    \fancyfoot[LE,RO]{\textbf{\thepage}}

%\sautpage



\section{Un exemple}


%\sautpage


%\begin{act}
Lors d'une enquête portant sur 1\,300 personnes, on a demandé le temps passé par jour devant le téléviseur. Les
données relevées ont été regroupées par classe car les 1\,300 données sont en trop grand nombre pour être
manipulées toutes ensembles.
On a obtenu le tableau suivant :
\begin{center}
\begin{tabular}{|*{8}{c|}}
\hline
Temps (h) & $[0\, ;\, 1[$ &$[1 \, ;\, 2[$ &$[2 \, ;\, 3[$ &$[3 \, ;\, 4[$ &$[4 \, ;\, 5[$& $[5 \, ;\, 6[$ &$[6 \, ;\, 7]$\\ \hline
Effectif& 170& 309& 432& 221& 103& 41& 24\\ \hline
\end{tabular}
\end{center}
Les 1\,300 données ne sont alors plus accessibles dans le détail. Peut-on malgré tout obtenir de ce tableau les
paramètres statistiques de la série (valeurs extrêmes, moyenne, médiane, etc.) ou, au moins, des valeurs
approchées ?

%\medskip
\subsection{Valeurs extrêmes}
	\begin{enumerate}
	\item Peut-on obtenir les valeurs minimale et maximale de la série ?
Si oui les donner.
Sinon donner une valeur (la plus grande possible) dont on est sûr qu'elle est plus petite que toutes les
données de la série et une valeur (la plus petite possible) dont on est sûr qu'elle est plus grande que toutes
les données de la série.
\item Peut-on obtenir l'étendue de la série ?
Si oui la donner.
Sinon donner l'étendue maximale que peut avoir la série.
\end{enumerate}

\sautpage

\subsection{Moyenne}
\begin{enumerate}
	\item Expliquer pourquoi on ne peut pas calculer la moyenne exacte du temps passé par jour devant la télévision.
	\item Pour obtenir une valeur approchée de la moyenne, on considère que toutes les données d'une classe sont
égales au centre de la classe.\\
Ainsi, par exemple, nous considèrerons que les 170 données de la classe $[0\, ;\, 1[$ sont égales à $\frac{0+1}{2}=0,5$.\\
Compléter alors le tableau suivant et en déduire une valeur approchée de la moyenne :
\begin{center}
\begin{tabular}{|*{8}{c|}}
\hline
Temps (h) & $[0\, ;\, 1[$ &$[1 \, ;\, 2[$ &$[2 \, ;\, 3[$ &$[3 \, ;\, 4[$ &$[4 \, ;\, 5[$& $[5 \, ;\, 6[$ &$[6 \, ;\, 7]$\\ \hline
Centre de la
classe & & & & & & & \\ \hline
Effectif& 170& 309& 432& 221& 103& 41& 24\\ \hline
\end{tabular}
\end{center}
\end{enumerate}

\subsection{Médiane}
\begin{enumerate}
	\item Quel est le rang de la médiane de cette série ?
	\item Expliquer pourquoi on ne peut pas savoir le temps médian exact (la médiane exacte de cette série) passé
par jour devant la télévision.
\item Compléter le tableau suivant :
\begin{center}
\begin{tabular}{|m{4cm}|*{7}{c|}}
\hline
Temps passé devant la télévision inférieur à & 1h &2h &3h &4h &5h& 6h &7h\\ \hline
Effectif& 170& 309& 432& 221& 103& 41& 24\\ \hline
Effectifs cumulés croissants& & & & & & & \\ \hline
\end{tabular}
\end{center}
\item À l'aide de ces effectifs cumulés croissants et du 1., déterminer à quelle classe appartient la médiane exacte
de cette série.
\item Cette donnée est souvent trop approximative pour être utile en statistique et l'on a souvent besoin d'une
estimation plus précise.
On l'obtient avec un graphique.
\begin{enumerate}
	\item Représenter le tableau sur un graphique en indiquant en abscisse les temps passés devant la
télévision (1 h = 2 cm) et en ordonnée les effectifs cumulés croissants (100 = 1 cm).
\item À l'aide de ce graphique et du 1., déterminer une valeur approchée de la médiane de cette série.
\end{enumerate}
\end{enumerate}

\subsection{Mode}
Lorsque les données sont regroupées par classes de même taille le mode n'est pas accessible mais la classe dont
l'effectif et le plus grand est appelée classe modale.
Lorsque les classes ne sont pas de la même taille, il existe des moyens d'estimer celle qui est modale, mais cette
compétence n'est pas au programme de la Seconde.

\sautpage

\section{Exercices}


\begin{exo}
%\vspace{-1em}\begin{multicols}{2}
Le tableau ci-contre donne une répartition des salaires mensuels en euros des employés dans une entreprise.

\begin{enumerate}
	\item Quel est le nombre d'employés de l'entreprise ?
	\item Quel est le nombre d'employés touchant un salaire mensuel supérieur ou égal à 1\,200 \euro{} ?
	\item Estimer le salaire moyen et le salaire m\'edian des employés de l'entreprise.
\end{enumerate}
%\sautcol

\begin{center}
\begin{tabular}{|*{2}{c|}}\hline
Salaire & Effectif \\ \hline
$[1\,000\,;\,1\,200[$ & 326 \\ \hline
$[1\,200\,;\,1\,500[$ & 112 \\ \hline
$[1\,500\,;\,2\,000[$ & 35 \\ \hline
$[2\,000\,;\,3\,000[$ & 8 \\ \hline
$[3\,000\,;\,10\,000[$& 3 \\ \hline
\end{tabular}
\end{center}

%\end{multicols}\vspace{-1em}
\end{exo}



\begin{exo}
%\vspace{-1em}\begin{multicols}{2}
Dans une petite ville fictive où la taxe d'habitation est proportionnelle à la superficie de
l'habitation, la répartition des habitations selon leur superficie est la suivante :
\begin{center}
\begin{tabular}{|c|c|}
\hline
Superficie en m$^2$ & Effectif \\ \hline
$[10\,;\,40[$ & 14 \\ \hline
$[40\,;\,70[$ & 24 \\ \hline
$[70\,;\,100[$ & 54 \\ \hline
$[100\,;\,120[$ & 64 \\ \hline
$[120\,;\,140[$ & 32 \\ \hline
$[140\,;\,170[$ & 12 \\ \hline
\end{tabular}
\end{center}
%\sautcol
\begin{enumerate}
	\item Déterminer une valeur approchée de la superficie moyenne des
habitations de cette ville.
\item Un membre du conseil municipal propose d'exonérer la moitié des personnes : celles dont les
habitations ont les superficies les plus faibles. Une personne dont
l'appartement mesure 80 m$^2$ serait-elle exonérée ? Une personne dont
l'appartement mesure 110 m$^2$ serait-elle exonérée ?
\item Un autre membre du conseil municipal propose d'exonérer le quart des personnes : celles dont les habitations ont les superficies les
plus faibles. Une personne dont l'appartement mesure 80 m$^2$ serait-elle exonérée ?
\end{enumerate}%\end{multicols}\vspace{-1em}
\end{exo}

\sautpage

\begin{exo}
%\vspace{-1em}\begin{multicols}{2}
Dans deux entreprises $A$ et $B$, les employés sont
classés en deux catégories : ouvriers et cadres.\\
Les deux tableaux qui suivent donnent la
répartition des employés en fonction de leur
catégorie professionnelle et de leur salaire
mensuel net, en euros.
On suppose qu'à l'intérieur de chaque classe, la
répartition est régulière.

\begin{small}\begin{center}

\textbf{Entreprise $A$}

\begin{tabular}{|*{4}{c|}}
\hline
Salaires & $[1\,000 \,;\, 2\,000[$ & $[2\,000 \,;\, 3\,000[$ & $[3\,000 \,;\, 4\,000[$ \\ \hline
Ouvriers & 114 & 66 & 0 \\ \hline
Cadres & 0 & 8 & 12 \\ \hline
\end{tabular}

\vskip 0.5em

\textbf{Entreprise $B$}

\begin{tabular}{|*{4}{c|}}
\hline
Salaires & $[1\,000 \,;\, 2\,000[$ & $[2\,000 \,;\, 3\,000[$ & $[3\,000 \,;\, 4\,000[$ \\ \hline
Ouvriers & 84 & 42 & 0 \\ \hline
Cadres & 0 & 12 & 12 \\ \hline
\end{tabular}
\end{center}            \end{small}\vspace{-1em}
\begin{enumerate}
	\item
\begin{enumerate}
	\item Calculer la moyenne des salaires de tous
les employés de l'entreprise $A$
\item Calculer la moyenne des salaires des
ouvriers de l'entreprise $A$
\item Calculer la moyenne des salaires des
cadres de l'entreprise $A$
\end{enumerate}
\item Faire les mêmes calculs pour l'entreprise $B$
\item Le P.D.G. de l'entreprise $A$ dit à celui de l'entreprise $B$ : \og Mes employés sont mieux payés que les vôtres. \fg

\og Faux \fg répond ce dernier, \og mes ouvriers sont mieux payés et mes cadres également. \fg\\
Expliquer ce paradoxe.
\end{enumerate}

%\end{multicols}\vspace{-1em}
\end{exo}


\begin{exo}
%\vspace{-1em}\begin{multicols}{2}
Lors d'une étude d'une population de rats, K. Miescher a observé l'évolution d'une population de 144 rats. 

Le tableau ci-contre indique la durée de vie (en mois) des rats.

Ainsi, un seul rat a vécu entre 10 et 15 mois, trois ont vécu entre 15 et 20 mois, neuf entre 20 et 25 mois etc.
On suppose que, dans chaque classe, la répartition est régulière.
\begin{enumerate}
	\item \'Evaluez l'étendue de cette série
	\item \'Evaluez la moyenne de la durée de vie d'un rat dans cette population
	\item Quelle est le rang de la durée de vie m\'ediane d'un rat dans cette population ?\\
	      \`A l'aide du polygone des effectifs cumul\'es croissants,\'evaluez le valeur de la
médiane ?
\item En observant la moyenne et la médiane, quel commentaire peut-on faire ?
\end{enumerate}

%\sautcol
%\vspace{1cm}
\begin{center}
%\begin{small}
\begin{tabular}{|c|c|}
\hline
Durée de vie (en mois) & Effectif \\ \hline
$[10\,;\,15[$ & 1 \\ \hline
$[15\,;\,20[$ & 3 \\ \hline
$[20\,;\,25[$ & 9 \\ \hline
$[25\,;\,28[$ & 12 \\ \hline
$[28\,;\,30[$ & 13 \\ \hline
$[30\,;\,32[$ & 20 \\ \hline
$[32\,;\,34[$ & 23 \\ \hline
$[34\,;\,36[$ & 26 \\ \hline
$[36\,;\,38[$ & 22 \\ \hline
$[38\,;\,40[$ & 11 \\ \hline
$[40\,;\,42[$ & 3 \\ \hline
$[42\,;\,43[$ & 1 \\ \hline
\end{tabular}
%\end{small}
\end{center}%\vspace{-1em}
%\end{multicols}\vspace{-1em}
\end{exo}









