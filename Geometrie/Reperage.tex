\chapter{Rep\'erage} \label{reperes}
\minitoc

\fancyhead{}
\fancyfoot{} % efface les entêtes précédentes
\fancyhead[LE,RO]{\footnotesize \em \rightmark}
		\fancyhead[RE,LO]{\scriptsize \em Seconde}
		\fancyfoot[RE]{\scriptsize \em \href{http://perpendiculaires.free.fr/}{http://perpendiculaires.free.fr/}}
		\fancyfoot[LO]{\scriptsize \em David ROBERT}
    \fancyfoot[LE,RO]{\textbf{\thepage}}


\section{Rep\`ere d'une droite}

\begin{definition}
 Soit $d$ une droite, $O$ et $I$ deux points distincts de cette droite, alors $(O,I)$ est appel\'e \emph{rep\`ere} de la droite $d$ ; $O$ est appel\'e \emph{origine} du rep\`ere et $OI$ est appel\'e \emph{unit\'e} du rep\`ere.
\end{definition}



\begin{prop}
 Soit $d$ une droite munie du rep\`ere $(O,I)$, alors tout point $M$ de la droite est associ\'e \`a un unique nombre $x$ d\'efini par :
\vspace{-1em}\begin{multicols}{2}\begin{itemize}
 \item $x=\frac{OM}{OI}$ si $M\in[OI)$ ;
 \item $x=-\frac{OM}{OI}$ si $M\notin[OI)$.
\end{itemize}\end{multicols}%\vspace{-1em}
$x$ est appel\'e \emph{abscisse} de $M$.
\end{prop}

\noindent On l'admettra.

\begin{exemple*}
Sur la droite $d$ ci-dessous, les points $O$ et $I$ sont distincts donc $(O,I)$ est un rep\`ere de $d$.\\
$M\in[OI)$ est tel que $OM=4OI$ donc son abscisse est 4.\\
$N\notin[OI)$ est tel que $ON=1,5OI$ donc son abscisse est $-1,5$.
\begin{center}
\def\xmin{0} \def\xmax{10} \def\ymin{-2} \def\ymax{0}
\begin{pspicture*}(\xmin,\ymin)(\xmax,\ymax)
%\psgrid[griddots=10,gridlabels=0pt,gridwidth=.3pt, gridcolor=black, subgridwidth=.3pt, subgridcolor=black, subgriddiv=1](0,0)(-6,-2)(14,6)
%\psaxes[labels=all,labelsep=1pt, Dx=1,Dy=1]{-}(0,0)(\xmin,\ymin)(\xmax,\ymax)
%\uput[dl](0,0){$O$}
%\pcline[linewidth=1pt]{->}(0,0)(1,0) \uput[d](0.5,0){\small $\vec i$}
%\pcline[linewidth=1pt]{->}(0,0)(0,1) \uput[l](0,0.5){\small $\vec j$}
\psplot[algebraic=true]{\xmin}{\xmax}{0.2*x-2}
\psdots(5,-1)(6,-0.8)(9,-0.2)(3.5,-1.3)
\uput[dr](5,-1){$O$}
\uput[dr](6,-0.8){$I$}
\uput[dr](9,-0.2){$M$}
\uput[dr](3.5,-1.3){$N$}
\uput[u](1,-1.6){$d$}
\end{pspicture*}                \end{center}
\end{exemple*}

%\sautpage

\section{Rep\`ere d'un plan}

\subsection{D\'efinitions}
\begin{definition}
 Soit $P$ un plan, $O$, $I$ et $J$ trois points non align\'es de ce plan, alors $(O,I,J)$ est appel\'e \emph{rep\`ere} du plan ; $O$ est appel\'ee \emph{origine} du rep\`ere et les droites $(OI)$ et $(OJ)$ sont appel\'ees \emph{axes} du rep\`ere.
\end{definition}



\noindent Soit $P$ un plan muni du rep\`ere $(O,I,J)$, alors, pour tout point $M$ du plan,
 il existe deux uniques points $M_x$ et $M_y$ tels que
 $M_x\in(OI)$, $M_y\in(OJ)$ et $OM_xMM_y$ parall\'elogramme (on l'admettra).\\
 On note $x$ l'abscisse de $M_x$ sur la droite $(OI)$ munie du rep\`ere $(O,I)$ et
 $y$ l'abscisse de $M_y$ sur la droite $(OJ)$ munie du rep\`ere $(O,J)$.

\begin{center}
\def\xmin{-2} \def\xmax{10} \def\ymin{-2} \def\ymax{6}
\begin{pspicture*}(\xmin,\ymin)(\xmax,\ymax)
%\psgrid[griddots=10,gridlabels=0pt,gridwidth=.3pt, gridcolor=black, subgridwidth=.3pt, subgridcolor=black, subgriddiv=1](0,0)(-6,-2)(14,6)
%\psaxes[labels=all,labelsep=1pt, Dx=1,Dy=1]{-}(0,0)(\xmin,\ymin)(\xmax,\ymax)
%\uput[dl](0,0){$O$}
%\pcline[linewidth=1pt]{->}(0,0)(1,0) \uput[d](0.5,0){\small $\vec i$}
%\pcline[linewidth=1pt]{->}(0,0)(0,1) \uput[l](0,0.5){\small $\vec j$}
\psplot[algebraic=true]{\xmin}{\xmax}{0.4*x-1}
\psplot[algebraic=true]{\xmin}{\xmax}{2*x-1}
\psdots(0,-1)(3,0.2)(1,1)(8,5)(6.25,1.5)(1.75,2.5)
\uput[dr](0,-1){$O$}
\uput[dr](3,0.2){$I$}
\uput[ul](1,1){$J$}
\uput[ur](8,5){$M$}
\uput[dr](6.25,1.5){$M_x$}
\uput[ul](1.75,2.5){$M_y$}
\psline[linestyle=dashed](6.25,1.5)(8,5)(1.75,2.5)
\end{pspicture*}                \end{center}

\noindent On a alors :

\begin{prop}
 Soit $P$ un plan muni d'un rep\`ere $(O,I,J)$, alors tout point $M$ de ce plan est associ\'e \`a un unique couple $(x\,;\,y)$, d\'efini ci-dessus,
 appel\'e \emph{coordonn\'ees} de $M$.\\
 $x$ est appel\'e \emph{abscisse} de $M$ et $y$ est appel\'e \emph{ordonn\'ee} de $M$.
\end{prop}

\noindent On l'admettra.

\begin{exemple*}
Sur le sch\'ema ci-dessus, $x=3,125$ et $y=1,5$ donc les coordonn\'ees de $M$ sont $(3,125\,;\,1,5)$.\\
L'abscisse de $M$ est 3,125, l'ordonn\'ee de $M$ est 1,5.
\end{exemple*}

%\sautpage

\subsection{Types de rep\`eres}

\begin{definition}
 Soit $P$ un plan muni d'un rep\`ere $(O,I,J)$.
\begin{itemize}
 \item Si le triangle $OIJ$ est quelconque, le rep\`ere est dit \emph{quelconque}.
 \item Si le triangle $OIJ$ est rectangle en $O$, le rep\`ere est dit \emph{orthogonal}.
 \item Si le triangle $OIJ$ est isoc\`ele en $O$, le rep\`ere est dit \emph{norm\'e}.
 \item Si le triangle $OIJ$ est rectangle et isoc\`ele en $O$, le rep\`ere est dit \emph{orthonorm\'e}.
\end{itemize}
\end{definition}

\begin{tabular}{ccc}
 Rep\`ere orthogonal & Rep\`ere norm\'e & Rep\`ere orthonorm\'e \\
\psset{xunit=0.5cm,yunit=0.5cm}
\def\xmin{-4} \def\xmax{6} \def\ymin{-4} \def\ymax{6}
\begin{pspicture*}(\xmin,\ymin)(\xmax,\ymax)
%\psgrid[griddots=10,gridlabels=0pt,gridwidth=.3pt, gridcolor=black, subgridwidth=.3pt, subgridcolor=black, subgriddiv=1](0,0)(-6,-2)(14,6)
%
\psaxes[labels=none,labelsep=1pt, Dx=10,Dy=10]{-}(0,0)(\xmin,\ymin)(\xmax,\ymax)
%\uput[dl](0,0){$O$}
%\pcline[linewidth=1pt]{->}(0,0)(1,0) \uput[d](0.5,0){\small $\vec i$}
%\pcline[linewidth=1pt]{->}(0,0)(0,1) \uput[l](0,0.5){\small $\vec j$}

\psdots(0,0)(2,0)(0,1)
\uput[dr](0,0){$O$}
\uput[dr](2,0){$I$}
\uput[ul](0,1){$J$}


\end{pspicture*}&
\psset{xunit=0.5cm,yunit=0.5cm}
\def\xmin{-4} \def\xmax{6} \def\ymin{-4} \def\ymax{6}
\begin{pspicture*}(\xmin,\ymin)(\xmax,\ymax)
%\psgrid[griddots=10,gridlabels=0pt,gridwidth=.3pt, gridcolor=black, subgridwidth=.3pt, subgridcolor=black, subgriddiv=1](0,0)(-6,-2)(14,6)
%
%\psaxes[labels=none,labelsep=1pt, Dx=10,Dy=10]{-}(0,0)(\xmin,\ymin)(\xmax,\ymax)
%\uput[dl](0,0){$O$}
%\pcline[linewidth=1pt]{->}(0,0)(1,0) \uput[d](0.5,0){\small $\vec i$}
%\pcline[linewidth=1pt]{->}(0,0)(0,1) \uput[l](0,0.5){\small $\vec j$}

\psdots(0,0)(2,0)(1,1.732050808)
\uput[dr](0,0){$O$}
\uput[dr](2,0){$I$}
\uput[ul](1,1.732050808){$J$}
\psplot[algebraic=true]{\xmin}{\xmax}{0}
\psplot[algebraic=true]{\xmin}{\xmax}{1.732050808*x}
\end{pspicture*}&
\psset{xunit=0.5cm,yunit=0.5cm}
\def\xmin{-4} \def\xmax{6} \def\ymin{-4} \def\ymax{6}
\begin{pspicture*}(\xmin,\ymin)(\xmax,\ymax)
%\psgrid[griddots=10,gridlabels=0pt,gridwidth=.3pt, gridcolor=black, subgridwidth=.3pt, subgridcolor=black, subgriddiv=1](0,0)(-6,-2)(14,6)
%
\psaxes[labels=none,labelsep=1pt, Dx=10,Dy=10]{-}(0,0)(\xmin,\ymin)(\xmax,\ymax)
%\uput[dl](0,0){$O$}
%\pcline[linewidth=1pt]{->}(0,0)(1,0) \uput[d](0.5,0){\small $\vec i$}
%\pcline[linewidth=1pt]{->}(0,0)(0,1) \uput[l](0,0.5){\small $\vec j$}

\psdots(0,0)(1,0)(0,1)
\uput[dl](0,0){$O$}
\uput[dr](1,0){$I$}
\uput[ul](0,1){$J$}


\end{pspicture*}
\end{tabular}

\subsection{Coordonn\'ees du milieu d'un segment}

\begin{act}
Sur le schéma ci-dessous :
\begin{enumerate}
	\item Placer les points $M(3;1)$, $N(-1;1,5)$, $P(-2;-1)$ et $Q(3;-1)$ ;
	\item Donner graphiquement les coordonnées des points $A$, $B$, $C$ et $D$ ;
	\item En faisant quelques essais, conjecturer le lien existant entre les coordonnées de deux points et les coordonnées du milieu de ces deux points.
\end{enumerate}
\end{act}
\begin{center}

\psset{xunit=1cm , yunit=1cm}
\begin{pspicture*}(-0.7,-0.7)(15.7,9.7)
\def\xmin{-0.5} \def\xmax{15.5} \def\ymin{-0.5} \def\ymax{9.5}

\multido{\i=0+1}{18}{%
\psline[linestyle=dotted](\i,\ymin)(\i,\ymax)
\psset{algebraic=true}
\psplot[linestyle=dotted]{\xmin}{\xmax} {x*0.5-8+\i}
\psplot[linestyle=dotted]{\xmin}{\xmax} {-x*0.5+16-\i}}
%\psline{->}(7,4.5)(7,5.5)
\psline(7,\ymin)(7,\ymax)
%\rput{90}(7.3,5){$\vec{i}$}
%\psline{->}(7,4.5)(5,5.5)
\psplot[algebraic=true]{\xmin}{\xmax} {-0.5*x+8}
%\rput{90}(6,4.5){$\vec{j}$}
\psdots[dotstyle=x, dotscale=2.0000](7,4.5)(7,2.5)(8,6)(5,7.5)(3,6.5)(7,5.5)(5,5.5)
\rput{90}(7.2,4.2){$O$}
\rput{90}(7.2,2.2){$B$}
\rput{90}(8.2,5.7){$A$}
\rput{90}(5.2,7.2){$C$}
\rput{90}(3.2,6.2){$D$}
\rput{90}(7.2,5.2){$I$}
\rput{90}(5.2,5.2){$J$}
\end{pspicture*}
\end{center}


\begin{prop}
 Soit $P$ un plan muni d'un rep\`ere quelconque.\\
 Soit $A(x_A;y_A)$ et $B(x_B;y_B)$ et $I(x_I;y_I)$ milieu de $[AB]$. \\
Alors
\vspace{-1em}\begin{itemize}\begin{multicols}{2}
	\item $x_I=\ldots\ldots\ldots$
	\item $y_I=\ldots\ldots\ldots$
\end{multicols}\end{itemize}
\end{prop}

\begin{proof} La preuve sera faite en classe \`a partir de cette figure :
\begin{center}
 \psset{xunit=0.5cm,yunit=0.5cm}
\def\xmin{-2} \def\xmax{15} \def\ymin{-2} \def\ymax{10}
\begin{pspicture*}(\xmin,\ymin)(\xmax,\ymax)
%\psgrid[griddots=10,gridlabels=0pt,gridwidth=.3pt, gridcolor=black, subgridwidth=.3pt, subgridcolor=black, subgriddiv=1](0,0)(-6,-2)(14,6)
%
%\psaxes[labels=none,labelsep=1pt, Dx=10,Dy=10]{-}(0,0)(\xmin,\ymin)(\xmax,\ymax)
%\uput[dl](0,0){$O$}
%\pcline[linewidth=1pt]{->}(0,0)(1,0) \uput[d](0.5,0){\small $\vec i$}
%\pcline[linewidth=1pt]{->}(0,0)(0,1) \uput[l](0,0.5){\small $\vec j$}

\psplot[algebraic=true]{\xmin}{\xmax}{0}
\psplot[algebraic=true]{\xmin}{\xmax}{2*x}
\psplot[algebraic=true,linestyle=dotted]{\xmin}{\xmax}{7}
\psplot[algebraic=true,linestyle=dotted]{\xmin}{\xmax}{5}
\psplot[algebraic=true,linestyle=dotted]{\xmin}{\xmax}{3}
\psplot[algebraic=true,linestyle=dotted]{\xmin}{\xmax}{2*x-5}
\psplot[algebraic=true,linestyle=dotted]{\xmin}{\xmax}{2*x-11}
\psplot[algebraic=true,linestyle=dotted]{\xmin}{\xmax}{2*x-17}

\psdots(0,0)(6,7)(8,5)(10,3)
\uput[ul](0,0){$O$}
\uput[ur](6,7){$A$}
\uput[ur](8,5){$I$}
\uput[ur](10,3){$B$}

\psline(6,7)(10,3)



\end{pspicture*}                \end{center}
\end{proof}

\subsection{Distance entre deux points dans un rep\`ere orthonorm\'e}

\begin{prop}
 Soit $P$ un plan muni d'un rep\`ere \underline{\textbf{orthonorm\'e}}.\\
 Soient $A$ et $B$ deux points du plan $P$ de coordonn\'ees respectives $(x_A\,;\,y_A)$ et $(x_B\,;\,y_B)$.\\
 Alors la distance $AB$ est donn\'ee par :
 \[AB=\sqrt{(x_B-x_A)^2+(y_B-y_A)^2}\]
\end{prop}

\begin{proof} La preuve sera faite en classe \`a partir de cette figure :
\begin{center}
 \psset{xunit=0.5cm,yunit=0.5cm}
\def\xmin{-2} \def\xmax{15} \def\ymin{-2} \def\ymax{10}
\begin{pspicture*}(\xmin,\ymin)(\xmax,\ymax)
%\psgrid[griddots=10,gridlabels=0pt,gridwidth=.3pt, gridcolor=black, subgridwidth=.3pt, subgridcolor=black, subgriddiv=1](0,0)(-6,-2)(14,6)
%
\psaxes[labels=none,labelsep=1pt, Dx=20,Dy=20]{-}(0,0)(\xmin,\ymin)(\xmax,\ymax)
%\uput[dl](0,0){$O$}
%\pcline[linewidth=1pt]{->}(0,0)(1,0) \uput[d](0.5,0){\small $\vec i$}
%\pcline[linewidth=1pt]{->}(0,0)(0,1) \uput[l](0,0.5){\small $\vec j$}

\psline[linestyle=dotted](6,\ymin)(6,\ymax)

\psline[linestyle=dotted](10,\ymin)(10,\ymax)
\psplot[algebraic=true,linestyle=dotted]{\xmin}{\xmax}{7}

\psplot[algebraic=true,linestyle=dotted]{\xmin}{\xmax}{3}

\psdots(0,0)(6,7)(10,3)
\uput[ul](0,0){$O$}
\uput[ur](6,7){$A$}

\uput[ur](10,3){$B$}

\psline(6,7)(10,3)



\end{pspicture*}                \end{center}
\end{proof}

\sautpage

\section{Exercices et probl\`emes}

\subsection{Rep\`ere donn\'e}

\begin{exo}
Sur le schéma ci-dessous :
\begin{enumerate}
	\item Placer les points $M(2;1)$, $N(-1,5;1)$, $P(-2;-1)$ et $Q(1,5;-1)$ ;
	\item Donner graphiquement les coordonnées des points $A$, $B$, $C$ et $D$ ;
\end{enumerate}
\begin{center}

\psset{xunit=1cm , yunit=1cm}
\begin{pspicture*}(-0.7,-0.7)(15.7,9.7)
\def\xmin{-0.5} \def\xmax{15.5} \def\ymin{-0.5} \def\ymax{9.5}

\multido{\i=0+1}{18}{%
\psline[linestyle=dotted](\i,\ymin)(\i,\ymax)
\psset{algebraic=true}
\psplot[linestyle=dotted]{\xmin}{\xmax} {x*0.5-8+\i}
\psplot[linestyle=dotted]{\xmin}{\xmax} {-x*0.5+16-\i}}
%\psline{->}(7,4.5)(7,5.5)
\psline(7,\ymin)(7,\ymax)
%\rput{90}(7.3,5){$\vec{i}$}
%\psline{->}(7,4.5)(5,5.5)
\psplot[algebraic=true]{\xmin}{\xmax} {0.5*x+1}
%\rput{90}(6,4.5){$\vec{j}$}
\psdots[dotstyle=x, dotscale=2.0000](7,4.5)(7,2.5)(8,6)(5,7.5)(3,6.5)(7,6.5)(6,4)
\rput{90}(7.2,4.2){$O$}
\rput{90}(7.2,2.2){$B$}
\rput{90}(8.2,5.7){$A$}
\rput{90}(5.2,7.2){$C$}
\rput{90}(3.2,6.2){$D$}
\rput{90}(7.2,6.2){$I$}
\rput{90}(6.2,4.2){$J$}
\end{pspicture*}
\end{center}
\end{exo}

\begin{multicols}{2}
\begin{exo}
Le quadrilat\`ere $ABCD$ donn\'e ci-dessous est un losange de centre $O$.\\
 Dans chacun des cas ci-dessous, dire de quel type est le rep\`ere et donner les coordonn\'ees de tous les points dans ce rep\`ere.
\begin{multicols}{2}
\begin{itemize}
 \item $(A,D,B)$
 \item $(O,C,B)$
 \item $(O,B,C)$
 \item $(D,C,O)$
\end{itemize}
\end{multicols}

\begin{center}
\psset{xunit=0.5cm , yunit=0.5cm}
\def\xmin{-6.1} \def\xmax{6.1} \def\ymin{-4.1} \def\ymax{4.1}
\begin{pspicture*}(\xmin,\ymin)(\xmax,\ymax)

\psline(-5,0)(0,3)(5,0)(0,-3)(-5,0)
\psline[linestyle=dotted](-5,0)(5,0)
\psline[linestyle=dotted](0,-3)(0,3)
\uput[dl](0,0){$O$}
\uput[l](-5,0){$A$}
\uput[u](0,3){$B$}
\uput[r](5,0){$C$}
\uput[d](0,-3){$D$}
\end{pspicture*}
\end{center}

\end{exo}

\sautcol

\begin{exo}
 Le plan est muni d'un rep\`ere orthonorm\'e.\\
 Dans chacun des cas suivants, d\'eterminer la nature du triangle $ABC$.
\begin{enumerate}
 \item $A\,(-4\,;\,-1)$, $B\,(4\,;\,-2)$ et $C\,(-2\,;\,2)$
 \item $A\,(-5\,;\,0)$, $B\,(3\,;\,-4)$ et $C\,(2\,;\,4)$
 \item $A\,(0\,;\,0)$, $B\,(4\,;\,2\sqrt{3})$ et $C\,(-1\,;\,3\sqrt{3})$
\end{enumerate}

\end{exo}

\begin{exo}
 Dans le rep\`ere orthonorm\'e $(O,I,J)$, on donne $A\,(-1\,;\,2)$, $B\,(-3\,;\,-1)$ et $C\,(5\,;\,-2)$.
\begin{enumerate}
 \item Quelle est la nature du triangle $ABC$ ?
 \item Montrer que :
       \begin{enumerate}
        \item Le p\'erim\`etre $p$ de $ABC$ vaut $\sqrt{13}(3+\sqrt{5})$ ;
	\item L'aire $a$ de $ABC$ est un nombre entier.
       \end{enumerate}

\end{enumerate}

\end{exo}

\sautpage

\begin{exo}
Dans le rep\`ere orthonorm\'e $(O,I,J)$, on donne $A\,(1\,;\,1)$, $B\,(4\,;\,5)$ et $C\,(10\,;\,8)$.
\begin{enumerate}
 \item D\'eterminer les longueurs $AB$, $AC$ et $BC$.
 \item Que peut-on en d\'eduire pour les points $A$, $B$ et $C$ ?
\end{enumerate}

\end{exo}

\begin{exo}
 Le plan est muni d'un rep\`ere quelconque.\\
  On donne les points $A\,(2\,;\,3)$ et $I\,(-4\,;\,1)$.\\
  On sait que $I$ est le milieu de $[AB]$.\\
 D\'eterminer les coordonn\'ees de $B$.
\end{exo}

%\sautcol




\begin{exo}
 Dans le rep\`ere orthonorm\'e $(O,I,J)$, on a trac\'e le cercle $\mathcal{C}$ de centre $O$ et de rayon 3.
\begin{enumerate}
 \item $A$ est le point de $\mathcal{C}$ d'abscisse 2.\\
D\'eterminer l'ordonn\'ee de $A$.
 \item Pour chacun des points suivants, d\'eterminer, par le calcul, s'il est sur le cercle $\mathcal{C}$ et, sinon, s'il est sur le disque d\'elimit\'e par $\mathcal{C}$.
  %\vspace{-1em}\begin{multicols}{3}
  \begin{itemize}
   \item $B\,(-1\,;\,-2,8)$
   \item $C\,(2,5\,;\,-1,7)$
   \item $D\,(-1,5\,;\,2,5)$
  \end{itemize}%\end{multicols}
\end{enumerate}



\begin{center}
\psset{unit=0.5cm}
\def\xmin{-4.1} \def\xmax{4.1} \def\ymin{-4.1} \def\ymax{4.1}
\begin{pspicture*}(\xmin,\ymin)(\xmax,\ymax)
\psgrid[griddots=10,gridlabels=0pt,gridwidth=.3pt, gridcolor=black, subgridwidth=.3pt, subgridcolor=black, subgriddiv=1](0,0)(\xmin,\ymin)(\xmax,\ymax)
\psaxes[labels=none,labelsep=1pt, Dx=10,Dy=10]{-}(0,0)(\xmin,\ymin)(\xmax,\ymax)
\psdots(0,0)(1,0)(0,1)(2,2.2)
\pscircle(0,0){3}
\uput[dl](0,0){$O$}
\uput[d](1,0){$I$}
\uput[l](0,1){$J$}

\uput[u](2,2.2){$A$}
\end{pspicture*}
\end{center}

\end{exo}


\begin{exo}\label{rep1}
Sur le schéma ci-contre :
\begin{enumerate}
	\item Placer les points $A(1\,;\,2)$, $B(3\,;\,1,5)$, $C(4\,;\,-0,5)$ et $D(2\,;\,0)$ ;
	\item Montrer que le quadrilat\`ere $ABCD$ est un parall\'elogramme.
\end{enumerate}



\begin{center}

\psset{xunit=1cm , yunit=1cm}
\def\xmin{-1.1} \def\xmax{6.1} \def\ymin{-1.1} \def\ymax{6.1}
\begin{pspicture*}(\xmin,\ymin)(\xmax,\ymax)


%\psgrid[griddots=10,gridlabels=0pt,gridwidth=.3pt, gridcolor=black, subgridwidth=.3pt, subgridcolor=black, subgriddiv=1](0,0)(\xmin,\ymin)(\xmax,\ymax)
%\psaxes[labels=all,labelsep=1pt, Dx=10,Dy=10]{-}(0,0)(\xmin,\ymin)(\xmax,\ymax)


\multido{\i=0+1}{18}{%
\psline[linestyle=dotted](\i,\ymin)(\i,\ymax)
\psset{algebraic=true}
\psplot[linestyle=dotted]{\xmin}{\xmax} {x*0.5-8+\i}
\psplot[linestyle=dotted]{\xmin}{\xmax} {-x*0.5+16-\i}}
%\psline{->}(7,4.5)(7,5.5)
\psline(0,\ymin)(0,\ymax)
%\rput{90}(7.3,5){$\vec{i}$}
%\psline{->}(7,4.5)(5,5.5)
\psplot[algebraic=true]{\xmin}{\xmax} {0.5*x}
%\rput{90}(6,4.5){$\vec{j}$}
\psdots[dotstyle=x, dotscale=2.0000](0,0)(1,0.5)(0,2)
\uput[ul](0,0){$O$}
\uput[ul](1,0.5){$I$}
\uput[l](0,2){$J$}
\end{pspicture*}
\end{center}
\end{exo}

%\sautpage

\begin{exo}
Le plan est muni d'un rep\`ere orthonorm\'e.\\
Dans chacun des cas suivants, d\'eterminer la nature du quadrilat\`ere $ABCD$ :
\begin{enumerate}
 \item $A\,(1\,;\,0)$, $B\,(1\,;\,3)$, $C\,(2\,;\,3)$ et $D\,(3\,;\,1)$ ;
 \item $A\,(1\,;\,2)$, $B\,(4\,;\,7)$, $C\,(1\,;\,6)$ et $D\,(-2\,;\,1)$ ;
 \item $A\,(1\,;\,0)$, $B\,(0\,;\,2)$, $C\,(4\,;\,4)$ et $D\,(5\,;\,2)$ ;
 \item $A\,(-4\,;\,-1)$, $B\,(4\,;\,-2)$, $C\,(8\,;\,5)$ et $D\,(0\,;\,6)$ ;
 \item $A\,(0\,;\,-2)$, $B\,(3\,;\,-1)$, $C\,(2\,;\,2)$ et $D\,(-1\,;\,1)$.
\end{enumerate}

\end{exo}

%\sautpage

\begin{exo}
 Le plan est muni d'un rep\`ere orthonorm\'e.\\
 On donne les points $A\,(-3\,;\,-4)$, $B\,(3\,;\,2)$, $C\,(7\,;\,-2)$ et $D\,(1\,;\,-8)$.
\begin{enumerate}
 \item Montrer que :
	\begin{enumerate}
	 \item $[AC]$ et $[BD]$ ont m\^eme milieu ;
	 \item $AC=BD$.
	\end{enumerate}
 \item \begin{enumerate}
        \item Quelle est la nature du quadrilat\`ere $ABCD$ ?
	\item Calculer le rayon du cercle circonscrit \`a ce quadrilat\`ere.
       \end{enumerate}


\end{enumerate}

\end{exo}



\begin{exo}\label{rep2}
 Le plan est muni d'un rep\`ere quelconque.\\
 On donne les points $A\,(-5\,;\,3)$, $B\,(-4\,;\,1)$ et $C\,(1\,;\,-4)$.
\begin{enumerate}
 \item D\'eterminer les coordonn\'ees de $I$, milieu de $[AC]$.
 \item D\'eterminer les coordonn\'ees de $D$ tel que $ABCD$ soit un parall\'elogramme.
\end{enumerate}

\end{exo}

\subsection{Rep\`ere \`a choisir}

\begin{exo}
 $ABCD$ est un parall\'elogramme, $I$ est le milieu de $[AD]$, $E$ est le sym\'etrique de $B$ par rapport \`a $I$.
\begin{enumerate}
 \item Faire une figure.
 \item Choisir un rep\`ere et montrer que $D$ est le milieu de $[EC]$
\end{enumerate}
\end{exo}

\begin{exo}
 $ABCD$ est un rectangle tel que $AB=8$\,cm et $AD=5$\,cm.\\
 $J$ est le point de $[AD]$ tel que $AJ=3$\,cm.\\
 $M$ est le point de $[AB]$ tel que $AM=3$\,cm.
\begin{enumerate}
 \item Faire une figure.
 \item Choisir un rep\`ere et d\'eterminer la nature du triangle $MJC$.
\end{enumerate}

\end{exo}

\begin{exo}
 $BOIS$ est un carr\'e de c\^ot\'e 12\,cm.\\
 $P$ est le milieu de $[BS]$ et $N$ est le point de $[BO]$ tel que $BN=3$\,cm.
 \begin{enumerate}
 \item Faire une figure.
 \item Choisir un rep\`ere et d\'eterminer si le triangle $PIN$ est rectangle.
\end{enumerate}
\end{exo}

\begin{exo}
 $ABCD$ est un carr\'e de c\^ot\'e 4\,cm.\\
 $E$ et $F$ sont les milieux respectifs de $[AB]$ et $[OD]$.
 \begin{enumerate}
 \item Faire une figure.
 \item Choisir un rep\`ere et d\'emontrer que le triangle $CFE$ est rectangle et isoc\`ele.
\end{enumerate}
\end{exo}


\subsection{Algorithmique}
\begin{exo}
 Que fait l'algorithme suivant ?
\begin{algo}
 \begin{verbatim}
  VARIABLES
    a, b, c, d, e, f : nombres
  DEBUT
    Saisir a
    Saisir b
    Saisir c
    Saisir d
    e prend la valeur (a+c)/2
    f prend la valeur (b+d)/2
    Afficher e
    Afficher f
  FIN
 \end{verbatim}

\end{algo}

\end{exo}

\begin{exo}
 \'Ecrire un algorithme prenant comme arguments les coordonn\'ees de deux points et retournant la distance entre ces deux points dans un rep\`ere orthonorm\'e.
\end{exo}


\begin{exo}
\'Ecrire un algorithme prenant comme arguments les coordonnées de trois points A, B et C et dessinant le triangle ABC dans un repère.
\end{exo}

\begin{exo}
\'Ecrire un algorithme prenant comme arguments les coordonnées de trois points A, B et C, calculant  les coordonnées du point D tel que ABCD est un parallélogramme et dessinant ABCD dans un repère.
\end{exo}

\begin{exo}
\'Ecrire un algorithme prenant comme argument les coordonnées de trois points A, B et C dans un rep\`ere orthonorm\'e
\begin{enumerate}
\item et indiquant si le triangle est iso\`ele.
\item et indiquant si le triangle est \'equilat\'eral.
\item et indiquant si le triangle est rectangle.
\item et indiquant la nature du triangle.
\end{enumerate}
\end{exo}

\end{multicols}




